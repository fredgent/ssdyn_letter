%% using aastex version 6.3
\documentclass[preprint2]{aastex63}

%% The default is a single spaced, 10 point font, single spaced article.
%% There are 5 other style options available via an optional argument. They
%% can be invoked like this:
%%
%% \documentclass[arguments]{aastex63}
%% 
%% where the layout options are:
%%
%%  twocolumn   : two text columns, 10 point font, single spaced article.
%%                This is the most compact and represent the final published
%%                derived PDF copy of the accepted manuscript from the publisher
%%  manuscript  : one text column, 12 point font, double spaced article.
%%  preprint    : one text column, 12 point font, single spaced article.  
%%  preprint2   : two text columns, 12 point font, single spaced article.
%%  modern      : a stylish, single text column, 12 point font, article with
%% 		  wider left and right margins. This uses the Daniel
%% 		  Foreman-Mackey and David Hogg design.
%%  RNAAS       : Preferred style for Research Notes which are by design 
%%                lacking an abstract and brief. DO NOT use \begin{abstract}
%%                and \end{abstract} with this style.
%%
%% Note that you can submit to the AAS Journals in any of these 6 styles.
%%
%% There are other optional arguments one can invoke to allow other stylistic
%% actions. The available options are:
%%
%%   astrosymb    : Loads Astrosymb font and define \astrocommands. 
%%   tighten      : Makes baselineskip slightly smaller, only works with 
%%                  the twocolumn substyle.
%%   times        : uses times font instead of the default
%%   linenumbers  : turn on lineno package.
%%   trackchanges : required to see the revision mark up and print its output
%%   longauthor   : Do not use the more compressed footnote style (default) for 
%%                  the author/collaboration/affiliations. Instead print all
%%                  affiliation information after each name. Creates a much 
%%                  longer author list but may be desirable for short 
%%                  author papers.
%% twocolappendix : make 2 column appendix.
%%   anonymous    : Do not show the authors, affiliations and acknowledgments 
%%                  for dual anonymous review.
%%
%% these can be used in any combination, e.g.
%%
%% \documentclass[twocolumn,linenumbers,trackchanges]{aastex63}
%%
%% AASTeX v6.* now includes \hyperref support. While we have built in specific
%% defaults into the classfile you can manually override them with the
%% \hypersetup command. For example,
%%
%% \hypersetup{linkcolor=red,citecolor=green,filecolor=cyan,urlcolor=magenta}
%%
%% will change the color of the internal links to red, the links to the
%% bibliography to green, the file links to cyan, and the external links to
%% magenta. Additional information on \hyperref options can be found here:
%% https://www.tug.org/applications/hyperref/manual.html#x1-40003
%%
%% Note that in v6.3 "bookmarks" has been changed to "true" in hyperref
%% to improve the accessibility of the compiled pdf file.
%%
%% If you want to create your own macros, you can do so
%% using \newcommand. Your macros should appear before
%% the \begin{document} command.
%%
\usepackage{xfrac}
\usepackage{amsmath}
\usepackage{upgreek}

\newcommand{\vdag}{(v)^\dagger}
\newcommand\aastex{AAS\TeX}
\newcommand\latex{La\TeX}
\newcommand\Rm{{\rm Rm} }
\newcommand\Rey{{\rm Re} }
\newcommand\Pm{{\rm Pm} }
\newcommand\kf{k_{\rm f} }
\newcommand\SNr{\dot\sigma_{\rm sn}}
\newcommand\OSN{\Omega_{\rm sn}}
\newcommand\ESK{E_{\rm kin}}
\newcommand\EST{E_{\rm th}}
\newcommand\ESN{E_{\sigma}}
\newcommand{\vect}[1]{{{\mbox{\boldmath $#1$}}}}%also makes bold Greek letters
\newcommand{\mathbfss}[1]{\textbf{\textsf{#1}}}
\newcommand\kpc{~ {\rm kpc}}
\newcommand\pc{~ {\rm pc}}
\newcommand\dx{~ {\delta x}}
\newcommand\Myr{~ {\rm Myr}}
\newcommand\erg{~ {\rm erg}}
\newcommand\kms{~ {\rm km~ s}^{-1}}

\definecolor{midblue}{rgb}{0.0,0.4,0.7}
\definecolor{midgreen}{rgb}{0.0,0.8,0.3}
\definecolor{mypurple}{rgb}{0.8,0.2,0.8}
\newcommand{\fg}[1]{\textcolor{midblue}{#1}}
\newcommand{\fag}[1]{\textcolor{midgreen}{FAG: #1}}
%\newcommand{\ns}[1]{\textcolor{orange}{NS: #1}}
\newcommand{\ns}[1]{\textcolor{orange}{#1}}

%% Reintroduced the \received and \accepted commands from AASTeX v5.2
\received{June 1, 2019}
\revised{January 10, 2019}
\accepted{\today}
%% Command to document which AAS Journal the manuscript was submitted to.
%% Adds "Submitted to " the argument.
\submitjournal{ApJL}

%% For manuscript that include authors in collaborations, AASTeX v6.3
%% builds on the \collaboration command to allow greater freedom to 
%% keep the traditional author+affiliation information but only show
%% subsets. The \collaboration command now must appear AFTER the group
%% of authors in the collaboration and it takes TWO arguments. The last
%% is still the collaboration identifier. The text given in this
%% argument is what will be shown in the manuscript. The first argument
%% is the number of author above the \collaboration command to show with
%% the collaboration text. If there are authors that are not part of any
%% collaboration the \nocollaboration command is used. This command takes
%% one argument which is also the number of authors above to show. A
%% dashed line is shown to indicate no collaboration. This example manuscript
%% shows how these commands work to display specific set of authors 
%% on the front page.
%%
%% For manuscript without any need to use \collaboration the 
%% \AuthorCollaborationLimit command from v6.2 can still be used to 
%% show a subset of authors.
%
%\AuthorCollaborationLimit=2
%
%% will only show Schwarz & Muench on the front page of the manuscript
%% (assuming the \collaboration and \nocollaboration commands are
%% commented out).
%%
%% Note that all of the author will be shown in the published article.
%% This feature is meant to be used prior to acceptance to make the
%% front end of a long author article more manageable. Please do not use
%% this functionality for manuscripts with less than 20 authors. Conversely,
%% please do use this when the number of authors exceeds 40.
%%
%% Use \allauthors at the manuscript end to show the full author list.
%% This command should only be used with \AuthorCollaborationLimit is used.

%% The following command can be used to set the latex table counters.  It
%% is needed in this document because it uses a mix of latex tabular and
%% AASTeX deluxetables.  In general it should not be needed.
%\setcounter{table}{1}

%%%%%%%%%%%%%%%%%%%%%%%%%%%%%%%%%%%%%%%%%%%%%%%%%%%%%%%%%%%%%%%%%%%%%%%%%%%%%%%%
%%
%% The following section outlines numerous optional output that
%% can be displayed in the front matter or as running meta-data.
%%
%% If you wish, you may supply running head information, although
%% this information may be modified by the editorial offices.
\shorttitle{Small-scale dynamo in the ISM}
\shortauthors{Gent et al.}
%%
%% You can add a light gray and diagonal water-mark to the first page 
%% with this command:
%% \watermark{text}
%% where "text", e.g. DRAFT, is the text to appear.  If the text is 
%% long you can control the water-mark size with:
%% \setwatermarkfontsize{dimension}
%% where dimension is any recognized LaTeX dimension, e.g. pt, in, etc.
%%
%%%%%%%%%%%%%%%%%%%%%%%%%%%%%%%%%%%%%%%%%%%%%%%%%%%%%%%%%%%%%%%%%%%%%%%%%%%%%%%%

%% This is the end of the preamble.  Indicate the beginning of the
%% manuscript itself with \begin{document}.

\begin{document}

\title{Small-scale dynamo in supernova driven ISM turbulence}

%%
%% The \author command is the same as before except it now takes an optional
%% argument which is the 16 digit ORCID. The syntax is:
%% \author[xxxx-xxxx-xxxx-xxxx]{Author Name}
%%
%%
%% Use \affiliation for affiliation information. The old \affil is now aliased
%% to \affiliation. AASTeX v6.3 will automatically index these in the header.
%% When a duplicate is found its index will be the same as its previous entry.
%%
%% Note that \altaffilmark and \altaffiltext have been removed and thus 
%% can not be used to document secondary affiliations. If they are used latex
%% will issue a specific error message and quit. Please use multiple 
%% \affiliation calls for to document more than one affiliation.
%%
%% The new \altaffiliation can be used to indicate some secondary information
%% such as fellowships. This command produces a non-numeric footnote that is
%% set away from the numeric \affiliation footnotes.  NOTE that if an
%% \altaffiliation command is used it must come BEFORE the \affiliation call,
%% right after the \author command, in order to place the footnotes in
%% the proper location.
%%
%% Use \email to set provide email addresses. Each \email will appear on its
%% own line so you can put multiple email address in one \email call. A new
%% \correspondingauthor command is available in V6.3 to identify the
%% corresponding author of the manuscript. It is the author's responsibility
%% to make sure this name is also in the author list.
%%
%% While authors can be grouped inside the same \author and \affiliation
%% commands it is better to have a single author for each. This allows for
%% one to exploit all the new benefits and should make book-keeping easier.
%%
%% If done correctly the peer review system will be able to
%% automatically put the author and affiliation information from the manuscript
%% and save the corresponding author the trouble of entering it by hand.

\correspondingauthor{Frederick Gent}
\email{frederick.gent@aalto.fi}
\email{mordecai@amnh.org}
\email{maarit.kapyla@aalto.fi}
\email{nishant@iucaa.in}

\author[0000-0002-1331-2260]{Frederick A. Gent}
\affiliation{
Astroinformatics, Department of Computer Science, Aalto University, PO Box 15400, FI-00076 Aalto, Finland
 }
\affiliation{
    School of Mathematics, Statistics and Physics,
      Newcastle University, NE1 7RU, UK 
 }

\author[0000-0003-0064-4060]{Mordecai-Mark {Mac Low}}
\affiliation{
American Museum of Natural History, 79th Street at Central Park West, New York, NY 10024, USA
}
\affiliation{
{Center for Computational Astrophysics, Flatiron Institute, New York,
NY 10010, USA} 
}

\author[0000-0002-9614-2200]{Maarit J. K\"apyl\"a}
\affiliation{
Astroinformatics, Department of Computer Science, Aalto University, PO Box 15400, FI-00076 Aalto, Finland
}
\affiliation{
Max Planck Institute for Solar System Research, Justus-von-Liebig-Weg 3, 37707 G\"ottingen, Germany
}
\affiliation{
    Nordic Institute for Theoretical Physics,
      Roslagstullsbacken 23, 106 91 Stockholm, Sweden 
}

\author[0000-0001-6097-688X]{Nishant K. Singh}
\affiliation{
Inter-University Centre for Astronomy \& Astrophysics, Post Bag 4, Ganeshkhind, Pune 411 007, India
}
\affiliation{
Max Planck Institute for Solar System Research, Justus-von-Liebig-Weg 3, 37707 G\"ottingen, Germany
}

%% Note that the \and command from previous versions of AASTeX is now
%% depreciated in this version as it is no longer necessary. AASTeX 
%% automatically takes care of all commas and "and"s between authors names.

%% AASTeX 6.3 has the new \collaboration and \nocollaboration commands to
%% provide the collaboration status of a group of authors. These commands 
%% can be used either before or after the list of corresponding authors. The
%% argument for \collaboration is the collaboration identifier. Authors are
%% encouraged to surround collaboration identifiers with ()s. The 
%% \nocollaboration command takes no argument and exists to indicate that
%% the nearby authors are not part of surrounding collaborations.

%% Mark off the abstract in the ``abstract'' environment. 
\begin{abstract}

%We present a numerical resolution study over a range of magnetic Reynolds
%number (Rm) to identify characteristics of the small-scale dynamo (SSD) under
%FAG: Reynolds -> resistivity
%MJK Currently Rms are not given, not listed, nor any results
%MJK presented as function of Rm, hence this statement cannot
%MJK remain, unless such analysis is added.
We present a numerical resolution study over a range of magnetic resistivity
(and by implication magnetic Reynolds number, Rm) to identify characteristics
of the small-scale dynamo (SSD) under supernova-driven turbulence of the
interstellar medium (ISM).
We confirm that conditions in the ISM are likely highly conducive to SSD in the
diffuse hot and warm turbulent ISM, given Rm in the real ISM is orders of
magnitude higher than we model at our highest resolution and a modest SN
%explosion frequency ($\SNr$), 20\% that of the solar neighbourhood.
%FAG: relabelling
explosion frequency, $\dot\sigma$, at 20\% that of the solar neighbourhood ($\SNr$).
This is despite the negative impact on SSD expected from the highly compressible
%MJK So we have varying SN rates; do these have a higher Ma? What I am after is whether we can quantify that statement?
%FAG: yes in ssdyn_paper I now added tables for sigma=\SNr and mean Ms increases
%FAG: from 0.4/0.5 at 2pc/4pc to 0.6/0.8, even though urms is similar for each resolution.
%MJK The other aspect is that we have LOW or unity for Pm, while the ISM
%MJK has high Pm. SSD should be easier for higher Pm flows.
%FAG. now include para in section 3.
%FAG: added
%NS: should we say "... in *some of* our experiments." in the line below?
%NS: I thought there are also high Pm cases which might be included as more relevant here?
%MJK No, unfortunately not. We have runs with nu=0, meaning nu_SGS only,
%MJK and eta=0, and increasing from there. These are effectively low Pm 
%MJK cases.
%NS: Oh, ok. Then I will perhaps suppress high Pm citations from sec 2.
%FAG: keep citations as I now reference runs to follow with high Pm.
%MJK Fred now has some higher Pm runs existing, but we do not have space to show a picture of them. We will just add a sentence somehwere that they show the expected behavior (with your ref) and then state that we see it too.
nature of the flow and low magnetic Prandtl number of our experiments.
%FAG: added in response to above
Dynamo growth rates for $\dot\sigma=\SNr$, relative to models with
$\dot\sigma=0.2\SNr$, are even greater despite higher Mach numbers.
%
We find convergence in the growth rate of the SSD with resolution for a given 
$\dot\sigma$ approaching 
sub-parsec scale. %FAG moved and reworded below:
% and a trend towards the suppression of SSD for resolution
%more coarse than 4 parsecs, where higher resistivity is required to resolve
%the magnetic field.
%MJK Also the following statement is not in line with the previous one with exclusion.
%Across the modelled range of 0.5 to 4 parsec resolution we find the SSD 
%FAG: added low resistivity
Across the modelled range of 0.5 to 4 parsec resolution we find with
sufficiently low resistivity the SSD saturates consistently at about 5\% of
the energy equipartion level, independent of the growth rate.
%FAG: Reworked from previous sentence
With higher resistivity required to resolve the magnetic field as resolution becomes
more coarse the trend suggests the small scale dynamo cannot be excited for grid
spacing much more than above 4\,pc.

\end{abstract}

%% Keywords should appear after the \end{abstract} command. 
%% See the online documentation for the full list of available subject
%% keywords and the rules for their use.
\keywords{Dynamo --- Magnetohydrodynmaical simulations --- Supernova dynamics --- Interstellar magnetic fields --- Turbulence}

\section{Introduction} \label{sec:intro}

%MJK The landscape to me seems to be the following:
%MJK GE20 do not explicitely state that they do not have SSD, but do go forward with QKTFM, which assumes that no magnetic background turbulence is present, hence they make an implicit claim of having no SSD.
%MJK Balsara04 clearly had a SSD, as LSD was not allowed for, but had an imposed
%MJK field, and the role of that is unclear.
%MJK Fred+gang claim LSD and SSD, but it is unclear whether this is true, as
%MJK separating SSD and LSD is difficult.
%MJK Steinwandel et al. have recently claimed that a SSD is present in a
%MJK galactic-scale simulation, but no LSD, and that SSD "dies" off at later
%MJK stages, when Kazantsev scaling is not observed.
%MJK So, our study now tries to address at which resolution a fully healthy
%MJK SSD can be expected in a full ISM simulation, and helps to interpret
%MJK the mess descripbed above.
%MJK Now we DO not have imposed field, and can confirm the conclusions of Balsara04
%MJK without it.
%MJK It would be interesting if a critical Rm could be nailed down, but I doubt we
%MJK can, and must resort to what is listed above.
%FAG 28.9 reword
\fag{
\begin{itemize}
\item
    Abstract – no more than 250 words
\item
    Main Text – no more than 3500 words (not including appendices or other supplementary material)
\item
    Figures and Tables – no more than 5 combined figures (each limited to 9 panels) and tables, e.g. 3 figures and 2 tables.
\item
    References – no more than 50 references
\end{itemize}}

This letter addresses the necessary conditions for and nature of 
small-scale dynamo (SSD) in the interstellar medium.
SSD are dynamo modes acting at small eddy scales of the turbulence.

 to support
the turbulence as distinct from large scale dynamo, which grows the field at 
the systemic scales of the galactic disk, spiral arm, or similar global 
structure.
recent results of \citet{GE20}, which asserts the 
absence of a small scale (fluctuation) dynamo (SSD) in the interstellar medium
(ISM) subject to supernova induced turbulence and multi-phase structure
applicable to the solar neighbourhood. 
They refute the earlier dynamo solutions of \citet{BKMM04}

In many simulations of supernova driven turbulence with realistic vertical 
stratification \citep[e.g.,][]{deAvillez:2005,PO07,Hill:2012a,HI14} the mechanisms
for inducing large scale magnetic fields are absent, such as rotation and shear.
To examine the effect of strong ordered magnetic fields these models then rely
on imposition a background or initial magnetic field, typically uniform,
which is then perturbed by supernova explosions.
The amplification of the magnetic field in these cases may include a small
scale dynamo in case of sufficently high magnetic Reynolds numbers, but any
amplification of the large scale field is limited to linear growth through
tangling of the imposed field.
Any initial large scale field would diffuse leaving only the random field. 
If an imposed field is sufficiently close to equipartition its characteristics
would dominate any MHD results.

More commonly models of supernova driven dynamo in the ISM have ignored the 
large scale effects and examined the small scale dynamo or the effect of the
turbulent magnetic field on the properties of the ISM
\citep[e.g.,][]{BKMM04,BalKim05,MacLow:2005}.
A seed or imposed field for such models could be uniform or random, but if the
simulated field results from tangling rather than dynamo, then any conclusions
about the ISM depend on the veracity of the chosen field rather than the 
nature of the MHD turbulence.

%MJK Part of this belongs to the introduction (first para, at least). Here we should only summarize the model used here, and differences and similarities to earlier work and our previous models. 
While in the SN experiments considered in this letter are restricted to
SSD, some large scale models do include the large scale dynamo
\citep[e.g.,][]{Gressel:2008,HWK09,WA09,Gent:2013b,EGSFB16,Pakmor17,SBADMN19,SDLMBP20,GE20}.
Most of these do not include a small scale dynamo.
\citet{Gent:2013b,EGSFB16} do include SSD, but without a clear understanding of
the characteristics of this SSD it is difficult to explore the effect this has
on the large scale dynamo (LSD) and the properties of the ISM.
\citet{SBADMN19,SDLMBP20} also suspect they have SSD.
In this letter we explore the SSD in realistic simulations of SN driven 
turbulence.
The resolution and parameter study is intended to identify the critical ranges
for excitation of SSD and understand dynamo growth rates and saturation 
conditions.
This will also identify the prerequisites for including or excluding SSD in the
LSD models and subsequently establish the importance of SSD-LSD interactions 
on galaxy dynamics. 
%MJK The two first paragraphs should go into the intro in some form. 
%MJK This section should basically only contain the toy model results. Toy model is also a bit degrading...
\section{Disentangling the dynamo} \label{sec:ssd-tang}

\begin{figure*}
\gridline{ \fig{figs/ssd-tang-brms.png}{0.45\textwidth}{(a)}
          }
\gridline{     \fig{figs/ssdBpower.png}{0.45\textwidth}{(b)}
          \fig{figs/tanglingBpower.png}{0.45\textwidth}{(c)}
          }
\caption{
%fag
% Comparison of simulations, one with tangling and the other with dynamo
%-amplification of magnetic field, from non-helical random forcing.
%-For the SSD model, $R_{\rm m}=148$ and $P_{\rm m}=50.0$, while the tangling
%-model has the dynamo suppressed by choosing $R_{\rm m}=7.40$ and $P_{\rm m}=2.50$.
%-Panel (a) displays $B_{\rm rms}$ as a proportion of $\overline{B_{\rm Eq}}$,
%
%% mm where $\overline{B_{\rm Eq}}=\overline{u_{\rm rms}}/\sqrt{2}$ is
%the time-averaged equipartion field. 
%% mm and $\overline{u_{\rm rms}}$ is the  time-averaged $u_{\rm rms}$ during the kinematic phase.
Simulation results for non-helical random forcing. 
The simulation with $R_{\rm m}=148.0$ and $P_{\rm m}=50.0$ supports dynamo 
amplification of magnetic field
With $R_{\rm m}=7.4$ and $P_{\rm m}=2.5$ dynamo is suppressed and 
amplification is limited to tangling of the magnetic field.
Panel (a) displays mean magnetic energy density, $e_B$, evolving as a
proportion of time-averaged kinetic energy density, $\overline{e_K}$.
The inset shows a zoom-in of the early linear growth of the tangled field.
%fag
%Time is scaled with $\kf \overline{u_{\rm rms}}$, the eddy turnover
%time at the forcing scale.
Time is normalised by the inverse eddy turnover time at the forcing scale,
$\kf \overline{u_{\rm rms}}$.
%fag
%Power spectra are displayed for magnetic energy of the SSD (b) and
%tangled model (c).  The legend shows the scaled
Compensated power spectra are displayed of magnetic energy for the model with
SSD (b) and with tangling (c).
%fag
%The legend shows the scaled 
The legend shows the normalised  
times for each spectrum.
%fag
%The forcing scale, $\kf=8$, is indicated by the vertical dotted line and the black dashed
The forcing scale, $\kf=8$, is indicated by the vertical dotted line.
$k_1=L/(2\pi)$, where $L$ is the largest length scale in the simulation domain,
with the largest wavenumber $k/k_1=128$.
\label{fig:tangling}}
\end{figure*}

In many simulations of supernova driven turbulence with realistic vertical 
stratification \citep[e.g.,][]{deAvillez:2005,PO07,Hill:2012a,HI14} the mechanisms
for inducing large scale magnetic fields are absent, such as rotation and shear.
To examine the effect of strong ordered magnetic fields these models then rely
on imposition a background or initial magnetic field, typically uniform,
which is then perturbed by supernova explosions.
The amplification of the magnetic field in these cases may include a small
%scale dynamo in case of sufficently high magnetic Reynolds numbers, but any
%NS: minor
scale dynamo in case of sufficiently high magnetic Reynolds numbers, but any
amplification of the 
%MJK
%large scale 
field is limited to linear growth through
tangling of the imposed field.
%\ns{The above sentence seems to suggest (linear) growth of large-scale field,
%but one might argue that tangling produces fields preferably on %the forcing
%scale, thus essentially leading to (linear) growth of fluctuating %field.
%(Aside) Interestingly, I think, it is only the tangling produced %magnetic
%noise (and not the SSD) which leads to quenching of large-scale %dynamo, as
%$\alpha_M$ is sourced primarily by tangling.}
%MJK see above; is it so easily solved, Nishant?
%NS: Yes, agreed. I will keep the comments above for now as I can formulate
%NS: a sentence on the Aside part which could be useful
Any initial large scale field would diffuse leaving only the random field. 
If an imposed field is sufficiently close to equipartition its characteristics
would dominate any MHD results.
%NS: added; pls feel free to decide whether it should be kept here or not
\ns{It is reasonable to expect that the tangling produced magnetic noise
will grow exponentially if the system supports the LSD instability.
Such a noise plays an important role in quenching the LSD by contributing
to the magnetic $\alpha$-effect $\alpha_M$ which leads to the saturation
of large-scale fields in the nonlinear regime.}
%NS.

More commonly models of supernova driven dynamo in the ISM have ignored the 
large scale effects and examined the small scale dynamo or the effect of the
turbulent magnetic field on the properties of the ISM
\citep[e.g.,][]{BKMM04,BalKim05,MacLow:2005}.
A seed or imposed field for such models could be uniform or random, but if the
simulated field results from tangling rather than dynamo, then any conclusions
about the ISM depend on the veracity of the chosen field rather than the 
nature of the MHD turbulence.

To illustrate some differences between tangling and small scale dynamo we adopt
a simplified model with non-helical random forcing with wavenumber $\kf=8$ applied to
isothermal uniform density in $256^3$ $2\pi$-periodic boxes.
A uniform field with energy density $e_B\simeq6\cdot10^{-22}\overline{e_K}$ is
imposed, where $\overline{e_K}$ is the time-averaged kinetic energy density.
The two simulations are distinguished only by use of dimensionless 
$\eta=10^{-4}$, exciting a 
small scale dynamo models, and $\eta=2\cdot10^{-3}$ subcritical for dynamo.
Our numerical implementations use the {\sc Pencil Code}\footnote{
\href{https://github.com/pencil-code}{https://github.com/pencil-code}}.

We plot some of the diagnostics from the results of these two simulations in 
Figure\,\ref{fig:tangling}.
Panel (a) illustrates that the small scale dynamo 
is characterised by exponential growth
%\fag{Nishant, please add some relevant citation to this expected behaviour. MJK: I do not think we need a ref. to the expectation that the dynamo is growing exponentially, but perhaps to the expected powerlaw, where the old Kazantsev paper 1968 should be cited.}, growing over 10 orders of magnitude in 
%NS: commented out the comments, above/below in this thread, and quoted ZRS83 book
%just over 400 eddy turnover times.
%NS: added
just over 400 eddy turnover times\ns{; see \cite{ZRS83} for the properties and excitation
conditions of SSD.}
%\ns{OK, I will add a couple of references, or
%even the book by Zeldovich where $Rm^{\rm crit}$ is also discussed in detail.}
Tangling results only in linear amplification
early on (see inset), saturating within 100 eddy turnover times at below 5 times
the imposed field energy density.

In panel (b) we plot for the SSD model some power spectra for the magnetic
energy, evolving over time alongside a kinetic energy spectrum at late stage.
The magnetic energy spectra are compensated by $k^{-3/2}$ and kinetic by
$k^{5/3}$.
A profile becoming horizontal, therefore, corresponds to the Kazentsev 
\fag{Anyone, a suitable reference for this expected behaviour?}
\ns{Schekochihin+2002 ApJ may be fine, or better, Bhat \& Subramanian 2014 where
3/2 scaling is recovered even in case of finite correlation time} $3/2$ inverse
cascade power law or Kolmogorov turbulent energy cascade.
In panel (c) see we show similar for the tangling model.
The forcing scale $\kf=8$ is evident in the kinetic spectra and highlighted by 
a vertical dotted line.
The SSD magnetic spectrum (b) evolves near self-similarly, with an 
uncompensated peak
wavenumber above 20 at early times reducing to below 20 upon saturation of the
dynamo.
The forcing scale has negligible effect on the magnetic spectra.
However, for the tangling case (c) the peak wavenumber for the magnetic energy
spectrum is strongly identified with the forcing scale.
The Kazentsev range of the spectrum extends to wavenumbers larger than the 
forcing scale for SSD, while confined to larger scales for the field with
tangling only.

\ns{Added 5 new references in refs.bib to quote earlier works on Kazantsev, low/high
Pm SSD; we can include those here as \citet{K68,KA92}; \cite{Sch02} (high Pm);
\cite{Sch07} (low Pm); \cite{BS14} (finite correlation time). Please feel free
to remove any from these that appear unsuitable here.}

Given the high magnetic Prandtl numbers, the magnetic spectra retain more energy
at smaller scales than the kinetic spectra.
Although the kinetic parameters are identical and are subject to the same 
viscous cutoff, the SSD extends the kinetic energy into the larger scales
due to the feedback from the Lorentz force.
With SSD there is a short inertial range for $10\leq k/k_1\lesssim 15$.
Thus, in the dynamo kinetic energy deposited along the Kolmogorov range to
smaller scales transfers energy to the magnetic field at these scales
inducing an inverse Kazentsev range begining at scales below the forcing
scale, while with tangling the energy tranfers to the magnetic field
only at scales between the forcing scale and scale of the imposed field.
There is only dissipation of the field at scales smaller than the Kazentsev 
range.


\section{SN turbulence model design} \label{sec:model}


%MJK We should mention that the domain is not subject to rotation, shear, nor stratification. And also in most cases no imposed field, I guess? 
To exclude any large scale magnetic field dynamics in these simulations, the
computational domain occupies a cube of length 120 pc, with periodic boundaries
on all sides.
Grid size $\delta x=0.5$, 1, 2 and 4$\pc$  along each side are considered.
We solve the system of non-ideal compressible MHD equations, including 
mass Eq\,\eqref{eq:mass}, momentum Eq\,\eqref{eq:mom}, energy Eq\,\eqref{eq:ent} and
induction Eq.\,\eqref{eq:ind}:
%----------------------------------------------------------------------------
  \begin{eqnarray}
  \label{eq:mass}
    \frac{D\rho}{Dt} &=& 
    -\vect\nabla \cdot (\rho \vect{u})
    +\vect\nabla \cdot\zeta_D\vect\nabla\rho,\\
%----------------------------------------------------------------------------
  \label{eq:mom}
    \rho\frac{D\vect{u}}{Dt} &=& 
    \vect\nabla{\ESK\sigma}
    -\rho c_{\rm s}^2\vect\nabla\left({s}/{c_{\rm p}}+\ln\rho\right)
    +\vect{j}\times\vect{B}
    \nonumber\\
    &+&\vect\nabla\cdot \left(2\rho\nu{\mathbfss W}\right)
    +\rho\,\vect\nabla\left(\zeta_{\nu}\vect\nabla \cdot \vect{u} \right)
    \nonumber\\
    &+&\vect\nabla\cdot \left(2\rho\nu_3{\mathbfss W}^{(3)}\right),\\
%----------------------------------------------------------------------------
  \label{eq:ent}
    \rho T\frac{D s}{Dt} &=&
     \EST\dot\sigma +\rho\Gamma
    -\rho^2\Lambda +\eta\mu_0\vect{j}^2 
    \nonumber\\
    &+&2 \rho \nu\left|{\mathbfss W}\right|^{2}
    +\rho\,\zeta_{\nu}\left(\vect\nabla \cdot \vect{u} \right)^2
    \nonumber\\
    &+&\vect\nabla\cdot\left(\zeta_\chi\rho T\vect\nabla s\right)
    +\rho T\chi_3\vect\nabla^6 s,\\
%----------------------------------------------------------------------------
  \label{eq:ind}
    \frac{\partial \vect{A}}{\partial t} &=&
    \vect{u}\times\vect{B}
    +\eta\vect\nabla^2\vect{A}
    +\eta_3\vect\nabla^6\vect{A},
  \end{eqnarray}
completing the system with the equation of state for an ideal gas.
Many symbols have their usual meaning, and a table (\ref{sec:table}) is included in the appendix
for clarity.
In particular, terms containing $\zeta_D,\,\zeta_\nu$ and $\zeta_\eta$ are the
application of artificial diffusion acting proportional to shocks in 
Eqs.\,\eqref{eq:mass},\,\eqref{eq:mom} and \eqref{eq:ent}, respectively.
This machinery is described in detail in \citet{GMKSH20}, which is included to
resolve shock discontinuities.
Shock diffusion is not applied to Eq.\,\eqref{eq:ind}, where it would result in 
unphysical dissipation of the magnetic field in the shock fronts, when in fact 
here it is highly enhanced by compression.
In Eqs.\,\eqref{eq:mom} -- \eqref{eq:ind}, expressions containing 
$\nu_3,\,\chi_3$ and $\eta_3$ apply sixth-order hyperdiffusion applying primarily
at the grid scale to resolve small-scale instabilities \citep[see, e.g.,][]{ABGS02,HB04}.

\begin{figure*}
\gridline{\fig{figs/eB-res-4eta.png}{0.45\textwidth}{(a)}
          \fig{figs/eB-res-3eta.png}{0.45\textwidth}{(b)}
          }
\caption{
The volume averaged magnetic energy density for models with $\dx$
between $0.5\pc$ and $4\pc$ are plotted over time.
These are scaled by reference to their
 time-averaged statistical-steady kinetic energy density.
Resistivity, $\eta=10^{-4}\kpc\kms$ in panel {\rm(a)} and $10^{-3}$
{\rm(b)}, is applied.
\label{fig:eb-res}}
\end{figure*}

The supernovae are injected randomly uniform in 3D space and as a
Poisson process proportional to the solar neigbourhood rate in the Milky Way,
 $\SNr\simeq 50\kpc^{-3}\Myr^{-1}$. 
Each explosion injects $10^{51}\erg$ as thermal energy, $\EST$, but in 
dense media a small proportion may be in kinetic energy, $\ESK$.
This is described in detail by \citet{GMKSH20}.
Non-adiabatic heating, $\Gamma$, and cooling, $\Lambda$, processes are included
following \citet{Wolfire:1995} and \citet{Sarazin:1987}, as described in 
\citet{Gent:2013a}. 

%MJK Somewhere we should point out that this strategy, in effect, results in us considering low Pm flows only. This is not ideal for the ISM, of course, but maybe our excuse here is that this regime is known to be more difficult for the SSD in any case, as then the eta cut off is places in the inertial range of the kinetic turbulence, hence most of our conclusions should definitely hold for the ISM.
In the subset of experiments presented in this letter we set viscosity, $\nu=0$,
with $\nu_3$ applying optimally for each $\dx$ to ensure the flow is 
well resolved.
We establish a benchmark for the magnetic field evolution setting $\eta=0$, with
numerical resistivity acting via the optimal setting for $\eta_3$.
We then vary $\eta\geq10^{-6}\kpc\kms$ to identify minimal range for which the physical
resistivity is dynamically dominating the numerical resistivity, and examine the
dependence of the dynamo on $\nu$ and grid resolution.
Henceforth, where $\eta$ is specified for the SN simulations it has units of $\kpc\kms$. 
In contrast to our earlier experiments \citep{Gent:2013a,Gent:2013b,GMKSH20},
we do not include thermal diffusivity, $\chi$, in these experiments as 
the artificial diffusivities are adequate to ensure numerical stability and
the physical effects of thermal conductivity can be expected to be relevant only
at time and spatial scales much shorter than considered here.
\fag{Mordecai, could you suggest some appropriate reference?}

An important consideration in dynamo theory is the magnetic Reynolds number, Rm,
and the magnetic Prandtl number, $\Pm = \Rm/\Rey \simeq \nu/\eta$, where $\Rey$
is the fluid Reynolds number.
A commonly used definition is \[\Rm=\frac{\ell u_{\rm rms}}{\eta},\] where
$\ell$ is the forcing scale.
However, in these simulations of the ISM, given the Galilean invariance of the 
turbulence, variation within hot and warm medium, and inclusion of
hyperdiffusion, defining a single forcing scale from the SN explosions or 
common rms velocity is unreliable.
We, therefore, directly compute a field of Reynolds numbers from the equations,
for example,
\begin{eqnarray}
  \Rm = \frac{\left|\vect{u}\times\vect{B}\right|}{
    \left|\eta\vect\nabla^2\vect{A}+\eta_3\vect\nabla^6\vect{A}\right|}.
\end{eqnarray}
Hence, for the high resolution runs at 20\,Myr we obtain linear 
$\langle\Rm\rangle=4$ with standard deviation 5.6, whereas 
$\Rm = 15000$, for $\eta=10^{-4}$, $\ell=50\pc$ and $u_{\rm rms}=30\kms$ using
the common formula.
In this suite of experiments, with $\nu=0$ and $\nu_3=\eta_3$ sufficient to
numerically resolve the grid scale, $\Pm<1$, which is an even more difficult
regime in which to excite the small scale dynamo than in the high $\Pm$
regime typical of the ISM.
In separate experiments with high Pm, which we shall include in the more in
depth future analysis, we do in fact see increased growth rates for higher
Pm.


\section{Resolution and resistivity} \label{sec:results}

\begin{figure*}
\gridline{\fig{figs/0_5pc-eB-nu4.png}{0.45\textwidth}{(a)}
            \fig{figs/1pc-eB-nu4.png}{0.45\textwidth}{(b)}
          }
\gridline{  \fig{figs/2pc-eB-nu4.png}{0.45\textwidth}{(c)}
            \fig{figs/4pc-eB-nu4.png}{0.45\textwidth}{(d)}
          }
\gridline{
            \fig{figs/2pc-eB-nu5.png}{0.45\textwidth}{(e)}
            \fig{figs/4pc-eB-nu6.png}{0.45\textwidth}{(f)}
          }
  \begin{picture}(0,0)(0,0)
    \put( 56,428){\begin{scriptsize}{\sf{$\delta x=0.5$pc}}\end{scriptsize}}
    \put(440,428){\begin{scriptsize}{\sf{$\delta x=1.0$pc}}\end{scriptsize}}
    \put( 56,256){\begin{scriptsize}{\sf{$\delta x=2.0$pc}}\end{scriptsize}}
    \put(440,256){\begin{scriptsize}{\sf{$\delta x=4.0$pc}}\end{scriptsize}}
    \put( 56, 78){\begin{scriptsize}{\sf{$\delta x=2.0$pc}}\end{scriptsize}}
    \put(440, 78){\begin{scriptsize}{\sf{$\delta x=4.0$pc}}\end{scriptsize}}
  \end{picture}
\caption{
The effect of resistivity $\eta$ is compared at each $\dx$, {\rm(a)} --
 {\rm(d)} for supernova rate $\dot\sigma=0.2\SNr$ and {\rm(e)} -- {\rm(f)} for
$\dot\sigma=\SNr$ at lower resolution, 
where $\SNr\simeq 50$\,kpc$^{-3}$\,Myr$^{-1}$ is the solar neighbourhood equivalent random SN frequency.
The time axes vary between plots sufficient to reach saturation of the dynamo.
At $\dx=4\pc$ the models with $\eta=10^{-4}$ are 
continued until the dynamo has saturated, for comparison with the higher 
resolution saturation levels.
\label{fig:eb-nu}}
\end{figure*}

In Figure\,\ref{fig:eb-res} the response of the magnetic energy density 
$e_B/\overline{e_K}$ to resolution is shown
for resistivity $\eta=10^{-4}$, panel\,(a) and
$\eta=10^{-3}$, panel\,(b).
The time in Myr is plotted on a log scale to better present the differences in 
the higher resolution runs, which saturate much faster then the $\dx=4\pc$ 
run.
Whilst for $\eta=10^{-4}$ the dynamo growth rates are very slow at $\dx=4\pc$,
the growth rate shows some convergence near $\dx=0.5\pc$ and 
the saturation level of above 5\% of $\overline{e_K}$ is independent of
resolution. 
%MJK How do we know that we currently capture the correct one? 
At $\eta=10^{-3}$, there is a \emph{false convergence} \citep{FMA91}
between $\dx=2$ and $4\pc$,
with the magnetic energy decaying at similar rates, however at higher resolution
we see that the results converge around an alternative solution, with a dynamo
amplification of the field occurring between 400 and 600\,Myr.

In Figure\,\ref{fig:eb-nu} we show how at each resolution $e_B/\overline{e_K}$
evolves for various values of $\eta$.
We note that at $\dx=0.5$, panel (a), and $1\pc$, panel (b), the profile
at $\eta=10^{-5}$ is indistinguishable from $\eta=0$, and so numerical
resistivity continues to control the dynamics.
At $\eta=10^{-4}$ the dynamo diverges from $\eta=0$, such that physical resistivity
is dynamically dominant, at least for some of the domain and Rm is unevenly
defined by combination of physical and numerical resistivity.
At all $\dx$ $\eta=10^{-3}$ is clearly well resolved and dynamics and
%MJK Has Rm been defined? I think we should try to make a connection to earlier studies by using the "standard" definition, and then the one directly computing from the code as a ratio of the relevant terms. 
Rm well defined by the physical application of resistivity.
%MJK This sounds like belonging to the simulation design, and not in the middle of results.
To reduce statistical noise, at all $\dx$ and $\eta$ the supernova are
scheduled at the same time and coordinates for $\dot\sigma=0.2\SNr$,
panels (a) -- (d), and for $\dot\sigma=\SNr$, panels (e) -- (f).
For low resolution the timestep is longer, such that the actual timing and 
environment of the explosions can differ between models, so the statistical
noise is more evident, particularly in panels (e) and (f). 

%MJK Would help if it was not introduced as a complication, but explained first as behavior seen in the figure, then given an explanation, and then convincing the reader that what is seen now only in Figs 3 a,b as puzzling, we can easily explain in these terms.
%MJK Overall remark of the figures: There are plenty of white space in between, and the figure captions are complicated. Use some of this white space to give more information in the title, for example:
%MJK In Fig. 3: SN rate in the title, perhaps together with the delta x?
%MJK Fig. 4: add a subplot to the kinetic energy spectra, showing the same in linear scale, to emphasize the bottleneck? Indicate eta values in the title of each plot? Times simply as smooth transition in color, without the multiple legends, extrema values and amount of increments in the caption?
%MJK Fig. 5: eta values in the title of each plot ((a) - (f)); resolution (g)-(h)?
The evolution is complicated by the impact \emph{thermal runaway}
\citep[see e.g.,][]{LOCBN15}, combination of multiple SN into superbubbles
resulting in persistent hot regions.
We shall report analysis of these results in depth in future work, but an 
initial observation of the growth profiles could lead to the conclusion that
growth rates for high resistivity (low Rm) can exceed comparable rates for high
Rm, in contradiction to dynamo theory and previous experimental results.
Closer inspection of each plot, zooming in on specific intervals, however,
confirms that growth rates are higher as expected for higher Rm.
For example between 8 and 15 Myr there is weak growth or decay in panels (a)
and (b) with critical $\eta$ for dynamo between $10^{-4}$ and $10^{-3}$.
Rm and corresponding growth at low $\eta$ is higher in the $0.5\pc$ simulation
with lower numerical resistivity, but 
converge for $\eta=10^{-3}$, where the physical resistivity is most resolved.
Between 15 and 20\,Myr there is a fast dynamo driven in the hot gas, in which
growth rates are related to the level of resistivity.
The high Rm models at $0.5\pc$ already saturate by 20\,Myr, but for $1\pc$ there
is another slow dynamo driven in the warm gas up to 40\,Myr and then a
subsequent 'hot' dynamo resulting in saturation for all models within 60\,Myr,
including for $\eta=10^{-3}$.

%MJK Now we have too many different names for this dynamo: hot, fast, and what not. Should choose one.
At low resolution, with $\dot\sigma=0.2\SNr$ panels (c) and (d), $\eta=10^{-4}$
is not distinct from numerical resistivity, and there is no hot dynamo at all
for $\dx=4\pc$ with the well resolved $\eta=10^{-3}$.
At 100\,Myr there is a hot dynamo at $\dx=2\pc$, but this is not
sustained and for $\eta=10^{-3}$ susbequently diffuses away.
At $\dot\sigma=\SNr$, panels (e) and (f), the increased Rm due to the 
higher forcing rate is sufficient to produce a dynamo at $\eta\geq10^{-3}$, but
not for $\eta=5\cdot10^{-3}$.
Given the increased statistical noise we cannot confirm that $\eta\leq10^{-4}$
applies beyond the numerical resistivity.
However, at $\dx=2\pc$ $\eta=5\cdot10^{-4}$ (magenta, solid) is resolved and at $\dx=4\pc$
 the resolved limit is $\eta\lesssim10^{-3}$ (cyan, dotted).
In panel (e) we see two hot dynamo phases at 90  and 110\,Myr, which are not 
present in panel (f).
There is a reduced susceptibility to thermal runaway at low resolution, 
and only lower Rm can be resolved, both an effect of more diffusive numerics.
Hence, as the resolution becomes more coarse it is likely that the small-scale 
dynamo will be suppressed in simulations of the ISM.
Higher supernova rates driving increased velocities may permit coarse models 
to attain the critical Rm for SSD, but it is likely that models with 
$\dx$ much above $4\pc$ will suppress the dynamo. 

To attain a truly convergent numerical regime simulations should apply 
resolution better than $2\pc$.
In the ISM with Rm much higher than we can resolve, our results confirm that
SSD is very easy to excite by SN, and may even be present in galaxies with
low SN rates.
As shown in panels (a) and (b), properties of the hot gas cannot exclude
SSD even with high resistivity in galaxies that produce sustained regions of
hot turbulence.
We shall examine the nature of this fast dynamo in a later report.
To exclude SSD in SN driven turbulence simulations will require
$\eta\gtrsim10^{-3}$ or supernova implementation which avoids thermal runaway.
In \citet{Gressel:2008,GE20}, $\dx$ is 8.3 and $6.7\pc$, respectively
and $\eta\simeq6.5\cdot10^{-3}\kpc\kms$, which excludes SSD,
while \citet{Gent:2013b} applied $\eta\simeq8\cdot10^{-4}\kpc\kms$ with $\dx=4\pc$,
which would support SSD, even without thermal runaway.
The latter obtain a large scale dynamo with galactic angular momentum
$\Omega=\OSN$, where $\OSN=25\kms\kpc^{-1}$ is the rate in the solar
neighbourhood, while the former require $\Omega\geq4\OSN$ to excite LSD.
Rm applying at the largest scales would be 7.5 times higher for the latter 
model, so alone is sufficient to explain the LSD.
How the SSD affects the galactic dynamo requires further study.
\fag{add discussion of Ulrich's results}

\begin{figure*}
\gridline{\fig{figs/0_5pcPm0e-4_0Bpower.png}{0.45\textwidth}{(a)}
          \fig{figs/0_5pcPm0e-4_0kpower.png}{0.45\textwidth}{(b)}
          }
\gridline{\fig{figs/0_5pcPm0e-3_0Bpower.png}{0.45\textwidth}{(c)}
          \fig{figs/0_5pcPm0e-3_0kpower.png}{0.45\textwidth}{(d)}
          }
\caption{
Compensated energy spectra at times in Myr given in the legends for 
$0.5\pc$ resolution.
Rm is super critical for dynamo applying $\eta=10^{-4}$ in
panels (a) and (b) and sub critical or marginal for dynamo applying
$\eta=10^{-3}$ in panels (c) and (d).
Energy spectra are compensated against theoretical profiles of Kazentsev
$k^{3/2}$, (a) and (c), and Kolmogorov $k^{-5/3}$, (b) and (d), 
each represented by the horizontal black dashed lines.
\label{fig:4power}}
\end{figure*}

In contrast to the simplified model in Section\,\ref{sec:ssd-tang}, SN driven
turbulence does not have a well-defined forcing scale, due to the heterogeneous
temperature and density of the ISM and random clustering of explosions.
In our simulations the forcing scale will range between 10 and $50\pc$,
or $k\in(3,16)$.
In Figure\,\ref{fig:4power} we show the evolving compensated spectra for the
high resolution kinetic energy in the kinematic phase (b) for $\eta=10^{-4}$
and during decay (d) for $\eta=10^{-3}$.
The spectra magnitude show the same fluctuations over time, except at 32\,Myr,
when the dynamo saturates and reduces slightly of the energy in (b) compared
to (d). 
Many of the spectra display a \emph{bottleneck} effect \fag{reference tba} with
a peak at $k\simeq50$.
The compensated magnetic energy spectra in panels (a) and (c) have ranges 
conforming to the Kazentsev inverse cascade in the SSD case (a) extending to
$k\gtrsim 20$, consistent with the SSD behaviour of the toy model in
Figure\,\ref{fig:tangling}\,(b).
In panel (c) of Figure\,\ref{fig:4power} the Kazentsev range in similar fashion
to Figure\,\ref{fig:tangling}\,(c) is in the range $K\lesssim10$ except
for 22\,Myr, which corresponds to a short growth spurt in
Figure\,\ref{fig:eb-res}\,(b) and consistent with the presence of SSD.

\begin{figure*}
\gridline{\vspace{-0.7cm} \fig{figs/nu0_Bpower.png}{0.45\textwidth}{\vspace{-1.3cm}\hspace{-2cm}(a)}
          \vspace{-0.7cm} \fig{figs/nu0_kpower.png}{0.45\textwidth}{\vspace{-1.3cm}\hspace{-2cm}(b)}}                                                                             
\gridline{\vspace{-0.7cm}\fig{figs/nu1_Bpower.png}{0.45\textwidth}{\vspace{-1.3cm}\hspace{-2cm}(c)}
          \vspace{-0.7cm}\fig{figs/nu1_kpower.png}{0.45\textwidth}{\vspace{-1.3cm}\hspace{-2cm}(d)}}                                                                             
\gridline{\vspace{-0.7cm} \fig{figs/nu10_Bpower.png}{0.45\textwidth}{\vspace{-1.3cm}\hspace{-2cm}(e)}
          \vspace{-0.7cm} \fig{figs/nu10_kpower.png}{0.45\textwidth}{\vspace{-1.3cm}\hspace{-2cm}(f)}}                                                                             
\gridline{\vspace{-0.3cm}  \fig{figs/SN_Bpower.png}{0.45\textwidth}{\vspace{-1.3cm}\hspace{-2cm}(g)}
          \vspace{-0.3cm}  \fig{figs/SN_kpower.png}{0.45\textwidth}{\vspace{-1.3cm}\hspace{-2cm}(h)} }
  \begin{picture}(0,0)(0,0)
    \put(190,455){\begin{scriptsize}{\sf{$\eta=0$}}\end{scriptsize}}
    \put(440,455){\begin{scriptsize}{\sf{$\eta=0$}}\end{scriptsize}}
    \put(190,325){\begin{scriptsize}{\sf{$\eta=10^{-4}$}}\end{scriptsize}}
    \put(440,325){\begin{scriptsize}{\sf{$\eta=10^{-4}$}}\end{scriptsize}}
    \put(190,192){\begin{scriptsize}{\sf{$\eta=10^{-3}$}}\end{scriptsize}}
    \put(440,192){\begin{scriptsize}{\sf{$\eta=10^{-3}$}}\end{scriptsize}}
  \end{picture}
\caption{
(a) -- (f): compensated energy spectra at $t=19.5\Myr$ ($\dx=0.5\pc$ \& $1\pc$),
$t=100\Myr$ ($\dx=2\pc$ \& $4\pc$).
Resistivity $\eta=0,\,10^{-4}$ and $10^{-3}$ in panel pairs (a,b), (c,d) and
(e,f), respectively.
Panels (g,h) for $\dx=2\pc$ at 100\,Myr for $\dot\sigma=0.2\SNr$ and
140\,Myr for $\dot\sigma=\SNr$ for
$\eta=10^{-4}$ or $10^{-3}$ as listed in the legends show the effect of 
supernova rate.
Energy spectra are compensated against theoretical profiles of Kazentsev
$k^{3/2}$, left panels, and Kolmogorov $k^{-5/3}$, right. 
\label{fig:3power}}
\end{figure*}

%MJK The bottleneck in the kinetic energy spectra seems independent of Pm EXCEPT for the high SN rate case. Why should that be so different?
In Figure\,\ref{fig:3power} panels (a) -- (f) the effect of resolution on the
compensated energy spectra is shown for $\eta=0$, $10^{-4}$ and $10^{-3}$
with $\dot\sigma=0.2\SNr$.
The time of the spectral snapshot for $\dx=0.5$ and $1\pc$ correspond to a
common phase of rapid magnetic growth fastest in the hot gas, while for
$\dx=2$ and $4\pc$ it corresponds to a period of hot-phase rapid growth at
$2\pc$, which is absent in the $4\pc$ model with low $\dot\sigma$.
The hydrodynamic parameters are fixed for each resolution and are 
consistent between panels (b), (d) and (f).
We note that, there is convergence in the kinetic energy spectrum for
$\dx=0.5\pc$ and $\dx=1\pc$, apart from the energy cutoff at higher $k$
consistent with higher resolution.
Also, the total kinetic energy is reduced with low resolution due to 
increased viscous dissipation acting at lower $k$.
To fully resolve the kinetic energetics for scales larger than $k=24$ corresponding to $\ell\simeq40\pc$ would therefore require $\dx\gtrsim1\pc$.
The energy spectrum at $\dx=2\pc$ shares the same characterstics as the high
resolution runs, but with some loss of energy.

There is a bottleneck evident, an energy cascade less efficient than $k^{-5/3}$
terminating at a peak, before rapid dissipation at high $k$.
This bottleneck shifts to lower $k$ as $\dx$ increases, but always at higher
$k$ than the Kazentsev range in the corresponding magnetic spectrum.


\section{Conclusion and discussion}\label{sec:conc}

\bibliography{refs}{}
\bibliographystyle{aasjournal}

\appendix

\section{Notation}\label{sec:table}

\begin{table*}[h]
%\begin{deluxetable*}{ccl}
%\tablenum{A1}
%\tablecaption{Notation\label{tab:notation}}
%\tablewidth{0pt}
%\tablehead{
%\colhead{Notation Symbol} & \colhead{Denoting} & \colhead{Units/Definition} 
%\\
%}
%\decimalcolnumbers
%\startdata\\
\begin{tabular}{ccl}
\hline\hline\\
{Notation Symbol} & {Denoting} & {Units/Definition}\\\hline\\
 $\dfrac{D}{Dt}$ & material derivative & $\dfrac{\partial }{\partial t}+\vect{u}\cdot \vect\nabla$ \\
 $\vect\nabla$ & gradient vector & e.g.,$\dfrac{\partial }{\partial x},\dfrac{\partial }{\partial y},\dfrac{\partial }{\partial z}$ \\
 $\rho$ & gas density & [g cm$^{-3}$]  \\
 $\vect u$ & gas velocity & [km s$^{-1}$] \\
 $t$ & time & [Myr] \\
 $s$ & specific entropy & [erg g$^{-1}$ K$^{-1}$] \\
 $T$ & gas temperature & [K] \\
 $\vect A$ & magnetic vector potential & [$\upmu$G cm] \\
 $\vect B$ & magnetic field & [$\upmu$G] \\
 $\vect j$ & current density & [Bi cm$^{-2}$] \\
 $\mathbfss W$ & traceless rate of strain tensor &
   ${\mathsf W}_{ij} = \dfrac{1}{2}\left(\dfrac{\partial u_i}{\partial x_j}
                  + \dfrac{\partial u_j}{\partial x_i}
                  -\dfrac{2}{3} \delta_{ij}\vect\nabla\cdot \vect u\right)$ \\
 $|\mathbfss W|^2$ & contracted rate of strain tensor &
   $|\mathbfss W|^2={\mathsf W}_{ij}{\mathsf W}_{ij}$\\
 $\mathbfss W^{(3)}$ & 6th order rate of strain tensor &
   ${\mathsf W}_{ij}^{(3)} = \dfrac{1}{2}\left(\dfrac{\partial^5 u_j}{\partial x_i^5}
                  + \dfrac{\partial^4}{\partial x_i^4}\left(\dfrac{\partial u_i}{\partial x_j}\right)
                  -\dfrac{1}{3}\dfrac{\partial^4}{\partial x_i^4}\left(\vect\nabla\cdot \vect u\right)\right)$ \\
 $\zeta_{D},\zeta_{\nu},\zeta_{\chi}$ & shock diffusion coefficient& $\propto \left(-\vect\nabla\cdot\vect u\right)_+$\\
 $\nu,\eta$ & viscosity, resistivity coefficient& [kpc km s$^{-1}$]\\
 $\nu_3,\chi_3,\eta_3$ & hyperdiffusion coefficient& [kpc$^{5}$ km s$^{-1}$]\\
 $\Gamma$ & UV-heating& [erg g$^{-1}$ s$^{-1}$]\\
 $\Lambda$ & radiative cooling& [erg cm$^{3}$ g$^{-2}$ s$^{-1}$]\\
 $\ESK+\EST$ & SN explosion energy& [10$^{51}$erg]\\
 $\dot\sigma$ & SN explosion rate & [kpc Myr$^{-1}$]\\
 $c_{\rm s}$ & sound speed & [km s$^{-1}$]\\
 $c_{\rm p}$ & specific heat at constant pressure & [erg g$^{-1}$ K$^{-1}$]\\
 $e_B$ & magnetic energy density & [erg cm$^{-3}$]\\
 $\overline{e_K}$ & time-averaged kinetic energy density & [erg cm$^{-3}$]\\
%\\
\hline
\end{tabular}
%\enddata
%\tablecomments{
%Summary of notation used in the text.}
%\end{deluxetable*}
%\tablecomments{\fag{to be continued \ldots}
%This table ``hides'' the third column in the \latex\ when compiled.
%The Distance is also centered on the decimals.  Note that when using decimal
%alignment you need to include the {\tt\string\decimals} command before
%{\tt\string\startdata} and all of the values in that column have to have a
%space before the next ampersand.}
\end{table*}
%\end{deluxetable*}

%% This command is needed to show the entire author+affiliation list when
%% the collaboration and author truncation commands are used.  It has to
%% go at the end of the manuscript.
%\allauthors

%% Include this line if you are using the \added, \replaced, \deleted
%% commands to see a summary list of all changes at the end of the article.
%\listofchanges
\end{document}

