%% using aastex version 6.3
\documentclass[preprint2]{aastex63}

%% The default is a single spaced, 10 point font, single spaced article.
%% There are 5 other style options available via an optional argument. They
%% can be invoked like this:
%%
%% \documentclass[arguments]{aastex63}
%% 
%% where the layout options are:
%%
%%  twocolumn   : two text columns, 10 point font, single spaced article.
%%                This is the most compact and represent the final published
%%                derived PDF copy of the accepted manuscript from the publisher
%%  manuscript  : one text column, 12 point font, double spaced article.
%%  preprint    : one text column, 12 point font, single spaced article.  
%%  preprint2   : two text columns, 12 point font, single spaced article.
%%  modern      : a stylish, single text column, 12 point font, article with
%% 		  wider left and right margins. This uses the Daniel
%% 		  Foreman-Mackey and David Hogg design.
%%  RNAAS       : Preferred style for Research Notes which are by design 
%%                lacking an abstract and brief. DO NOT use \begin{abstract}
%%                and \end{abstract} with this style.
%%
%% Note that you can submit to the AAS Journals in any of these 6 styles.
%%
%% There are other optional arguments one can invoke to allow other stylistic
%% actions. The available options are:
%%
%%   astrosymb    : Loads Astrosymb font and define \astrocommands. 
%%   tighten      : Makes baselineskip slightly smaller, only works with 
%%                  the twocolumn substyle.
%%   times        : uses times font instead of the default
%%   linenumbers  : turn on lineno package.
%%   trackchanges : required to see the revision mark up and print its output
%%   longauthor   : Do not use the more compressed footnote style (default) for 
%%                  the author/collaboration/affiliations. Instead print all
%%                  affiliation information after each name. Creates a much 
%%                  longer author list but may be desirable for short 
%%                  author papers.
%% twocolappendix : make 2 column appendix.
%%   anonymous    : Do not show the authors, affiliations and acknowledgments 
%%                  for dual anonymous review.
%%
%% these can be used in any combination, e.g.
%%
%% \documentclass[twocolumn,linenumbers,trackchanges]{aastex63}
%%
%% AASTeX v6.* now includes \hyperref support. While we have built in specific
%% defaults into the classfile you can manually override them with the
%% \hypersetup command. For example,
%%
%% \hypersetup{linkcolor=red,citecolor=green,filecolor=cyan,urlcolor=magenta}
%%
%% will change the color of the internal links to red, the links to the
%% bibliography to green, the file links to cyan, and the external links to
%% magenta. Additional information on \hyperref options can be found here:
%% https://www.tug.org/applications/hyperref/manual.html#x1-40003
%%
%% Note that in v6.3 "bookmarks" has been changed to "true" in hyperref
%% to improve the accessibility of the compiled pdf file.
%%
%% If you want to create your own macros, you can do so
%% using \newcommand. Your macros should appear before
%% the \begin{document} command.
%%
\usepackage{xfrac}
\usepackage{amsmath}
\usepackage{upgreek}

\newcommand{\vdag}{(v)^\dagger}
\newcommand\aastex{AAS\TeX}
\newcommand\latex{La\TeX}
\newcommand\Rm{{\rm Rm} }
\newcommand\Rey{{\rm Re} }
\newcommand\Pm{{\rm Pm} }
\newcommand\kf{k_{\rm f} }
\newcommand\SNr{\dot\sigma_{\rm sn}}
\newcommand\OSN{\Omega_{\rm sn}}
\newcommand\ESK{E_{\rm kin}}
\newcommand\EST{E_{\rm th}}
\newcommand\ESN{E_{\sigma}}
\newcommand{\vect}[1]{{{\mbox{\boldmath $#1$}}}}%also makes bold Greek letters
\newcommand{\mathbfss}[1]{\textbf{\textsf{#1}}}
\newcommand\kpc{~ {\rm kpc}}
\newcommand\pc{~ {\rm pc}}
\newcommand\dx{~ {\delta x}}
\newcommand\Myr{~ {\rm Myr}}
\newcommand\erg{~ {\rm erg}}
\newcommand\kms{~ {\rm km~ s}^{-1}}

\definecolor{midblue}{rgb}{0.0,0.4,0.7}
\definecolor{midgreen}{rgb}{0.0,0.8,0.3}
\definecolor{mypurple}{rgb}{0.8,0.2,0.8}
\newcommand{\fg}[1]{\textcolor{midblue}{#1}}
\newcommand{\fag}[1]{\textcolor{midgreen}{FAG: #1}}
%\newcommand{\ns}[1]{\textcolor{orange}{NS: #1}}
\newcommand{\ns}[1]{\textcolor{orange}{#1}}

%% Reintroduced the \received and \accepted commands from AASTeX v5.2
\received{June 1, 2019}
\revised{January 10, 2019}
\accepted{\today}
%% Command to document which AAS Journal the manuscript was submitted to.
%% Adds "Submitted to " the argument.
\submitjournal{ApJL}

%% For manuscript that include authors in collaborations, AASTeX v6.3
%% builds on the \collaboration command to allow greater freedom to 
%% keep the traditional author+affiliation information but only show
%% subsets. The \collaboration command now must appear AFTER the group
%% of authors in the collaboration and it takes TWO arguments. The last
%% is still the collaboration identifier. The text given in this
%% argument is what will be shown in the manuscript. The first argument
%% is the number of author above the \collaboration command to show with
%% the collaboration text. If there are authors that are not part of any
%% collaboration the \nocollaboration command is used. This command takes
%% one argument which is also the number of authors above to show. A
%% dashed line is shown to indicate no collaboration. This example manuscript
%% shows how these commands work to display specific set of authors 
%% on the front page.
%%
%% For manuscript without any need to use \collaboration the 
%% \AuthorCollaborationLimit command from v6.2 can still be used to 
%% show a subset of authors.
%
%\AuthorCollaborationLimit=2
%
%% will only show Schwarz & Muench on the front page of the manuscript
%% (assuming the \collaboration and \nocollaboration commands are
%% commented out).
%%
%% Note that all of the author will be shown in the published article.
%% This feature is meant to be used prior to acceptance to make the
%% front end of a long author article more manageable. Please do not use
%% this functionality for manuscripts with less than 20 authors. Conversely,
%% please do use this when the number of authors exceeds 40.
%%
%% Use \allauthors at the manuscript end to show the full author list.
%% This command should only be used with \AuthorCollaborationLimit is used.

%% The following command can be used to set the latex table counters.  It
%% is needed in this document because it uses a mix of latex tabular and
%% AASTeX deluxetables.  In general it should not be needed.
%\setcounter{table}{1}

%%%%%%%%%%%%%%%%%%%%%%%%%%%%%%%%%%%%%%%%%%%%%%%%%%%%%%%%%%%%%%%%%%%%%%%%%%%%%%%%
%%
%% The following section outlines numerous optional output that
%% can be displayed in the front matter or as running meta-data.
%%
%% If you wish, you may supply running head information, although
%% this information may be modified by the editorial offices.
\shorttitle{Small-scale dynamo in the ISM}
\shortauthors{Gent et al.}
%%
%% You can add a light gray and diagonal water-mark to the first page 
%% with this command:
%% \watermark{text}
%% where "text", e.g. DRAFT, is the text to appear.  If the text is 
%% long you can control the water-mark size with:
%% \setwatermarkfontsize{dimension}
%% where dimension is any recognized LaTeX dimension, e.g. pt, in, etc.
%%
%%%%%%%%%%%%%%%%%%%%%%%%%%%%%%%%%%%%%%%%%%%%%%%%%%%%%%%%%%%%%%%%%%%%%%%%%%%%%%%%

%% This is the end of the preamble.  Indicate the beginning of the
%% manuscript itself with \begin{document}.

\begin{document}

\title{Small-scale dynamo in supernova driven ISM turbulence}

%%
%% The \author command is the same as before except it now takes an optional
%% argument which is the 16 digit ORCID. The syntax is:
%% \author[xxxx-xxxx-xxxx-xxxx]{Author Name}
%%
%%
%% Use \affiliation for affiliation information. The old \affil is now aliased
%% to \affiliation. AASTeX v6.3 will automatically index these in the header.
%% When a duplicate is found its index will be the same as its previous entry.
%%
%% Note that \altaffilmark and \altaffiltext have been removed and thus 
%% can not be used to document secondary affiliations. If they are used latex
%% will issue a specific error message and quit. Please use multiple 
%% \affiliation calls for to document more than one affiliation.
%%
%% The new \altaffiliation can be used to indicate some secondary information
%% such as fellowships. This command produces a non-numeric footnote that is
%% set away from the numeric \affiliation footnotes.  NOTE that if an
%% \altaffiliation command is used it must come BEFORE the \affiliation call,
%% right after the \author command, in order to place the footnotes in
%% the proper location.
%%
%% Use \email to set provide email addresses. Each \email will appear on its
%% own line so you can put multiple email address in one \email call. A new
%% \correspondingauthor command is available in V6.3 to identify the
%% corresponding author of the manuscript. It is the author's responsibility
%% to make sure this name is also in the author list.
%%
%% While authors can be grouped inside the same \author and \affiliation
%% commands it is better to have a single author for each. This allows for
%% one to exploit all the new benefits and should make book-keeping easier.
%%
%% If done correctly the peer review system will be able to
%% automatically put the author and affiliation information from the manuscript
%% and save the corresponding author the trouble of entering it by hand.

\correspondingauthor{Frederick Gent}
\email{Email: frederick.gent@aalto.fi, mordecai@amnh.org,\\ maarit.kapyla@aalto.fi, nishant@iucaa.in}

\author[0000-0002-1331-2260]{Frederick A. Gent}
\affiliation{
Astroinformatics, Department of Computer Science, Aalto University, PO Box 15400, FI-00076 Aalto, Finland
 }
\affiliation{
    School of Mathematics, Statistics and Physics,
      Newcastle University, NE1 7RU, UK 
 }

\author[0000-0003-0064-4060]{Mordecai-Mark {Mac Low}}
\affiliation{
American Museum of Natural History, 79th Street at Central Park West, New York, NY 10024, USA
}
\affiliation{
{Center for Computational Astrophysics, Flatiron Institute, New York,
NY 10010, USA} 
}

\author[0000-0002-9614-2200]{Maarit J. K\"apyl\"a}
\affiliation{
Astroinformatics, Department of Computer Science, Aalto University, PO Box 15400, FI-00076 Aalto, Finland
}
\affiliation{
Max Planck Institute for Solar System Research, Justus-von-Liebig-Weg 3, 37707 G\"ottingen, Germany
}
\affiliation{
    Nordic Institute for Theoretical Physics,
      Roslagstullsbacken 23, 106 91 Stockholm, Sweden 
}

\author[0000-0001-6097-688X]{Nishant K. Singh}
\affiliation{
Inter-University Centre for Astronomy \& Astrophysics, Post Bag 4, Ganeshkhind, Pune 411 007, India
}
\affiliation{
Max Planck Institute for Solar System Research, Justus-von-Liebig-Weg 3, 37707 G\"ottingen, Germany
}

%% AASTeX 6.3 has the new \collaboration and \nocollaboration commands to
%% provide the collaboration status of a group of authors. These commands 
%% can be used either before or after the list of corresponding authors. The
%% argument for \collaboration is the collaboration identifier. Authors are
%% encouraged to surround collaboration identifiers with ()s. The 
%% \nocollaboration command takes no argument and exists to indicate that
%% the nearby authors are not part of surrounding collaborations.

%% Mark off the abstract in the ``abstract'' environment. 
\begin{abstract}
%We present a numerical resolution study over a range of magnetic Reynolds
%number (Rm) to identify characteristics of the small-scale dynamo (SSD) under
%FAG: Reynolds -> resistivity
%MJK Currently Rms are not given, not listed, nor any results
%MJK presented as function of Rm, hence this statement cannot
%MJK remain, unless such analysis is added.
%mm [lead with science, not numerics] We present a numerical resolution study over a range of magnetic resistivity
%(and by implication magnetic Reynolds number, Rm) to identify characteristics
%of the small-scale dynamo (SSD) under supernova-driven turbulence of the
%interstellar medium (ISM).
%FAG: OK
Magnetic fields appear to grow quickly even at early cosmological
times, suggesting the action of a small-scale dynamo (SSD) in the
interstellar medium (ISM) of galaxies. Previous studies have focused
on idealized turbulent driving of the SSD. We here simulate more
realistic supernova-driven turbulence to determine whether it can
drive an SSD.  We vary the physical resistivity (and thus the magnetic
Reynolds number), as well as the numerical resolution and supernova
rate to delineate the regime in which an SSD occurs.
%mm [this is a conclusion, so moved down to the end] We confirm that
%conditions in the ISM are likely highly conducive to SSD in the
%diffuse hot and warm turbulent ISM, given Rm in the real ISM is
%orders of magnitude higher than we model at our highest resolution and a modest SN
%explosion frequency ($\SNr$), 20\% that of the solar neighbourhood.
%FAG OK
%mm [moved findings up]
%FAG OK
We find convergence in the growth rate of the SSD with resolution approaching 
sub-parsec scale for a given supernova explosion frequency
$\dot\sigma$. Across the modelled range of 0.5 to 4 parsec resolution we find that with
sufficiently low resistivity the SSD saturates consistently at about 5\% of
the energy equipartion level, independent of the growth rate. Dynamo
growth rates for $\dot\sigma=\SNr$, the solar neighbourhood rate, relative to models with
$\dot\sigma=0.2\SNr$, are greater despite higher Mach numbers.
 %FAG moved and reworded below:
%FAG: relabelling
%mm [didn't understand how this fit into the sentence] explosion
%frequency, $\dot\sigma$, at 20\% that of the solar neighbourhood
%($\SNr$). 
%FAG OK
This is despite the negative impact on SSD expected from the highly compressible
%MJK So we have varying SN rates; do these have a higher Ma? What I am after is whether we can quantify that statement?
%FAG: yes in ssdyn_paper I now added tables for sigma=\SNr and mean Ms increases
%FAG: from 0.4/0.5 at 2pc/4pc to 0.6/0.8, even though urms is similar for each resolution.
%MJK The other aspect is that we have LOW or unity for Pm, while the ISM
%MJK has high Pm. SSD should be easier for higher Pm flows.
%FAG. now include para in section 3.
%FAG: added
%NS: should we say "... in *some of* our experiments." in the line below?
%NS: I thought there are also high Pm cases which might be included as more relevant here?
%MJK No, unfortunately not. We have runs with nu=0, meaning nu_SGS only,
%MJK and eta=0, and increasing from there. These are effectively low Pm 
%MJK cases.
%NS: Oh, ok. Then I will perhaps suppress high Pm citations from sec 2.
%FAG: keep citations as I now reference runs to follow with high Pm.
%MJK Fred now has some higher Pm runs existing, but we do not have space to show a picture of them. We will just add a sentence somehwere that they show the expected behavior (with your ref) and then state that we see it too.
%FAG OK
nature of the flow and low magnetic Prandtl number of our experiments.
%FAG: added in response to above
%
% and a trend towards the suppression of SSD for resolution
%more coarse than 4 parsecs, where higher resistivity is required to resolve
%the magnetic field.
%MJK Also the following statement is not in line with the previous one with exclusion.
%Across the modelled range of 0.5 to 4 parsec resolution we find the SSD 
%FAG: added low resistivity
%FAG: Reworked from previous sentence
%mm [this is kind of backwards] With higher resistivity required to resolve the magnetic field as resolution becomes
%more coarse the trend suggests the small scale dynamo cannot be excited for grid
%spacing much more than above 4\,pc.
%FAG OK
As the grid becomes coarser, the minimum physical resistivity that can
be resolved increases as well. The trend suggests that an SSD cannot
%FAG added much - haven't identified the hard limit yet
%be excited for grid spacing exceeding 4~pc, as the numerical
be excited for grid spacing much exceeding 4~pc, as the numerical
resistivity suppresses it.


\end{abstract}
\keywords{Dynamo --- Magnetohydrodynmaical simulations --- Supernova dynamics --- Interstellar magnetic fields --- Turbulence}


\section{Introduction}\label{sec:intro}
%==============================================================================
\fag{Perhaps here, first a paragraph addressing the importance of MF in the ISM
at all}

%MJK The landscape to me seems to be the following:
%MJK GE20 do not explicitely state that they do not have SSD, but do go forward with QKTFM, which assumes that no magnetic background turbulence is present, hence they make an implicit claim of having no SSD.
%MJK Balsara04 clearly had an SSD, as LSD was not allowed for, but had an imposed
%MJK field, and the role of that is unclear.
%MJK Fred+gang claim LSD and SSD, but it is unclear whether this is true, as
%MJK separating SSD and LSD is difficult.
%MJK Steinwandel et al. have recently claimed that an SSD is present in a
%MJK galactic-scale simulation, but no LSD, and that SSD "dies" off at later
%MJK stages, when Kazantsev scaling is not observed.
%MJK So, our study now tries to address at which resolution a fully healthy
%MJK SSD can be expected in a full ISM simulation, and helps to interpret
%MJK the mess descripbed above.
%MJK Now we DO not have imposed field, and can confirm the conclusions of Balsara04
%MJK without it.
%MJK It would be interesting if a critical Rm could be nailed down, but I doubt we
%MJK can, and must resort to what is listed above.
%NS: also commented below; could 10^-3 case be used for an estimate of Rm_crit (eq 5 based)?
%FAG 28.9 reword

This letter addresses the necessary conditions for and nature of a small-scale
dynamo (SSD) in the interstellar medium (ISM).
The SSD consists of dynamo modes acting at small eddy scales of the
turbulence,
%mm
   so it drives magnetic field growth at the corresponding short
   turnover times.
%FAG OK
There are large scales of separation between the fastest growing SSD
instabilities and the large-scale dynamo (LSD) modes generating the magnetic
field structures organised at the systemic scales of the galactic disk, spiral
arms, or similar.
As such, simulations capable of capturing the LSD alongside the faster
growing modes of SSD are computationally challenging.
However, it is likely that the interaction of SSD modes with the LSD fundamentally
alter the evolution and structure of the magnetic field.

In many simulations of supernova (SN) driven turbulence with realistic vertical 
stratification \citep[e.g.,][]{deAvillez:2005,PO07,Hill:2012a,HI14} the
mechanisms for inducing large-scale magnetic fields, such as rotation
and shear, are absent. To examine the effect of strong ordered magnetic fields these models then rely
on imposition of a background or initial magnetic field, typically uniform,
which is then perturbed by SN explosions.
The amplification of the magnetic field in these cases may include a
SSD in case of sufficiently high magnetic Reynolds numbers, Rm, but
any amplification of the 
%MJK re-instating large scale; there seems to be some sort of hyphen issue throughout, which I cannot sort out. I feel like there should be a hyphen in large-scale field, Fred obviously does not, and I am not sure whether it should be single or double hyphenated.
%FAG OK - I think it would be single and have applied throughout, also for small
large-scale field is limited to linear growth through tangling of the imposed
field.
%MJK and adding this; trying to resolve the issue.
The fluctuations can then undergo either linear or exponential growth
dependending on the absence or presence of an SSD.
%MJK
%FAG OK
%mm [I do not understand the above sentence: is it arguing that you
%can't have exponential growth from an SSD in case of an imposed field??]
%NS: i agree with Mordecai; need to modify previous two sentences.
%\ns{The above sentence seems to suggest (linear) growth of large-scale field,
%but one might argue that tangling produces fields preferably on %the forcing
%scale, thus essentially leading to (linear) growth of fluctuating %field.
%MJK I see that the reformulation due to Nishant's concerng leads to a conflict with Fred's statement in the beginning of the sentence.... See above an attempt to remedy this.
%(Aside) Interestingly, I think, it is only the tangling produced %magnetic
%noise (and not the SSD) which leads to quenching of large-scale %dynamo, as
%$\alpha_M$ is sourced primarily by tangling.}
%MJK see above; is it so easily solved, Nishant?
%NS: Yes, agreed. I will keep the comments above for now as I can formulate
%NS: a sentence on the Aside part which could be useful
Any initial large-scale field would diffuse leaving only the random field. 
If an imposed field is sufficiently close to equipartition its characteristics
would dominate any MHD results.
%NS: added; pls feel free to decide whether it should be kept here or not
%FAG OK
%mm It is reasonable to expect that the tangling produced magnetic
%noise
     On the other hand, the magnetic noise produced by tangling
will grow exponentially if the system supports the LSD instability.
Such a noise plays an important role in quenching the LSD by contributing
to the magnetic $\alpha$-effect $\alpha_M$, which leads to the saturation
of large-scale fields in the nonlinear regime.
%FAG OK
%NS.
%MJK Is this paragraph now too complicated? Should we only try to highlight the aspects relevant here? I am not sure how much text we can have here in the first place...
%FAG OK

%MJK The previous para was about neglecting the LS effects, and this paragraph continues on the same topic. Is this really intended? I guess we could get a rid of this simply. I understand models referred to above have vertical stratification and those below not, but cannot we simply group them together? Btw, stratification alone and even boundary conditions that are inhomogeneous in the vertical, can of course drive some sort of a dynamo effect, but definitely not of the type we believe to take place in the galaxies. Does anybody see a way how to reconcile this?
More commonly, models of the SN-driven dynamo in the ISM have ignored the 
large-scale effects and examined the SSD or the effect of the
turbulent magnetic field on the properties of the ISM
\citep[e.g.,][]{BKMM04,BalKim05,MacLow:2005}.
A seed or imposed field for such models could be uniform or random, but if the
simulated field
%mm [counterfactual results from
   were the result of
tangling rather than dynamo, then any conclusions
about the ISM
%mm [counterfactual]
%FAG OK
    would
depend on the veracity of the chosen field rather than the 
nature of the MHD turbulence.

%mm [where does Korpi:1999a fit into this taxonomy??]
%MJK I guess it belongs to the class of "miserable early failures to produce any dynamo", which does not need to be listed.
%FAG OK
While the SN experiments considered in this letter are restricted to SSD, some
large-scale models do seek to include the LSD
\citep[e.g.,][]{Korpi:1999b,Gressel:2008,HWK09,WA09,Gent:2013b,EGSFB16,Pakmor17,SBADMN19,SDLMBP20,GE20}.
Most of these do not include a small-scale dynamo.
%\citet{Gent:2013b,EGSFB16} do include SSD, but without a clear
%FAG: added appear to
\citet{Gent:2013b,EGSFB16} do appear to include SSD, but without a clear
understanding of the characteristics of this SSD it is difficult to explore the
effect this has on the LSD and the properties of the ISM.
%FAG: revised
%\citet{SBADMN19,SDLMBP20} also suspect they have SSD.
\citet{SBADMN19,SDLMBP20} also claim that SSD is present in a galactic-scale
simulation, but do not excite LSD. 
At late stage the SSD ceases,
%MJK this is their interpretation only! I do not think it goes anywhere, but LSD appears simply. Does their magnetic energy decay, btw?
%mm corresponding to
     as shown by
the disappearance of Kazantsev scaling in the magnetic power spectra.
%FAG OK
%FAG: ended and new para

%FAG: added
Given the uncertainty over the presence of SSD in the large-scale simulations
we seek to confirm how to identify, quantify and characterise SSD, so that we
can adequately interpret its action in simulations of LSD.
\citet{BKMM04} report SSD in ideal magnetohydrodynamics (MHD), but with only numerical diffusion and
a limited set of resolution and parameters it was not possible for them to demonstrate 
convergence of the solutions nor dependence on Rm or magnetic Prandtl number, Pm.
They include a weak imposed uniform field; we shall show that it is possible
to demonstrate conclusively that the amplification of their field is a result of
dynamo action and not just tangling of the field.
%MJK But we are not going to show any runs with imposed fields?
%MJK On the contrary, I think we can show that imposed field is NOT A REQUIREMENT to provide dynamo action.
%
In this letter we explore the SSD in realistic simulations of SN driven 
turbulence
%FAG added
in isolation from any drivers of LSD.
%
The broad resolution and parameter study is intended to identify the critical
ranges for excitation of SSD and understand dynamo growth rates and saturation 
conditions.
This will also identify the prerequisites for including or excluding SSD in the
LSD models and subsequently establish the importance of SSD-LSD interactions 
on galaxy dynamics. 
We aim to find objective criteria with which to determine the presence of SSD in 
simulations \citep[such as][]{Gent:2013b,GE20,SBADMN19}.

%==============================================================================
\begin{figure*}
\gridline{ \fig{figs/ssd-tang-brms.png}{0.45\textwidth}{(a)}
          }
\gridline{     \fig{figs/ssdBpower.png}{0.45\textwidth}{(b)}
          \fig{figs/tanglingBpower.png}{0.45\textwidth}{(c)}
          }
\caption{
%Simulation results for non-helical random forcing. 
%The simulation with $R_{\rm m}=148.0$ and $P_{\rm m}=50.0$ supports dynamo 
%amplification of magnetic field.
%With $R_{\rm m}=7.4$ and $P_{\rm m}=2.5$ dynamo is suppressed and 
%amplification is limited to tangling of the magnetic field.
%Panel (a) displays mean magnetic energy density, $e_B$, evolving as a
%proportion of time-averaged kinetic energy density, $\overline{e_K}$.
%The inset shows a zoom-in of the early linear growth of the tangled field.
%Time is normalised by the inverse eddy turnover time at the forcing scale,
%$\kf \overline{u_{\rm rms}}$.
%Compensated power spectra are displayed of magnetic energy for the model with
%SSD (b) and with tangling (c).
%The legend shows the normalised times for each spectrum.
%The forcing scale, $\kf=8$, is indicated by the vertical dotted line.
%$k_1=L/(2\pi)$, where $L$ is the largest length scale in the simulation domain,
%with the largest wavenumber $k/k_1=128$.
%FAG streamlining
Panel (a) displays mean magnetic energy density, $e_B$,
evolving due to non-helical random forcing, scaled to time-averaged kinetic
energy density, $\overline{e_K}$.
The inset shows a zoom-in of the early linear growth of the tangled field.
%Time is normalised by the inverse eddy turnover time at the forcing scale,
%$\kf \overline{u_{\rm rms}}$.
%NS: modified by removing "inverse"
Time is normalised by the eddy turnover time at the forcing scale,
$1/\kf \overline{u_{\rm rms}}$.
%NS: do we need overline over u_rms, as Rm definition (unnumbered) doesn't have it either
%NS: panel a above has it with overbar; its just looks a bit odd
Sample compensated power spectra at times indicated in the legends are
displayed for the model with SSD (b) and with tangling (c). 
%The forcing scale, $\kf=8$, is indicated by the vertical dotted line.
%NS: kf => kf/k1; is it better to mention other parameters Rm etc in the caption?
The forcing scale, $\kf/k_1=8$, is indicated by the vertical dotted line.
\label{fig:tangling}}
\end{figure*}


%==============================================================================
\section{Disentangling the dynamo} \label{sec:ssd-tang}
%==============================================================================

%MJK We should introduce abbrv. SSD early on and then use it consistently.
%To illustrate some differences between tangling and small-scale dynamo we adopt
%a simplified model with non-helical random forcing with wavenumber $\kf=8$
%applied to isothermal uniform density in $256^3$ $2\pi$-periodic boxes.
%NS: used SSD; kf => kf/k1; and modified a bit
%FAG: OK
To illustrate some differences between tangling and SSD we adopt
a simplified model with non-helical random forcing with wavenumber $\kf/k_1=8$
applied to isothermal uniform density in $256^3$, $2\pi$-periodic boxes with
$k_1=1$ being the lowest wavenumber in the domain.
A uniform field with energy density $e_B\simeq6\cdot10^{-22}\overline{e_K}$ is
imposed, where $\overline{e_K}$ is the time-averaged kinetic energy density.
The two simulations are distinguished only by use of dimensionless
$\eta=10^{-4}$, exciting an SSD, and $\eta=2\cdot10^{-3}$
subcritical for dynamo.
%FAG: moved from caption
%The simulation with $R_{\rm m}=148.0$ and $P_{\rm m}=50.0$ supports dynamo 
%amplification of magnetic field.
%With $R_{\rm m}=7.4$ and $P_{\rm m}=2.5$ dynamo is suppressed and 
%amplification is limited to tangling of the magnetic field.
These yield $R_{\rm m}=148.0$ with $P_{\rm m}=50.0$ for the SSD and
$R_{\rm m}=7.4$ with $P_{\rm m}=2.5$ where the dynamo is suppressed and 
amplification is limited to tangling of the imposed magnetic field.
%
Our numerical implementations use the {\sc Pencil Code}\footnote{
\href{https://github.com/pencil-code}{https://github.com/pencil-code}}.

We plot some of the diagnostics from the results of these two simulations in 
Figure\,\ref{fig:tangling}.
Panel (a) illustrates that the SSD is characterised by
exponential growth just over 400 eddy turnover times; see \cite{ZRS83} for the
properties and excitation conditions of SSD.
Tangling results only in linear amplification early on (see inset), saturating
within 100 eddy turnover times at below 5 times the imposed field energy
density.

In panel (b) we plot for the SSD model some power spectra for the magnetic
energy, evolving over time alongside a kinetic energy spectrum at late stage.
%FAG: moved from caption
$k$ is normalised by $k_1=L/(2\pi)$, where $L$ is the largest length scale in
the simulation domain, with the largest wavenumber $k/k_1=128$.
%
The magnetic energy spectra are compensated by $k^{-3/2}$ and kinetic by
$k^{5/3}$.
A profile becoming horizontal, therefore, corresponds to the Kazantsev's
$3/2$ scaling \citep{Sch02,BS14} or the Kolomogorov turbulent energy spectrum.
In panel (c) see we show similar for the tangling model.
The forcing scale $\kf=8$ is evident in the kinetic spectra and highlighted by 
a vertical dotted line.
The SSD magnetic spectrum (b) evolves near self-similarly, with an uncompensated
peak wavenumber above 20 at early times reducing to below 20 upon saturation of
the dynamo.
The forcing scale has negligible effect on the magnetic spectra.
However, for the tangling case (c) the peak wavenumber for the magnetic energy
spectrum is strongly identified with the forcing scale.
The Kazantsev range of the spectrum extends to wavenumbers larger than the 
forcing scale for SSD, while confined to larger scales for the field with
tangling only.

Given the high magnetic Prandtl numbers, the magnetic spectra retain more energy
%NS: is it high or low Pm meant in the line above (abstract has low only)
%NS: but i agree that at high Pm, M_k > E_k at large k
%FAG: in these experiments the PM are high, but in the SN runs they are low Pm
%FAG: although we have some runs with high Pm for inclusion in the full paper.
%MJK This is certainly a point that the referee might critisize (ask us to repeat the simplified model at low Pm). But we can worry about that then. 
at smaller scales than the kinetic spectra.
Although the kinetic parameters are identical and are subject to the same 
viscous cutoff, the SSD extends the kinetic energy into the 
smaller scales
due to the feedback from the Lorentz force.
With SSD there is a short inertial range for $10\leq k/k_1\lesssim 15$.
%MJK Is there not also for the tangling case?
%MJK Why in Fig 1 the y-scales are so much different when eb/ek is close to one?
%FAG the spectra are integral quantities and all the kinetic energy is concentrated around k=8, which biases the peak value for eK and also there are the k-power prefactors
Thus, in the dynamo kinetic energy deposited along the Kolmogorov range to
smaller scales transfers energy to the magnetic field at these scales
inducing an inverse Kazantsev range begining at scales below the forcing
scale, while with tangling the energy transfers to the magnetic field
only at scales between the forcing scale and scale of the imposed field.
There is only dissipation of the field at scales smaller than the Kazantsev 
range.

\section{SN turbulence model design} \label{sec:model}

To exclude any large-scale magnetic field dynamics in these simulations, the
%FAG: correction
%computational domain occupies a cube of length 120 pc, with periodic boundaries
computational domain occupies a cube of length 256 pc, with periodic boundaries
on all sides.
%FAG: added
In particular rotation, shear and stratification of the simulation domain are
not included.
There is no imposed field, in contrast to \citet{BKMM04}, so we can exclude
tangling as a source of amplification of the sub-nG initial random seed field.
%FAG
Grid size $\delta x=0.5$, 1, 2 and 4$\pc$  along each side are considered.
We solve the system of non-ideal compressible MHD equations, including 
mass Eq\,\eqref{eq:mass}, momentum Eq\,\eqref{eq:mom}, energy Eq\,\eqref{eq:ent} and
induction Eq.\,\eqref{eq:ind}:
%NS: may be better to cite some paper here (Gent+2020?)
%FAG: We are using a different form of the equations, which justify giving them explicitly, we include for the first time hyper diffusion and exclude for the first time shock resistivity and laplacian thermal diffusivity
%-------------------------------------------------------------------------------
  \begin{eqnarray}
  \label{eq:mass}
  %NS: shouldn't it be del / del t instead of D/Dt in line below?
  %FAG: no in this case it is the material derivative on the left
    \frac{D\rho}{Dt} &=& 
    -\vect\nabla \cdot (\rho \vect{u})
    +\vect\nabla \cdot\zeta_D\vect\nabla\rho,
  \end{eqnarray}
%-------------------------------------------------------------------------------
  \begin{eqnarray}
  \label{eq:mom}
    \rho\frac{D\vect{u}}{Dt} &=& 
    \vect\nabla{\ESK\sigma}
    -\rho c_{\rm s}^2\vect\nabla\left({s}/{c_{\rm p}}+\ln\rho\right)
    +\vect{j}\times\vect{B}
    \nonumber\\
    &+&\vect\nabla\cdot \left(2\rho\nu{\mathbfss W}\right)
    +\rho\,\vect\nabla\left(\zeta_{\nu}\vect\nabla \cdot \vect{u} \right)
    \nonumber\\
    &+&\vect\nabla\cdot \left(2\rho\nu_3{\mathbfss W}^{(3)}\right),
  \end{eqnarray}
%-------------------------------------------------------------------------------
  \begin{eqnarray}
  \label{eq:ent}
    \rho T\frac{D s}{Dt} &=&
     \EST\dot\sigma +\rho\Gamma
    -\rho^2\Lambda +\eta\mu_0\vect{j}^2 
    \nonumber\\
    &+&2 \rho \nu\left|{\mathbfss W}\right|^{2}
    +\rho\,\zeta_{\nu}\left(\vect\nabla \cdot \vect{u} \right)^2
    \nonumber\\
    &+&\vect\nabla\cdot\left(\zeta_\chi\rho T\vect\nabla s\right)
    +\rho T\chi_3\vect\nabla^6 s,
  \end{eqnarray}
%-------------------------------------------------------------------------------
  \begin{eqnarray}
  \label{eq:ind}
    \frac{\partial \vect{A}}{\partial t} &=&
    \vect{u}\times\vect{B}
    +\eta\vect\nabla^2\vect{A}
    +\eta_3\vect\nabla^6\vect{A},
  \end{eqnarray}
%-------------------------------------------------------------------------------
completing the system with the equation of state for an ideal gas.
Many symbols have their usual meaning, and a table (\ref{sec:table}) is
included in the Appendix for clarity.
In particular, terms containing $\zeta_D,\,\zeta_\nu$ and $\zeta_\eta$ are the
application of artificial diffusion acting proportional to shocks in 
Eqs.\,\eqref{eq:mass},\,\eqref{eq:mom} and \eqref{eq:ent}, respectively.
This machinery is described in detail in \citet{GMKSH20}, which is included to
resolve shock discontinuities.
In contrast to \citet{Gent:2013b} shock diffusion is not applied to
Eq.\,\eqref{eq:ind}, where it would result in unphysical dissipation of the
magnetic field in the shock fronts, when in fact in the shocks it is highly
enhanced by compression.
In Eqs.\,\eqref{eq:mom} -- \eqref{eq:ind}, expressions containing 
$\nu_3,\,\chi_3$ and $\eta_3$ apply sixth-order hyperdiffusion applying
primarily at the grid scale to resolve small-scale instabilities
\citep[see, e.g.,][]{ABGS02,HB04}.

%-------------------------------------------------------------------------------
\begin{figure*}
\gridline{\fig{figs/eB-res-4eta.png}{0.45\textwidth}{(a)}
          \fig{figs/eB-res-3eta.png}{0.45\textwidth}{(b)}
          }
\caption{
The volume averaged magnetic energy density for models with $\dx$ between
$0.5\pc$ and $4\pc$ are plotted over time.
These are scaled by reference to their time-averaged statistical-steady kinetic
energy density.
Resistivity, $\eta=10^{-4}\kpc\kms$ in panel {\rm(a)} and $10^{-3}$ {\rm(b)}, is
applied.
\label{fig:eb-res}}
\end{figure*}
%-------------------------------------------------------------------------------

The SNe are injected randomly uniform in 3D space and as a Poisson
process proportional to the solar neigbourhood rate in the Milky Way,
 $\SNr\simeq 50\kpc^{-3}\Myr^{-1}$.
Each explosion injects $10^{51}\erg$ as thermal energy, $\EST$, but in dense
media a small proportion may be in kinetic energy, $\ESK$.
This is described in detail by \citet{GMKSH20}.
To reduce statistical noise, at all $\dx$ and $\eta$ the SN are
scheduled at the same time and coordinates for $\dot\sigma=0.2\SNr$ and for 
correspondingly for $\dot\sigma=\SNr$.
Non-adiabatic heating, $\Gamma$, and cooling, $\Lambda$, processes are included
following \citet{Wolfire:1995} and \citet{Sarazin:1987}, as described in 
\citet{Gent:2013a}.

In the subset of experiments presented in this letter we set viscosity, $\nu=0$,
with $\nu_3$ applying optimally for each $\dx$ to ensure the flow is well
resolved.
We establish a benchmark for the magnetic field evolution setting $\eta=0$, with
numerical resistivity acting via the optimal setting for $\eta_3$.
We then vary $\eta\geq10^{-6}\kpc\kms$ to identify minimal range for which the
physical resistivity is dynamically dominating the numerical resistivity, and
examine the dependence of the dynamo on $\eta$ and grid resolution.
Henceforth, where $\eta$ is specified for the SN simulations it has units of
$\kpc\kms$. 
In contrast to our earlier experiments \citep{Gent:2013a,Gent:2013b,GMKSH20},
we do not include thermal diffusivity, $\chi$, in these experiments as the
artificial diffusivities are adequate to ensure numerical stability and the
physical effects of thermal conductivity can be expected to be relevant only
at time and spatial scales much shorter than considered here.
\fag{Mordecai, could you suggest some appropriate reference?}

An important 
parameter for the excitation of the SSD
is the magnetic Reynolds number, Rm,
%and the magnetic Prandtl number, $\Pm = \Rm/\Rey \simeq \nu/\eta$, where $\Rey$
%NS: why simeq? changed it to = for now to keep it standard
and the magnetic Prandtl number, $\Pm = \Rm/\Rey = \nu/\eta$, where $\Rey$
is the fluid Reynolds number.
A commonly used definition is \[\Rm=\frac{\ell u_{\rm rms}}{\eta},\] where
$\ell$ is the forcing scale.
However, in these simulations of the ISM, given the 
%MJK ?
%FAG opposit of what I meant - reworded
%Galilean invariance of the 
inhomogeneity of the 
turbulence, variation within hot and warm medium, and inclusion of
hyperdiffusion, defining a single forcing scale from the SN explosions or 
common rms velocity is unreliable.
%MJK "a field" sounds alien to me
%FAG: absolutely no alienation allowed
%We, therefore, directly compute a field of Reynolds numbers from the equations,
We, therefore, compute Reynolds numbers at each grid point
directly from the equations,
for example,
%-------------------------------------------------------------------------------
\begin{eqnarray}\label{eq:Rm}
  \Rm = \frac{\left|\vect{u}\times\vect{B}\right|}{
    \left|\eta\vect\nabla^2\vect{A}+\eta_3\vect\nabla^6\vect{A}\right|}.
\end{eqnarray}
%-------------------------------------------------------------------------------
%NS: should LHS in the above equation be <Rm> ?
%FAG: at each grid poitn we compute Rm. From the distribution we compute <Rm> whatever that tells us!!
%Hence, for the high resolution runs at 20\,Myr we obtain linear
%NS: by "linear", is it meant of order unity?
%FAG: no just not a log norm or exp norm distribution, but this is not clear and no need to go into it unless we compare different norms, so
Hence, for the high resolution runs at 20\,Myr we obtain for $\eta=10^{-4}$ 
%NS: Rm = 4 by directly using eq 5? It might be good to have a number for also
%NS: the standard definition, so that we could simply say, 4, 15000 using Eqs (), (5).
%MJK Everything goes nuts when one starts looking at the Rms. Still I ask: what would be our Rm_crit according to these definitions?
%NS: I have commented on fig3a that 10^-4 is close to 0, so perhaps not resolved?
%NS: A rough estimate of Rm_crit (based on Eq 5) is perhaps from eta=10^-3 case,
%NS: which is perhaps marginal?
%FAG: I think maybe we need to consider further how eta_3 actually applies to the effective Rm, but not here
$\langle\Rm\rangle=4.1$ with standard deviation 5.7, whereas 
$\Rm = 15000$, for $\eta=10^{-4}$, $\ell=50\pc$ and $u_{\rm rms}=30\kms$ using
the common formula.
%FAG: added
The spread of Rm, using equation\,\eqref{eq:Rm}, extends well above 15000 and to
$\ll1$.
Given the multiphase structure of the ISM, the mean Rm appears unhelpful and 
the concept of a critical Rm for SSD may differ between phases and regions.
In the real ISM such variation in Rm will be present.
Microscopic resistivity may be temperature dependent \citep{CSR50}, but even if
we consider only the turbulent resistivity as relevant to the definition of 
Rm, this will also vary between phases due to the differences in typical 
length scales and velocities.
We shall address this challenge in future work, but here we shall use the 
explicit input $\eta$ rather than Rm to discriminate between models.
%

In this suite of experiments, with $\nu=0$ and $\nu_3=\eta_3$ sufficient to
numerically resolve the grid scale, $\Pm<1$, which is an even more difficult
regime in which to excite the SSD than in the high $\Pm$
regime typical of the ISM \citep{HBD04}.
In separate experiments with high Pm, which we shall include in the more in
%depth future analysis, we do in fact see increased growth rates for higher Pm.
%NS: added citations, both Pm<1 (2007 paper) and Pm>1 (2002 paper); inserted "as expected"
%NS: 2007 paper is good enough i guess (fig 1 there)
depth future analysis, we do in fact see increased growth rates for higher Pm,
as expected \citep{Sch07}.
%MJK Reference(s) to this expectation needed.
%FAG: done
%FAG: added
As with Rm in this system we have a range $Pm\ll1$ and $Pm\gg1$, due to the
inclusion of shock capturing viscosity and differences in the definition of
hyper viscosity to hyper ressitivity.
Hence, in all models some part of the domain will be characterised by high Pm,
and how this effects interpretation of the SSD properties shall also be
addressed in later work.
%
%-------------------------------------------------------------------------------
\section{Resolution and resistivity} \label{sec:results}
%-------------------------------------------------------------------------------

%-------------------------------------------------------------------------------
\begin{figure*}
\gridline{\fig{figs/0_5pc-eB-nu4.png}{0.45\textwidth}{(a)}
            \fig{figs/1pc-eB-nu4.png}{0.45\textwidth}{(b)}
          }
\gridline{  \fig{figs/2pc-eB-nu4.png}{0.45\textwidth}{(c)}
            \fig{figs/4pc-eB-nu4.png}{0.45\textwidth}{(d)}
          }
\gridline{
            \fig{figs/2pc-eB-nu5.png}{0.45\textwidth}{(e)}
            \fig{figs/4pc-eB-nu6.png}{0.45\textwidth}{(f)}
          }
  \begin{picture}(0,0)(0,0)
    \put( 56,428){\begin{scriptsize}{\sf{$\delta x=0.5$pc, $\sigma=0.2\SNr$}}\end{scriptsize}}
    \put(445,438){\begin{scriptsize}{\sf{$\delta x=1.0$pc}}\end{scriptsize}}
    \put(445,428){\begin{scriptsize}{\sf{$\sigma=0.2\SNr$}}\end{scriptsize}}
    \put( 56,256){\begin{scriptsize}{\sf{$\delta x=2.0$pc, $\sigma=0.2\SNr$}}\end{scriptsize}}
    \put(445,266){\begin{scriptsize}{\sf{$\delta x=4.0$pc}}\end{scriptsize}}
    \put(445,256){\begin{scriptsize}{\sf{$\sigma=0.2\SNr$}}\end{scriptsize}}
    \put( 56, 78){\begin{scriptsize}{\sf{$\delta x=2.0$pc, $\sigma=\SNr$}}\end{scriptsize}}
    \put(405, 78){\begin{scriptsize}{\sf{$\delta x=4.0$pc, $\sigma=\SNr$}}\end{scriptsize}}
  \end{picture}
\caption{
%The effect of resistivity $\eta$ is compared at each $\dx$, {\rm(a)} -- {\rm(d)}
%for SN rate $\dot\sigma=0.2\SNr$ and {\rm(e)} -- {\rm(f)} for
%$\dot\sigma=\SNr$ at lower resolution, where
%$\SNr\simeq 50$\,kpc$^{-3}$\,Myr$^{-1}$ is the solar neighbourhood equivalent
%random SN frequency.
%The time axes vary between plots sufficient to reach saturation of the dynamo.
%At $\dx=4\pc$ the models with $\eta=10^{-4}$ are continued until the dynamo has
%saturated, for comparison with the higher resolution saturation levels.
%FAG: streamlining
The effect of resistivity $\eta$ is compared for given $\dx$ and $\dot\sigma$.
$\SNr\simeq 50$\,kpc$^{-3}$\,Myr$^{-1}$ is the solar neighbourhood equivalent
random SN frequency.
The time axes vary sufficiently to attain SSD saturation at all $\dx$
with $\eta=10^{-4}$ to allow comparison across resolution.
\label{fig:eb-nu}}
\end{figure*}
%-------------------------------------------------------------------------------

In Figure\,\ref{fig:eb-res} the response of the magnetic energy density 
$e_B/\overline{e_K}$ to resolution is shown for resistivity $\eta=10^{-4}$,
panel\,(a) and $\eta=10^{-3}$, panel\,(b).
The time in Myr is plotted on a log scale to better present the differences in 
the higher resolution runs, which saturate much faster than at $\dx=4\pc$.
Whilst for $\eta=10^{-4}$ the dynamo growth rates are very slow at $\dx=4\pc$,
the growth rate shows some convergence near $\dx=0.5\pc$ and the saturation
level of 
%MJK around instead of above? Slightly above?
above 5\% of $\overline{e_K}$ is independent of resolution.
%MJK How do we know that we currently capture the correct one?
%FAG Not sure to what 'one' you refer?
%MJK To false/true convergence?
At $\eta=10^{-3}$, there is a \emph{false convergence} \citep{FMA91} between
$\dx=2$ and $4\pc$, with the magnetic energy decaying at similar rates, however
at higher resolution we see that the results converge around an alternative
solution, with a dynamo amplification of the field occurring between
%400 and 600\,Myr.
%NS: perhaps you meant 40 and 60; changed, but pls check
40 and 60\,Myr.

In Figure\,\ref{fig:eb-nu} we show how at each resolution $e_B/\overline{e_K}$
evolves for various values of $\eta$.
We note that at $\dx=0.5$, panel (a), and $1\pc$, panel (b), the profile at
$\eta=10^{-5}$ is indistinguishable from $\eta=0$, and so numerical resistivity
continues to control the dynamics.
At $\eta=10^{-4}$ the dynamo diverges from $\eta=0$, such that physical
%FAG: take out Rm
%resistivity is dynamically dominant, at least for some of the domain and Rm is
%unevenly defined by a combination of physical and numerical resistivity.
resistivity is dynamically dominant, at least for some of the domain and
unevenly determined by a combination of physical and numerical resistivity.
%FAG: removing Rm
%At all $\dx$ $\eta=10^{-3}$ is clearly well resolved and dynamics and
%Rm well defined by the physical application of resistivity, varying locally with
%$3.8\lesssim\langle\Rm\rangle\lesssim6$, as discussed above from the induction
%equation of $\Rm=15000$ using the common definition. 
At all $\dx$ $\eta=10^{-3}$ is clearly well resolved and dynamics
well defined by the physical application of resistivity.
Although models with the same SN rate have the same schedule and location of
SN, at low resolution the timestep is longer, such that the actual timing and 
environment of the explosions can differ between models, so the statistical
noise is more evident, particularly in panels (e) and (f).


%MJK Would help if it was not introduced as a complication, but explained first as behavior seen in the figure, then given an explanation, and then convincing the reader that what is seen now only in Figs 3 a,b as puzzling, we can easily explain in these terms.

%FAG: added
%The evolution is complicated by the impact \emph{thermal runaway}
%\citep[see e.g.,][]{LOCBN15}, combination of multiple SN into superbubbles
%resulting in persistent hot regions.
Amongst the general suite of simulations in addition to those presented in this
letter, some models exhibit the phenomenon of \emph{thermal runaway}
\citep[see e.g.,][]{LOCBN15}.
The hot gas fills a sufficient fractional volume, that the supernova remnants 
form bubbles, which cannot cool rapidly and the domain becomes saturated 
with hot gas.
Simulations with high resolution are more easily perturbed into this regime, but
also low resolution runs with higher $\dot\sigma$.
We observe that during some phases of accelerated dynamo growth, the magnetic
field is growing most rapidly in the hot gas, whilst during majority of the 
duration of the runs magnetic fields grow most rapidly in the warm gas and the
dynamo growth overall follows a lower exponential.
This is one reason for the complication in characterising the turbulence in 
these SSDs, with Rm, Pm and $u_{\rm rms}$ varying is space and over time.

%FAG removing Rm
We shall report analysis of these results in depth in future work, but an 
initial observation of the growth profiles could lead to the conclusion that
%growth rates for high resistivity (low Rm) can exceed comparable rates for high
%Rm, in contradiction to dynamo theory and previous experimental results.
growth rates for high resistivity can exceed comparable rates for low $\eta$,
in contradiction to dynamo theory and previous experimental results.
Closer inspection of each plot, zooming in on specific intervals, however,
%FAG :changed to eta, as table in paper suggest mean Rm cannot explain this
%confirms that growth rates are higher as expected for higher Rm.
confirms that growth rates are higher as expected for lower $\eta$.
%MJK Hmmh...
%MJK Which figure?
\fag{Fred, revisit this next statement}
%FAG: added Fig
%For example between 8 and 15 Myr there is weak growth or decay in panels (a)
For example between 8 and 15 Myr there is weak growth or decay in panels (a)
and (b) with critical $\eta$ for dynamo between $10^{-4}$ and $10^{-3}$.
Rm and corresponding growth at low $\eta$ is higher in the $0.5\pc$ simulation
with lower numerical resistivity, but 
converge for $\eta=10^{-3}$, where the physical resistivity is most resolved.
%FAG: fast replaced with acceleration, etc.
%Between 15 and 20\,Myr there is a fast dynamo driven in the hot gas, in which
Between 15 and 20\,Myr there is an acceleration in the dynamo driven in the hot
gas, in which growth rates are related to the level of resistivity.
%MJK Rm is not given for all models... difficult to know what are you referring to...
%FAG removing Rm
%The high Rm models at $0.5\pc$ already saturate by 20\,Myr, but for $1\pc$ there
The low $\eta$ models at $0.5\pc$ already saturate by 20\,Myr, but for $1\pc$ there
%FAG: lower
%is another slow dynamo driven in the warm gas up to 40\,Myr and then a
%subsequent 'hot' dynamo resulting in saturation for all models within 60\,Myr,
is another lower growth rate for the dynamo driven in the warm gas up to
40\,Myr and then a subsequent accelerated dynamo resulting in saturation for
all models within 60\,Myr, including for $\eta=10^{-3}$.

%MJK Now we have too many different names for this dynamo: hot, fast, and what not. Should choose one.
%FAG: using terms of accelerated growth and lower growth rate
At low resolution, with $\dot\sigma=0.2\SNr$ panels (c) and (d), $\eta=10^{-4}$
%FAG not hot
%is not distinct from numerical resistivity, and there is no hot dynamo at all
is not distinct from numerical resistivity, and there is no separate
acceleration of the dynamo at all for $\dx=4\pc$ with the well resolved $\eta=10^{-3}$.
%FAG not hot
%At 100\,Myr there is a hot dynamo at $\dx=2\pc$, but this is not
At 100\,Myr there is a period of accelerated dynamo at $\dx=2\pc$, but this is
not sustained and for $\eta=10^{-3}$ susbequently diffuses away.
%FAG removing Rm
%At $\dot\sigma=\SNr$, panels (e) and (f), the increased Rm due to the 
At $\dot\sigma=\SNr$, panels (e) and (f), the 
higher forcing rate is sufficient to produce a dynamo at $\eta\geq10^{-3}$, but
not for $\eta=5\cdot10^{-3}$.
Given the increased statistical noise we cannot confirm that $\eta\leq10^{-4}$
applies beyond the numerical resistivity.
However, at $\dx=2\pc$ $\eta=5\cdot10^{-4}$ (magenta, solid) is resolved and at $\dx=4\pc$
 the resolved limit is $\eta\lesssim10^{-3}$ (cyan, dotted).
%FAG not hot
%In panel (e) we see two hot dynamo phases at 90  and 110\,Myr, which are not 
In panel (e) we see two phases of accelerated dynamo growth at 90  and
110\,Myr, which are not  present in panel (f).
%FAG: moved to summary
%There is a reduced susceptibility to thermal runaway at low resolution, 
%and only lower Rm can be resolved, both an effect of more diffusive numerics.
%Hence, as the resolution becomes more coarse it is likely that the small-scale 
%dynamo will be suppressed in simulations of the ISM.
%Higher SN rates driving increased velocities may permit coarse models 
%to attain the critical Rm for SSD, but it is likely that models with 
%$\dx$ much above $4\pc$ will suppress the dynamo. 
%
%To attain a truly convergent numerical regime simulations should apply 
%resolution better than $2\pc$.
%In the ISM with Rm much higher than we can resolve, our results confirm that
%SSD is very easy to excite by SN, and may even be present in galaxies with
%low SN rates.

%%FAG para break
%%FAG: moved to summary
%As shown in panels (a) and (b), properties of the hot gas cannot exclude
%SSD even with high resistivity in galaxies that produce sustained regions of
%hot turbulence.
%%FAG not fast
%%We shall examine the nature of this fast dynamo in a later report.
%We shall examine the nature of acceleration of the dynamo in a later report.
%To exclude SSD in SN driven turbulence simulations will require
%$\eta\gtrsim10^{-3}$ or SN implementation which avoids thermal runaway.
%In \citet{Gressel:2008,GE20}, $\dx$ is 8.3 and $6.7\pc$, respectively
%and $\eta\simeq6.5\cdot10^{-3}\kpc\kms$, which excludes SSD,
%while \citet{Gent:2013b} applied $\eta\simeq8\cdot10^{-4}\kpc\kms$ with $\dx=4\pc$,
%which would support SSD, even without thermal runaway.
%The latter obtain a LSD with galactic angular momentum
%$\Omega=\OSN$, where $\OSN=25\kms\kpc^{-1}$ is the rate in the solar
%neighbourhood, while the former require $\Omega\geq4\OSN$ to excite LSD.
%Rm applying at the largest scales would be 7.5 times higher for the latter 
%model, so alone is sufficient to explain the LSD.
%How the SSD affects the galactic dynamo requires further study.
%\fag{add discussion of Ulrich's results}

%-------------------------------------------------------------------------------
\begin{figure*}
\gridline{\fig{figs/0_5pcPm0e-4_0Bpower.png}{0.45\textwidth}{(a)}
          \fig{figs/0_5pcPm0e-4_0kpower.png}{0.45\textwidth}{(b)}
          }
\gridline{\fig{figs/0_5pcPm0e-3_0Bpower.png}{0.45\textwidth}{(c)}
          \fig{figs/0_5pcPm0e-3_0kpower.png}{0.45\textwidth}{(d)}
          }
\caption{
%Compensated energy spectra at times in Myr given in the legends for 
%$0.5\pc$ resolution.
%Rm is super critical for dynamo applying $\eta=10^{-4}$ in
%panels (a) and (b) and sub critical or marginal for dynamo applying
%$\eta=10^{-3}$ in panels (c) and (d).
%Energy spectra are compensated against theoretical profiles of Kazantsev
%$k^{3/2}$, (a) and (c), and Kolmogorov $k^{-5/3}$, (b) and (d), 
%each represented by the horizontal black dashed lines.
%FAG: streamlining
%Compensated energy spectra for $\dx=0.5\pc$ at time in Myr listed in the
%legends.
%$\eta=10^{-4}$ (a), (b) is supercritical for SSD and $\eta=10^{-3}$ (c),(d) 
%is subcritical or marginal for SSD.
%Energy spectra are compensated against theoretical profiles of Kazantsev
%$k^{3/2}$, (a) and (c), and Kolmogorov $k^{-5/3}$, (b) and (d).
%NS: minor changes
Compensated energy spectra for $\dx=0.5\pc$ at time in Myr listed in the
legends; $\eta=10^{-4}$ (a, b) is supercritical for SSD, and $\eta=10^{-3}$ (c, d) 
is subcritical or marginal for SSD.
Energy spectra are compensated against theoretical profiles of Kazantsev
$k^{3/2}$ (a, c), and Kolmogorov $k^{-5/3}$ (b, d).
%
%NS: in (c), has the spectrum saturated by t=9 Myr? If so, then earlier
%NS: time may also be included; or may be there was growth at very late time
%NS: as indicated from Fig 3a, between 40-60 Myr.
%NS: have not yet removed "subcritical" from above, but maybe we should
%NS: as in that case there will be a clear decay; here only high res case discussed, right?
\label{fig:4power}}
\end{figure*}
%-------------------------------------------------------------------------------

In contrast to the simplified model in Section\,\ref{sec:ssd-tang}, SN driven
turbulence does not have a well-defined forcing scale, due to the heterogeneous
temperature and density of the ISM and random clustering of explosions.
In our simulations the forcing scale will range between
%MJK How did you come up with this value?
10 and $50\pc$,
or $k\in(3,16)$.
In Figure\,\ref{fig:4power} we show the evolving compensated spectra for the
high resolution kinetic energy in the kinematic phase (b) for $\eta=10^{-4}$
and during decay (d) for $\eta=10^{-3}$.
%MJK You mean here the kinetic spectra moving up and down in magnitude, or what?
The spectra magnitude between the models show the same fluctuations over time,
except at 32\,Myr, when saturation of the dynamo, (b), reduces
the energy slightly, compared to (d). 
%Many of the spectra display a \emph{bottleneck} effect
%NS: removed "of the"
Many spectra display a \emph{bottleneck} effect
\citep{Falkovich94,HBD03} with a peak at $k\simeq50$.
%MJK Bump would be more descriptive.
The compensated magnetic energy spectra in panels (a) and (c) have ranges 
conforming to the Kazantsev inverse cascade in the SSD case (a) extending to
$k\gtrsim 20$, consistent with the SSD behaviour of the simplified model in
Figure\,\ref{fig:tangling}\,(b).
In panel (c) of Figure\,\ref{fig:4power} the Kazantsev range in similar fashion
to Figure\,\ref{fig:tangling}\,(c) is in the range 
%MJK small k
$K\lesssim10$ except
for 22\,Myr, which corresponds to a short growth spurt in
Figure\,\ref{fig:eb-res}\,(b) and consistent with the presence of SSD.
%FAG: added 
%MJK 'SN' needed?
So for these SN low Pm models any Kazantsev spectral range dissipates 
at $k$ below the extent of Kolmogorov range.
This makes the SSD in the models even less efficient 
%MJK PM -> Pm
than the high PM ISM,
where transfer from kinetic energy can occur at every wavenumber in the
Kolmogorov spectrum.

%-------------------------------------------------------------------------------
\begin{figure*}
\gridline{\vspace{-0.7cm} \fig{figs/nu0_Bpower.png}{0.45\textwidth}{\vspace{-1.3cm}\hspace{-2cm}(a)}
          \vspace{-0.7cm} \fig{figs/nu0_kpower.png}{0.45\textwidth}{\vspace{-1.3cm}\hspace{-2cm}(b)}}                                                                             
\gridline{\vspace{-0.7cm}\fig{figs/nu1_Bpower.png}{0.45\textwidth}{\vspace{-1.3cm}\hspace{-2cm}(c)}
          \vspace{-0.7cm}\fig{figs/nu1_kpower.png}{0.45\textwidth}{\vspace{-1.3cm}\hspace{-2cm}(d)}}                                                                             
\gridline{\vspace{-0.7cm} \fig{figs/nu10_Bpower.png}{0.45\textwidth}{\vspace{-1.3cm}\hspace{-2cm}(e)}
          \vspace{-0.7cm} \fig{figs/nu10_kpower.png}{0.45\textwidth}{\vspace{-1.3cm}\hspace{-2cm}(f)}}                                                                             
\gridline{\vspace{-0.3cm}  \fig{figs/SN_Bpower.png}{0.45\textwidth}{\vspace{-1.3cm}\hspace{-2cm}(g)}
          \vspace{-0.3cm}  \fig{figs/SN_kpower.png}{0.45\textwidth}{\vspace{-1.3cm}\hspace{-2cm}(h)} }
  \begin{picture}(0,0)(0,0)
    \put(165,455){\begin{scriptsize}{\sf{$\eta=0$, $\dot\sigma=0.2\SNr$}}\end{scriptsize}}
    \put(420,455){\begin{scriptsize}{\sf{$\eta=0$, $\dot\sigma=0.2\SNr$}}\end{scriptsize}}
    \put(150,325){\begin{scriptsize}{\sf{$\eta=10^{-4}$, $\dot\sigma=0.2\SNr$}}\end{scriptsize}}
    \put(410,325){\begin{scriptsize}{\sf{$\eta=10^{-4}$, $\dot\sigma=0.2\SNr$}}\end{scriptsize}}
    \put(150,192){\begin{scriptsize}{\sf{$\eta=10^{-3}$, $\dot\sigma=0.2\SNr$}}\end{scriptsize}}
    \put(410,192){\begin{scriptsize}{\sf{$\eta=10^{-3}$, $\dot\sigma=0.2\SNr$}}\end{scriptsize}}
    \put(60,62){\begin{scriptsize}{\sf{$\dx=2\pc$}}\end{scriptsize}}
    \put(315,62){\begin{scriptsize}{\sf{$\dx=2\pc$}}\end{scriptsize}}
  \end{picture}
\caption{
%(a) -- (f): compensated energy spectra at $t=19.5\Myr$ ($\dx=0.5\pc$ \& $1\pc$),
%$t=100\Myr$ ($\dx=2\pc$ \& $4\pc$).
%Resistivity $\eta=0,\,10^{-4}$ and $10^{-3}$ in panel pairs (a,b), (c,d) and
%(e,f), respectively.
%Panels (g,h) for $\dx=2\pc$ at 100\,Myr for $\dot\sigma=0.2\SNr$ and
%140\,Myr for $\dot\sigma=\SNr$ for
%$\eta=10^{-4}$ or $10^{-3}$ as listed in the legends show the effect of 
%SN rate.
%Energy spectra are compensated against theoretical profiles of Kazantsev
%$k^{3/2}$, left panels, and Kolmogorov $k^{-5/3}$, right. 
%FAG: streamlining
Compensated energy spectra as in Figure\,\ref{fig:4power}.
(a) -- (f):
 $t=19.5\Myr$ for $\dx=0.5$ \& $1\pc$ and 100\,Myr for $\dx=2$ \& $4\pc$.
$\dx$ are as listed in the legends.
(g) -- (h): For $\dot\sigma=0.2\SNr$ $t=100\Myr$ and for $\dot\sigma=\SNr$ 
$t=140\Myr$. $\dot\sigma$ and $\eta$ as listed in the legends.
\label{fig:3power}}
\end{figure*}
%-------------------------------------------------------------------------------

In Figure\,\ref{fig:3power} panels (a) -- (f) the effect of resolution on the
compensated energy spectra is shown for $\eta=0$, $10^{-4}$ and $10^{-3}$
with $\dot\sigma=0.2\SNr$.
%FAG: updated 
%The time of the spectral snapshot for $\dx=0.5$ and $1\pc$ correspond to a
%common phase of rapid magnetic growth fastest in the hot gas, while for
%$\dx=2$ and $4\pc$ it corresponds to a period of hot-phase rapid growth at
The time, $19.5\Myr$, of the spectral snapshot
corresponds to a common phase of accelerated magnetic growth for $\dx=0.5$ and $1\pc$.
For $\dx=2$ and $4\pc$ the time 100\,Myr corresponds to a period of accelerated
growth at $2\pc$, which is absent in the $4\pc$ model.
The hydrodynamic parameters are fixed for each resolution and are 
consistent between panels (b), (d) and (f).
We note the convergence in the kinetic energy spectrum for
$\dx=0.5\pc$ and $\dx=1\pc$, apart from the energy cutoff at higher $k$
arising from differences in resolution.
Also, the total kinetic energy is reduced with low resolution due to 
increased viscous dissipation acting at lower $k$.

%FAG para break
%To fully resolve the kinetic energetics for scales larger than $k=24$ corresponding to $\ell\simeq40\pc$ would therefore require $\dx\gtrsim1\pc$.
To fully resolve the kinetic energetics for scales larger than $k=24$, that is
$\ell\simeq40\pc$, would therefore require $\dx\lesssim1\pc$.
The energy spectrum at $\dx=2\pc$ shares the same characterstics as the high
%FAG added 
%resolution runs, but with some loss of energy.
resolution runs, but with some loss of energy, as does $\dx=4\pc$, but with 
even greater dissipative losses and peak $k\simeq3$.
There is a bottleneck evident, an energy cascade less efficient than $k^{-5/3}$
terminating at a peak, before rapid dissipation at high $k$.
This bottleneck shifts to lower $k$ as $\dx$ increases, but always at higher
$k$ than the Kazantsev range in the corresponding magnetic spectrum.

Perhaps surprisingly, there is very little difference in the kinetic energy
spectrum, Figure\,\ref{fig:3power}(h), between simulations with
$\dot\sigma=0.2\SNr$ and $\SNr$, despite five times as much energy being 
applied from the forcing in the latter case.
From the models with $\eta=10^{-3}$ it is evident that there is more energy
near the bottleneck in the high $\dot\sigma$ case (cyan dotted line), so more
energy can be tranferred to the SSD.
The comparison is obscured for $\eta=10^{-4}$, because the high $\dot\sigma$
magnetic field is already 3 or 4 orders of magnitude stronger, aquiring some of
the kinetic energy (red dash-dotted line).
More hot gas from the higher SN rate may also improve the efficiency of SSD,
which we shall investigate in future work.  



\section{Summary of results}\label{sec:conc}

%MJK Now the summary (but it could be also outdated)
%MJK does not really address the main results presented in the paper. 
%MJK 1) SSD can be excited without imposed field present in the system, i.e. the confirmation of Balsara results being not caused by the presence of it, as Detlef thought.
%MJK 2) Convergence (and hopefully a true one) in between 0.5 and 1 pc res, and their implication for the previously published simulations.
%MJK 3) General agreement (hopefully) with the picture of growth rate being somehow related to Rm, and some (maybe forcefully handwavy) estimate of the critical value.
%NS: As Balsara paper talks also about helicity, may be we could briefly say
%NS: that this being SSD is unlikely to be affected by the kinetic helicity
%NS: at least the growth rates.
%MJK But do we know this for sure? Have we looked at the helicity?
%MJK In Miikka's paper we find evidence for kin. helicity making SSD much easier to excite.
%NS: Ok, I recall some papers (also by Axel I think) where helical/non-helical cases
%NS: produced identical growth rates, but this is perhaps minor issue here.
%FAG: added
In this letter we confirm, without the use of an imposed magnetic field, that
the field amplification demonstrated by \citet{BKMM04} was evidence of 
SSD in the ISM and not just caused by
tangling of their imposed field.
Through the most extensive resolution and parameter study to date,
%FAG Moved from above
%mm in the ISM with Rm much higher than we can resolve,
we conclude that SN driven turbulence easily excites SSD even at SN
rates well below the Galactic value.
% mm
     Our conclusion is supported by noting that the resistivity of the
     ISM is far smaller than we can resolve numerically, so the ISM is
     far more susceptible to dynamo action than our models.
Our models with $\dx=0.5$ and $1\pc$ with
$\eta=10^{-3}\kpc\kms$ (see
Figure\,\ref{fig:eb-res}b), 
show that a seed field of less than 1~nG can be amplified to saturation at
microgauss levels within about 10\,Myr.

We further show that simulations with insufficient resolution
     % mm may
     can
appear to converge to a false solution
%mm . These
     lacking dynamo activity.  This can occur because these     
simulations are not scale independent.
The SN energy input and the physically motivated ISM cooling processes impose
length and time scales that must be adequately resolved.
%FAG Moved from above
To
%mm attain a truly convergent numerical regime simulations should apply resolution at least as fine as $1\pc$.
      reach true convergence requires resolution of $1\pc$ or better.
In our models, resolutions $\dx\geq2\pc$ not only give an incorrect dynamo
solution, but also exhibit significant kinetic energy losses due to excess
dissipation.
%
%FAG: added
We do, however find, independent of resolution, that when an SSD is excited it
saturates at about 5\% of the energy equipartition level.
At low resolution this is a lower bound, because the time-averaged
kinetic energy density is understated, due to dissipative losses.
This might account for the discrepency between the energy density of the mean
magnetic field and the random magnetic field in \citet{Gent:2013b}, due to the
LSD being well resolved while the SSD remained under resolved.

%FAG: added
We find that the conventional approach from dynamo theory to categorise the 
turbulence according to Rm based on a forcing scale $\ell$, random velocity and
resistivity, is inadequate for such a complicated system.
The ISM appears to host at least two regimes of SSD, apparently
embedded in the hot or warm phases, and with radically different 
dynamo characteristics and growth rates, and which occupy changing fractional
volumes and epochs.
This is further exacerbated by the susceptibility to thermal runaway, which 
alters the SSD properties.
Explaining the interaction of these SSDs will require more sophisticated
statistical and perhaps toplogical techniques, in advance of being able to 
address how such SSD interacts with LSD.
%FAG Moved from above
Due to the more diffusive numerics there is a reduced susceptibility to
thermal runaway at low resolution. 
Higher SN rates may permit coarse models to enter thermal
runaway and more easily excite SSD, but it is likely that models with 
$\dx$ much above $4\pc$ will suppress SSD in simulations of the ISM.

%FAG Moved from above
To exclude SSD in SN driven turbulence simulations will require
$\eta\gtrsim10^{-3}\kpc\kms$ or SN implementation which avoids thermal runaway.
In \citet{Gressel:2008,GE20}, $\dx$ is 8.3 and $6.7\pc$, respectively
and $\eta\simeq6.5\cdot10^{-3}\kpc\kms$, which in all probabilty excludes SSD.
\citet{Gent:2013b} applied $\eta\simeq8\cdot10^{-4}\kpc\kms$ with $\dx=4\pc$,
which would support SSD with SN rates similar to the solar neighbourhood, even
without thermal runaway.
The latter obtain a LSD with galactic angular momentum $\Omega=\OSN$, where
$\OSN=25\kms\kpc^{-1}$ is the rate in the solar neighbourhood.
The former require $\Omega\geq4\OSN$ to excite LSD.
Rm applying at the largest scales would be 7.5 times higher for the latter 
model, so alone is sufficient to explain the LSD at lower $\Omega$.
%FAG: added
Early efforts by \citet{Korpi:1999b} were unable to detect even the LSD.
With a resolution even less than \citet{Gressel:2008} $\Omega\gg\OSN$ would be 
required to excite the LSD, and a SSD would be ruled out.
%
How the SSD affects the galactic dynamo requires further study.
%FAG added ???? Not sure this is actually so relevant having read it? Though I find it puzzling that they have SSD but no LSD
Using an SPH code \citet{SBADMN19} report an SSD in a global galaxy simulation
with minimum resolution equivalent to about 20\,pc.
In this model the SN are not directly modelled and the turbulence is driven
on a much larger forcing scale by the galactic wind and fountain.
The disk height is relatively small-scale in the model, so the LSD model
to which this letter is a preliminary step, might appropriately fit into
design of a subgrid-scale model for their halo model.
%

%FAG Moved from above
As shown in Figure\,\ref{fig:eb-nu}(a) and (b), turbulence 
can excite accelerated SSD even with high resistivity in galaxies that
sustain regions of hot gas.
To understand this we shall examine the nature of the acceleration of the
dynamo in a later submission.





\acknowledgments
The authors wish to acknowledge CSC – IT Center for Science, Finland, for computational
resources.
 M-MML was partly supported by US NSF grant AST18-15461.
 FAG and MJK acknowledge the support of the Academy of Finland
ReSoLVE Centre of Excellence (grant number 307411).
This project has received funding from the European Research Council (ERC)
under the European Union's Horizon 2020 research and innovation
programme (Project UniSDyn, grant agreement n:o 818665).
\software{Pencil Code \citep{brandenburg2002,Pencil-JOSS}}

\bibliography{refs}{}
\bibliographystyle{aasjournal}

\appendix
\fag{
\begin{itemize}
\item
    Abstract – no more than 250 words
\item
    Main Text – no more than 3500 words (not including appendices or other supplementary material)
\item
    Figures and Tables – no more than 5 combined figures (each limited to 9 panels) and tables, e.g. 3 figures and 2 tables.
\item
    References – no more than 50 references
\end{itemize}}

\section{Notation}\label{sec:table}

\begin{table*}[h]
%\begin{deluxetable*}{ccl}
%\tablenum{A1}
%\tablecaption{Notation\label{tab:notation}}
%\tablewidth{0pt}
%\tablehead{
%\colhead{Notation Symbol} & \colhead{Denoting} & \colhead{Units/Definition} 
%\\
%}
%\decimalcolnumbers
%\startdata\\
\begin{tabular}{ccl}
\hline\hline\\
{Notation Symbol} & {Denoting} & {Units/Definition}\\\hline\\
 $\dfrac{D}{Dt}$ & material derivative & $\dfrac{\partial }{\partial t}+\vect{u}\cdot \vect\nabla$ \\
 %$\vect\nabla$ & gradient vector & e.g.,$\dfrac{\partial }{\partial x},\dfrac{\partial }{\partial y},\dfrac{\partial }{\partial z}$ \\
 %NS: brackets for vector, and space after comma
 $\vect\nabla$ & gradient vector & e.g., $\left(\dfrac{\partial }{\partial x},\dfrac{\partial }{\partial y},\dfrac{\partial }{\partial z}\right)$ \\
 $\rho$ & gas density & [g cm$^{-3}$]  \\
 $\vect u$ & gas velocity & [km s$^{-1}$] \\
 $t$ & time & [Myr] \\
 $s$ & specific entropy & [erg g$^{-1}$ K$^{-1}$] \\
 $T$ & gas temperature & [K] \\
 $\vect A$ & magnetic vector potential & [$\upmu$G cm] \\
 $\vect B$ & magnetic field & [$\upmu$G] \\
 $\vect j$ & current density & [Bi cm$^{-2}$] \\
 $\mathbfss W$ & traceless rate of strain tensor &
   ${\mathsf W}_{ij} = \dfrac{1}{2}\left(\dfrac{\partial u_i}{\partial x_j}
                  + \dfrac{\partial u_j}{\partial x_i}
                  -\dfrac{2}{3} \delta_{ij}\vect\nabla\cdot \vect u\right)$ \\
 $|\mathbfss W|^2$ & contracted rate of strain tensor &
   $|\mathbfss W|^2={\mathsf W}_{ij}{\mathsf W}_{ij}$\\
 $\mathbfss W^{(3)}$ & 6th order rate of strain tensor &
   ${\mathsf W}_{ij}^{(3)} = \dfrac{1}{2}\left(\dfrac{\partial^5 u_j}{\partial x_i^5}
                  + \dfrac{\partial^4}{\partial x_i^4}\left(\dfrac{\partial u_i}{\partial x_j}\right)
                  -\dfrac{1}{3}\dfrac{\partial^4}{\partial x_i^4}\left(\vect\nabla\cdot \vect u\right)\right)$ \\
 $\zeta_{D},\zeta_{\nu},\zeta_{\chi}$ & shock diffusion coefficient& $\propto \left(-\vect\nabla\cdot\vect u\right)_+$\\
 $\nu,\eta$ & viscosity, resistivity coefficient& [kpc km s$^{-1}$]\\
 $\nu_3,\chi_3,\eta_3$ & hyperdiffusion coefficient& [kpc$^{5}$ km s$^{-1}$]\\
 $\Gamma$ & UV-heating& [erg g$^{-1}$ s$^{-1}$]\\
 $\Lambda$ & radiative cooling& [erg cm$^{3}$ g$^{-2}$ s$^{-1}$]\\
 $\ESK+\EST$ & SN explosion energy& [10$^{51}$erg]\\
 %$\dot\sigma$ & SN explosion rate & [kpc Myr$^{-1}$]\\
 %NS: kpc => kpc^-3 for per unit volume, but pls check
 $\dot\sigma$ & SN explosion rate & [kpc$^{-3}$ Myr$^{-1}$]\\
 $c_{\rm s}$ & sound speed & [km s$^{-1}$]\\
 $c_{\rm p}$ & specific heat at constant pressure & [erg g$^{-1}$ K$^{-1}$]\\
 $e_B$ & magnetic energy density & [erg cm$^{-3}$]\\
 $\overline{e_K}$ & time-averaged kinetic energy density & [erg cm$^{-3}$]\\
%\\
\hline
\end{tabular}
%\enddata
%\tablecomments{
%Summary of notation used in the text.}
%\end{deluxetable*}
%\tablecomments{\fag{to be continued \ldots}
%This table ``hides'' the third column in the \latex\ when compiled.
%The Distance is also centered on the decimals.  Note that when using decimal
%alignment you need to include the {\tt\string\decimals} command before
%{\tt\string\startdata} and all of the values in that column have to have a
%space before the next ampersand.}
\end{table*}
%\end{deluxetable*}

%% This command is needed to show the entire author+affiliation list when
%% the collaboration and author truncation commands are used.  It has to
%% go at the end of the manuscript.
%\allauthors

%% Include this line if you are using the \added, \replaced, \deleted
%% commands to see a summary list of all changes at the end of the article.
%\listofchanges
\end{document}

