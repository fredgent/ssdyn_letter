%% using aastex version 6.3
\documentclass[preprint2]{aastex63}

%% The default is a single spaced, 10 point font, single spaced article.
%% There are 5 other style options available via an optional argument. They
%% can be invoked like this:
%%
%% \documentclass[arguments]{aastex63}
%% 
%% where the layout options are:
%%
%%  twocolumn   : two text columns, 10 point font, single spaced article.
%%                This is the most compact and represent the final published
%%                derived PDF copy of the accepted manuscript from the publisher
%%  manuscript  : one text column, 12 point font, double spaced article.
%%  preprint    : one text column, 12 point font, single spaced article.  
%%  preprint2   : two text columns, 12 point font, single spaced article.
%%  modern      : a stylish, single text column, 12 point font, article with
%% 		  wider left and right margins. This uses the Daniel
%% 		  Foreman-Mackey and David Hogg design.
%%  RNAAS       : Preferred style for Research Notes which are by design 
%%                lacking an abstract and brief. DO NOT use \begin{abstract}
%%                and \end{abstract} with this style.
%%
%% Note that you can submit to the AAS Journals in any of these 6 styles.
%%
%% There are other optional arguments one can invoke to allow other stylistic
%% actions. The available options are:
%%
%%   astrosymb    : Loads Astrosymb font and define \astrocommands. 
%%   tighten      : Makes baselineskip slightly smaller, only works with 
%%                  the twocolumn substyle.
%%   times        : uses times font instead of the default
%%   linenumbers  : turn on lineno package.
%%   trackchanges : required to see the revision mark up and print its output
%%   longauthor   : Do not use the more compressed footnote style (default) for 
%%                  the author/collaboration/affiliations. Instead print all
%%                  affiliation information after each name. Creates a much 
%%                  longer author list but may be desirable for short 
%%                  author papers.
%% twocolappendix : make 2 column appendix.
%%   anonymous    : Do not show the authors, affiliations and acknowledgments 
%%                  for dual anonymous review.
%%
%% these can be used in any combination, e.g.
%%
%% \documentclass[twocolumn,linenumbers,trackchanges]{aastex63}
%%
%% AASTeX v6.* now includes \hyperref support. While we have built in specific
%% defaults into the classfile you can manually override them with the
%% \hypersetup command. For example,
%%
%% \hypersetup{linkcolor=red,citecolor=green,filecolor=cyan,urlcolor=magenta}
%%
%% will change the color of the internal links to red, the links to the
%% bibliography to green, the file links to cyan, and the external links to
%% magenta. Additional information on \hyperref options can be found here:
%% https://www.tug.org/applications/hyperref/manual.html#x1-40003
%%
%% Note that in v6.3 "bookmarks" has been changed to "true" in hyperref
%% to improve the accessibility of the compiled pdf file.
%%
%% If you want to create your own macros, you can do so
%% using \newcommand. Your macros should appear before
%% the \begin{document} command.
%%
\usepackage{xfrac}
\usepackage{amsmath}

\newcommand{\vdag}{(v)^\dagger}
\newcommand\aastex{AAS\TeX}
\newcommand\latex{La\TeX}
\newcommand\Rm{{\rm Rm} }
\newcommand\kf{k_{\rm f} }
\newcommand\SNr{\sigma_{\rm SN}}
\newcommand{\vect}[1]{{{\mbox{\boldmath $#1$}}}}%also makes bold Greek letters
\newcommand{\mathbfss}[1]{\textbf{\textsf{#1}}}

\definecolor{midblue}{rgb}{0.0,0.4,0.7}
\definecolor{midgreen}{rgb}{0.0,0.8,0.3}
\definecolor{mypurple}{rgb}{0.8,0.2,0.8}
\newcommand{\fg}[1]{\textcolor{midblue}{#1}}
\newcommand{\fag}[1]{\textcolor{midgreen}{FAG: #1}}

%% Reintroduced the \received and \accepted commands from AASTeX v5.2
\received{June 1, 2019}
\revised{January 10, 2019}
\accepted{\today}
%% Command to document which AAS Journal the manuscript was submitted to.
%% Adds "Submitted to " the argument.
\submitjournal{AJ}

%% For manuscript that include authors in collaborations, AASTeX v6.3
%% builds on the \collaboration command to allow greater freedom to 
%% keep the traditional author+affiliation information but only show
%% subsets. The \collaboration command now must appear AFTER the group
%% of authors in the collaboration and it takes TWO arguments. The last
%% is still the collaboration identifier. The text given in this
%% argument is what will be shown in the manuscript. The first argument
%% is the number of author above the \collaboration command to show with
%% the collaboration text. If there are authors that are not part of any
%% collaboration the \nocollaboration command is used. This command takes
%% one argument which is also the number of authors above to show. A
%% dashed line is shown to indicate no collaboration. This example manuscript
%% shows how these commands work to display specific set of authors 
%% on the front page.
%%
%% For manuscript without any need to use \collaboration the 
%% \AuthorCollaborationLimit command from v6.2 can still be used to 
%% show a subset of authors.
%
%\AuthorCollaborationLimit=2
%
%% will only show Schwarz & Muench on the front page of the manuscript
%% (assuming the \collaboration and \nocollaboration commands are
%% commented out).
%%
%% Note that all of the author will be shown in the published article.
%% This feature is meant to be used prior to acceptance to make the
%% front end of a long author article more manageable. Please do not use
%% this functionality for manuscripts with less than 20 authors. Conversely,
%% please do use this when the number of authors exceeds 40.
%%
%% Use \allauthors at the manuscript end to show the full author list.
%% This command should only be used with \AuthorCollaborationLimit is used.

%% The following command can be used to set the latex table counters.  It
%% is needed in this document because it uses a mix of latex tabular and
%% AASTeX deluxetables.  In general it should not be needed.
%\setcounter{table}{1}

%%%%%%%%%%%%%%%%%%%%%%%%%%%%%%%%%%%%%%%%%%%%%%%%%%%%%%%%%%%%%%%%%%%%%%%%%%%%%%%%
%%
%% The following section outlines numerous optional output that
%% can be displayed in the front matter or as running meta-data.
%%
%% If you wish, you may supply running head information, although
%% this information may be modified by the editorial offices.
\shorttitle{Small-scale dynamo in the ISM}
\shortauthors{Gent et al.}
%%
%% You can add a light gray and diagonal water-mark to the first page 
%% with this command:
%% \watermark{text}
%% where "text", e.g. DRAFT, is the text to appear.  If the text is 
%% long you can control the water-mark size with:
%% \setwatermarkfontsize{dimension}
%% where dimension is any recognized LaTeX dimension, e.g. pt, in, etc.
%%
%%%%%%%%%%%%%%%%%%%%%%%%%%%%%%%%%%%%%%%%%%%%%%%%%%%%%%%%%%%%%%%%%%%%%%%%%%%%%%%%

%% This is the end of the preamble.  Indicate the beginning of the
%% manuscript itself with \begin{document}.

\begin{document}

\title{Small-scale dynamo in supernova driven ISM turbulence}

%%
%% The \author command is the same as before except it now takes an optional
%% argument which is the 16 digit ORCID. The syntax is:
%% \author[xxxx-xxxx-xxxx-xxxx]{Author Name}
%%
%%
%% Use \affiliation for affiliation information. The old \affil is now aliased
%% to \affiliation. AASTeX v6.3 will automatically index these in the header.
%% When a duplicate is found its index will be the same as its previous entry.
%%
%% Note that \altaffilmark and \altaffiltext have been removed and thus 
%% can not be used to document secondary affiliations. If they are used latex
%% will issue a specific error message and quit. Please use multiple 
%% \affiliation calls for to document more than one affiliation.
%%
%% The new \altaffiliation can be used to indicate some secondary information
%% such as fellowships. This command produces a non-numeric footnote that is
%% set away from the numeric \affiliation footnotes.  NOTE that if an
%% \altaffiliation command is used it must come BEFORE the \affiliation call,
%% right after the \author command, in order to place the footnotes in
%% the proper location.
%%
%% Use \email to set provide email addresses. Each \email will appear on its
%% own line so you can put multiple email address in one \email call. A new
%% \correspondingauthor command is available in V6.3 to identify the
%% corresponding author of the manuscript. It is the author's responsibility
%% to make sure this name is also in the author list.
%%
%% While authors can be grouped inside the same \author and \affiliation
%% commands it is better to have a single author for each. This allows for
%% one to exploit all the new benefits and should make book-keeping easier.
%%
%% If done correctly the peer review system will be able to
%% automatically put the author and affiliation information from the manuscript
%% and save the corresponding author the trouble of entering it by hand.

\correspondingauthor{Frederick Gent}
\email{frederick.gent@aalto.fi}
\email{mordecai@amnh.org}
\email{maarit.kapyla@aalto.fi}
\email{nishant@iucaa.in}

\author[0000-0002-1331-2260]{Frederick A. Gent}
\affiliation{
Astroinformatics, Department of Computer Science, Aalto University, PO Box 15400, FI-00076 Aalto, Finland
 }
\affiliation{
    School of Mathematics, Statistics and Physics,
      Newcastle University, NE1 7RU, UK 
 }

\author[0000-0003-0064-4060]{Mordecai-Mark {Mac Low}}
\affiliation{
American Museum of Natural History, 79th Street at Central Park West, New York, NY 10024, USA
}
\affiliation{
{Center for Computational Astrophysics, Flatiron Institute, New York,
NY 10010, USA} 
}

\author[0000-0002-9614-2200]{Maarit J. K\"apyl\"a}
\affiliation{
Astroinformatics, Department of Computer Science, Aalto University, PO Box 15400, FI-00076 Aalto, Finland
}
\affiliation{
Max Planck Institute for Solar System Research, Justus-von-Liebig-Weg 3, 37707 G\"ottingen, Germany
}
\affiliation{
    Nordic Institute for Theoretical Physics,
      Roslagstullsbacken 23, 106 91 Stockholm, Sweden 
}

\author{Nishant Singh}
\affiliation{
Inter-University Centre for Astronomy \& Astrophysics, Pune 411 007, India
}

%% Note that the \and command from previous versions of AASTeX is now
%% depreciated in this version as it is no longer necessary. AASTeX 
%% automatically takes care of all commas and "and"s between authors names.

%% AASTeX 6.3 has the new \collaboration and \nocollaboration commands to
%% provide the collaboration status of a group of authors. These commands 
%% can be used either before or after the list of corresponding authors. The
%% argument for \collaboration is the collaboration identifier. Authors are
%% encouraged to surround collaboration identifiers with ()s. The 
%% \nocollaboration command takes no argument and exists to indicate that
%% the nearby authors are not part of surrounding collaborations.

%% Mark off the abstract in the ``abstract'' environment. 
\begin{abstract}
We present a numerical resolution study over a range of magnetic Reynolds
number (Rm) to identify characteristics of the small-scale dynamo (SSD) under
supernova-driven turbulence of the interstellar medium (ISM).
We confirm that conditions in the ISM are likely highly conducive to 
SSD in the diffuse hot and warm turbulent ISM, given Rm in the real ISM is 
orders of magnitude higher than we model at our highest resolution and modest
SN explosion frequency ($\SNr$), 20\% that of the solar neighbourhood.
This is despite the negative impact on SSD expected from the highly compressible
nature of the flow.
We find convergence in the growth rate of the SSD with resolution for a given 
$\SNr$ approaching 
sub-parsec scale and a trend towards the exclusion of SSD for resolution
more coarse than 4 parsecs.
Across the modelled range of 0.5 to 4 parsec resolution we find the SSD 
saturates consistently at about 5\% of the energy equipartion level, independent
of the growth rate.

\end{abstract}

%% Keywords should appear after the \end{abstract} command. 
%% See the online documentation for the full list of available subject
%% keywords and the rules for their use.
\keywords{tba ---}

\section{Introduction} \label{sec:intro}


To be revised \ldots

This letter addresses the recent results of \citet{GE20}, which asserts the 
absence of a small scale (fluctuation) dynamo (SSD) in the interstellar medium
(ISM) subject to supernova induced turbulence and multi-phase structure
applicable to the solar neighbourhood. 
They refute the earlier dynamo solutions of \citet{BKMM04}

\section{Disentangling the dynamo} \label{sec:ssd-tang}

\begin{figure*}
\gridline{ \fig{figs/ssd-tang-brms.png}{0.5\textwidth}{(a)}
          }
\gridline{     \fig{figs/ssdBpower.png}{0.5\textwidth}{(b)}
          \fig{figs/tanglingBpower.png}{0.5\textwidth}{(c)}
          }
\caption{
%fag
% Comparison of simulations, one with tangling and the other with dynamo
%-amplification of magnetic field, from non-helical random forcing.
%-For the SSD model, $R_{\rm m}=148$ and $P_{\rm m}=50.0$, while the tangling
%-model has the dynamo suppressed by choosing $R_{\rm m}=7.40$ and $P_{\rm m}=2.50$.
%-Panel (a) displays $B_{\rm rms}$ as a proportion of $\overline{B_{\rm Eq}}$,
%
%% mm where $\overline{B_{\rm Eq}}=\overline{u_{\rm rms}}/\sqrt{2}$ is
%the time-averaged equipartion field. 
%% mm and $\overline{u_{\rm rms}}$ is the  time-averaged $u_{\rm rms}$ during the kinematic phase.
Simulation results for non-helical random forcing. 
The simulation with $R_{\rm m}=148$ and $P_{\rm m}=50.0$ supports dynamo 
amplification of magnetic field
With $R_{\rm m}=7.4$ and $P_{\rm m}=2.50$ dynamo is suppressed and 
amplification is limited to tangling of the magnetic field.
Panel (a) displays mean magnetic energy density, $e_B$, evolving as a
proportion of time-averaged kinetic energy density, $\overline{e_K}$.
The inset shows a zoom-in of the early linear growth of the tangled field.
%fag
%Time is scaled with $\kf \overline{u_{\rm rms}}$, the eddy turnover
%time at the forcing scale.
Time is normalised by the inverse eddy turnover time at the forcing scale,
$\kf \overline{u_{\rm rms}}$.
%fag
%Power spectra are displayed for magnetic energy of the SSD (b) and
%tangled model (c).  The legend shows the scaled
Compensated power spectra are displayed of magnetic energy for the model with
SSD (b) and with tangling (c).
%fag
%The legend shows the scaled 
The legend shows the normalised  
times for each spectrum.
%fag
%The forcing scale, $\kf=8$, is indicated by the vertical dotted line and the black dashed
The forcing scale, $\kf=8$, is indicated by the vertical dotted line.
$k_1=L/(2\pi)$, where $L$ is the largest length scale in the simulation domain,
with the largest wavenumber $k/k_1=128$.
\label{fig:tangling}}
\end{figure*}

In many simulations of supernova driven turbulence with realistic vertical 
stratification \citep[e.g.,][]{deAvillez:2005,PO07,Hill:2012a,HI14} the mechaniisms
for inducing large scale magnetic fields are absent, such as rotation and shear.
To examine the effect of strong ordered magnetic fields these models then rely
on imposition a background or initial magnetic field, typically uniform,
which is then perturbed by supernova explosions.
The amplification of the magnetic field in these cases may include a small
scale dynamo in case of sufficently high magnetic Reynolds numbers, but any
amplification of the large scale field is limited to linear growth through
tangling of the imposed field.
Any initial large scale field would diffuse leaving only the random field. 
If an imposed field is sufficiently close to equipartition its characteristics
would dominate any MHD results.

More commonly models of supernova driven dynamo in the ISM have ignored the 
large scale effects and examined the small scale dynamo or the effect of the
turbulent magnetic field on the properties of the ISM
\citep[e.g.,][]{BKMM04,BalKim05,MacLow:2005}.
A seed or imposed field for such models could be uniform or random, but if the
simulated field results from tangling rather than dynamo, then any conclusions
about the ISM depend on the veracity of the chosen field rather than the 
nature of the MHD turbulence.

To illustrate some differences between tangling and small scale dynamo we adopt
a toy model with non-helical random forcing with wavenumber $\kf=8$ applied to
isothermal uniform density in $256^3$ $2\pi$-periodic boxes.
A uniform field $\sqrt2\cdot10^{-11}B_{\rm Eq}$ is imposed, where $B_{\rm Eq}$
is the strength of the equipartion magnetic field.
The two simulations are distinguished only by use of $\eta=10^{-4}$, exciting a 
small scale dynamo models, and $\eta=2\cdot10^{-3}$ subcritical for dynamo.
Our numerical implementations use the {\sc Pencil Code}\footnote{
\url{https://github.com/pencil-code}}.

We plot some of the diagnostics from the results of these two simulations in 
Figure\,\ref{fig:tangling}.
Panel (a) illustrates that the small scale dynamo 
is characterised by exponential growth \fag{Nishant, please add some relevant citation to this expected behaviour}, growing over 10 orders of magnitude in 
just over 400 eddy turnover times.
Tangling results only in linear amplification
early on (see inset), saturating within 100 eddy turnover times at below 5 times
the imposed field energy density.

In panel (b) we plot for the SSD model some power spectra for the magnetic
energy, evolving over time alongside a kinetic energy spectrum at late stage.
The magnetic energy spectra are compensated by $k^{-3/2}$ and kinetic by
$k^{5/3}$.
A profile becoming horizontal, therefore, corresponds to the Kazentsev 
\fag{Anyone, a suitable reference for this expected behaviour?} $3/2$ inverse
cascade power law or Kolmogorov turbulent energy cascade.
In panel (c) see we show similar for the tangling model.
The forcing scale $\kf=8$ is evident in the kinetic spectra and highlighted by 
a vertical dotted line.
The SSD magnetic spectrum (b) evolves near self-similarly, with an 
uncompensated peak
wavenumber above 20 at early times reducing to below 20 upon saturation of the
dynamo.
The forcing scale has negligible effect on the magnetic spectra.
However, for the tangling case (c) the peak wavenumber for the magnetic energy
spectrum is strongly identified with the forcing scale.
The Kazentsev range of the spectrum extends to wavenumbers larger than the 
forcing scale for SSD, while confined to larger scales for the field with
tangling only. 

Given the high magnetic Prandtl numbers, the magnetic spectra retain more energy
at smaller scales than the kinetic spectra.
Although the kinetic parameters are identical and are subject to the same 
viscous cutoff, the SSD extends the kinetic energy into the larger scales
due to the feedback from the Lorentz force.
With SSD there is a short inertial range for $10\leq k/k_1\lesssim 15$.
Thus, in the dynamo kinetic energy deposited along the Kolmogorov range to
smaller scales transfers energy to the magnetic field at these scales
inducing an inverse Kazentsev range begining at scales below the forcing
scale, while with tangling the energy tranfers to the magnetic field
only at scales between the forcing scale and scale of the imposed field.
There is only dissipation of the field at scales smaller than the Kazentsev 
range.


\section{SN turbulence model design} \label{sec:model}

While in the SN experiments considered in this letter are restricted to
SSD, some large scale models do include the large scale dynamo
\citep[e.g.,][]{Gressel:2008,HWK09,WA09,Gent:2013b,EGSFB16,Pakmor17,GE20}.
Most of these do not include a small scale dynamo.
\citet{Gent:2013b,EGSFB16} do include SSD, but without a clear understanding of
the characteristics of this SSD it is difficult to explore the effect this has
on the large scale dynamo (LSD) and the properties of the ISM.
In this letter we explore the SSD in realistic simulations of SN driven 
turbulence.
The resolution and parameter study is intended to identify the critical ranges
for excitation of SSD and understand dynamo growth rates and saturation 
conditions.
This will also identify the prerequisites for including or excluding SSD in the
LSD models and subsequently establish the importance of SSD-LSD interactions 
on galaxy dynamics. 


To exclude any large scale magnetic field dynamics in these simulations, the
computational domain occupies a cube of length 120 pc, with periodic boundaries
on all sides. 
We solve the system of non-ideal compressible MHD equations, including 
mass Eq\,\eqref{eq:mass}, momentum Eq\,\eqref{eq:mom}, energy Eq\,\eqref{eq:ent} and
induction Eq\,\eqref{eq:ind}:
%----------------------------------------------------------------------------
  \begin{eqnarray}
  \label{eq:mass}
  \frac{D\rho}{Dt} &=& - \vect\nabla \cdot (\rho \vect{u}) + \vect\nabla \cdot\zeta_D\vect\nabla\rho,\\
%----------------------------------------------------------------------------
  \label{eq:mom}
  \frac{D\vect{u}}{Dt} &=&
    -c_{\rm s}^2\vect\nabla\left({s}/{c_{\rm p}}+\ln\rho\right)
    +\rho^{-1}\vect{j}\times\vect{B}\nonumber\\
    &+&\rho^{-1}\vect\nabla\cdot \left(2\rho\nu{\mathbfss W}\right)+
       \zeta_{\nu}\left(\vect\nabla\vect\nabla \cdot \vect{u} \right)\nonumber\\
    & 
     +&
       \rho^{-1}\vect\nabla\cdot \left(2\rho\nu_3{\mathbfss W}^{(3)}\right),\\
%----------------------------------------------------------------------------
  \label{eq:ent}
  \rho T\frac{D s}{Dt} &=& \dot{\SNr} +\rho\Gamma -\rho^2\Lambda +\eta\mu_0\vect{j}^2 
     +2 \rho \nu\left|{\mathbfss W}\right|^{2}
   \nonumber\\
  &+&\vect\nabla\cdot\left(\zeta_\chi\rho T\vect\nabla s\right)
  + 
  \rho T\chi_3\vect\nabla^6 s,\\
%----------------------------------------------------------------------------
  \label{eq:ind}
    \frac{\partial \vect{A}}{\partial t} &=&
        \vect{u}\times\vect{B}
      + \eta\vect\nabla^2\vect{A}
       +\eta_3\vect\nabla^6\vect{A},
  \end{eqnarray}
completing the system with the equation of state for an ideal gas.
Symbols have their usual meanings, but a table is included in the appendix for 
clarity.
The terms ... have meaning ... 

\begin{figure*}[ht!]
\gridline{\fig{figs/eB-res-4eta.png}{0.5\textwidth}{(a)}
          \fig{figs/eB-res-3eta.png}{0.5\textwidth}{(b)}
          }
\caption{
The volume averaged magnetic energy density for models with resolution
between 0.5\,pc and 4\,pc are plotted over time.
These are scaled by reference to their
 time-averaged statistical-steady kinetic energy density.
Resistivity, $\eta=10^{-4}$\,kpc\,km\,s$^{-1}$ in panel {\rm(a)} and $10^{-3}$
{\rm(b)}, is applied.
\label{fig:eb-res}}
\end{figure*}


\begin{figure*}
\gridline{\fig{figs/0_5pc-eB-nu4.png}{0.5\textwidth}{(a)}
            \fig{figs/1pc-eB-nu4.png}{0.5\textwidth}{(b)}
          }
\gridline{  \fig{figs/2pc-eB-nu4.png}{0.5\textwidth}{(c)}
            \fig{figs/4pc-eB-nu4.png}{0.5\textwidth}{(d)}
          }
\gridline{
            \fig{figs/2pc-eB-nu5.png}{0.5\textwidth}{(e)}
            \fig{figs/4pc-eB-nu6.png}{0.5\textwidth}{(f)}
          }
  \begin{picture}(0,0)(0,0)
    \put( 56,468){\begin{scriptsize}{\sf{$\delta x=0.5$pc}}\end{scriptsize}}
    \put(457,468){\begin{scriptsize}{\sf{$\delta x=1.0$pc}}\end{scriptsize}}
    \put( 56,277){\begin{scriptsize}{\sf{$\delta x=2.0$pc}}\end{scriptsize}}
    \put(457,277){\begin{scriptsize}{\sf{$\delta x=4.0$pc}}\end{scriptsize}}
    \put( 56, 85){\begin{scriptsize}{\sf{$\delta x=2.0$pc}}\end{scriptsize}}
    \put(457, 85){\begin{scriptsize}{\sf{$\delta x=4.0$pc}}\end{scriptsize}}
  \end{picture}
\caption{
The effect of resistivity $\eta$ is compared at each resolution, {\rm(a)} --
 {\rm(d)} for supernova rate $\sigma=0.2\sigma_{\rm SN}$ and {\rm(e)} -- {\rm(f)} for $\sigma=\sigma_{\rm SN}$ at lower resolution, 
where $\sigma_{\rm SN}\simeq 50$\,kpc$^{-3}$\,Myr$^{-1}$ is the solar neighbourhood equivalent random SN frequency.
The time axes vary between plots sufficient to reach saturation of the dynamo.
At 4\,pc resolution the models with $\eta=10^{-4}$\,kpc\,km\,s$^{-1}$ are 
continued until the dynamo has saturated, for comparison with the higher 
resolution saturation levels.
\label{fig:eb-nu}}
\end{figure*}

\begin{figure*}
\gridline{\fig{figs/0_5pcPm0e-4_0Bpower.png}{0.5\textwidth}{(a)}
          \fig{figs/0_5pcPm0e-4_0kpower.png}{0.5\textwidth}{(b)}
          }
\gridline{\fig{figs/0_5pcPm0e-3_0Bpower.png}{0.5\textwidth}{(c)}
          \fig{figs/0_5pcPm0e-3_0kpower.png}{0.5\textwidth}{(d)}
          }
\caption{
Compensated energy spectra at times in Myr given in the captions for 
0.5\,pc resolution.
Rm is super critical for dynamo applying $\eta=10^{-4}$ in
panels (a) and (b) and sub critical or marginal for dynamo applying
$\eta=10^{-3}$ in panels (c) and (d).
Energy spectra are compensated against theoretical profiles of Kazentsev
$k^{3/2}$, (a) and (c), and Kolmogorov $k^{-5/3}$, (b) and (d), 
each represented by the horizontal black dashed lines.
\label{fig:4power}}
\end{figure*}

\begin{figure*}
\gridline{\vspace{-0.7cm} \fig{figs/nu0_Bpower.png}{0.5\textwidth}{\vspace{-1.3cm}\hspace{-2cm}(a)}
          \vspace{-0.7cm} \fig{figs/nu0_kpower.png}{0.5\textwidth}{\vspace{-1.3cm}\hspace{-2cm}(b)}}                                                                             
\gridline{\vspace{-0.7cm}\fig{figs/nu10_Bpower.png}{0.5\textwidth}{\vspace{-1.3cm}\hspace{-2cm}(c)}
          \vspace{-0.7cm}\fig{figs/nu10_kpower.png}{0.5\textwidth}{\vspace{-1.3cm}\hspace{-2cm}(d)}}                                                                             
\gridline{\vspace{-0.7cm} \fig{figs/nu1_Bpower.png}{0.5\textwidth}{\vspace{-1.3cm}\hspace{-2cm}(e)}
          \vspace{-0.7cm} \fig{figs/nu1_kpower.png}{0.5\textwidth}{\vspace{-1.3cm}\hspace{-2cm}(f)}}                                                                             
\gridline{\vspace{-0.3cm}  \fig{figs/SN_Bpower.png}{0.5\textwidth}{\vspace{-1.3cm}\hspace{-2cm}(g)}
          \vspace{-0.3cm}  \fig{figs/SN_kpower.png}{0.5\textwidth}{\vspace{-1.3cm}\hspace{-2cm}(h)} }
\caption{
Compensated energy spectra at dynamo saturation (or late time) for each
resolution.
Resistivity $\eta=0,\,10^{-4}$ and $10^{-3}$ in panel pairs (a,b), (c,d) and
(e,f), respectively.
Energy spectra are compensated against theoretical profiles of Kazentsev
$k^{3/2}$, left panels, and Kolmogorov $k^{-5/3}$, right, 
each represented by the horizontal black dashed lines.
In panels (g,h) for 2\,pc resolution at 200\,Myr effect of supernova rate
$\sigma=0.2\sigma_{\rm SN}$ or $\sigma_{\rm SN}$ with
$\eta=10^{-4}$ or $10^{-3}$ as listed in the captions are displayed.
\label{fig:3power}}
\end{figure*}

\bibliography{refs}{}
\bibliographystyle{aasjournal}

%% This command is needed to show the entire author+affiliation list when
%% the collaboration and author truncation commands are used.  It has to
%% go at the end of the manuscript.
%\allauthors

%% Include this line if you are using the \added, \replaced, \deleted
%% commands to see a summary list of all changes at the end of the article.
%\listofchanges
\end{document}

