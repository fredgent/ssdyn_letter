%% using aastex version 6.3
\documentclass[preprint2]{aastex63}

%% The default is a single spaced, 10 point font, single spaced article.
%% There are 5 other style options available via an optional argument. They
%% can be invoked like this:
%%
%% \documentclass[arguments]{aastex63}
%% 
%% where the layout options are:
%%
%%  twocolumn   : two text columns, 10 point font, single spaced article.
%%                This is the most compact and represent the final published
%%                derived PDF copy of the accepted manuscript from the publisher
%%  manuscript  : one text column, 12 point font, double spaced article.
%%  preprint    : one text column, 12 point font, single spaced article.  
%%  preprint2   : two text columns, 12 point font, single spaced article.
%%  modern      : a stylish, single text column, 12 point font, article with
%% 		  wider left and right margins. This uses the Daniel
%% 		  Foreman-Mackey and David Hogg design.
%%  RNAAS       : Preferred style for Research Notes which are by design 
%%                lacking an abstract and brief. DO NOT use \begin{abstract}
%%                and \end{abstract} with this style.
%%
%% Note that you can submit to the AAS Journals in any of these 6 styles.
%%
%% There are other optional arguments one can invoke to allow other stylistic
%% actions. The available options are:
%%
%%   astrosymb    : Loads Astrosymb font and define \astrocommands. 
%%   tighten      : Makes baselineskip slightly smaller, only works with 
%%                  the twocolumn substyle.
%%   times        : uses times font instead of the default
%%   linenumbers  : turn on lineno package.
%%   trackchanges : required to see the revision mark up and print its output
%%   longauthor   : Do not use the more compressed footnote style (default) for 
%%                  the author/collaboration/affiliations. Instead print all
%%                  affiliation information after each name. Creates a much 
%%                  longer author list but may be desirable for short 
%%                  author papers.
%% twocolappendix : make 2 column appendix.
%%   anonymous    : Do not show the authors, affiliations and acknowledgments 
%%                  for dual anonymous review.
%%
%% these can be used in any combination, e.g.
%%
%% \documentclass[twocolumn,linenumbers,trackchanges]{aastex63}
%%
%% AASTeX v6.* now includes \hyperref support. While we have built in specific
%% defaults into the classfile you can manually override them with the
%% \hypersetup command. For example,
%%
%% \hypersetup{linkcolor=red,citecolor=green,filecolor=cyan,urlcolor=magenta}
%%
%% will change the color of the internal links to red, the links to the
%% bibliography to green, the file links to cyan, and the external links to
%% magenta. Additional information on \hyperref options can be found here:
%% https://www.tug.org/applications/hyperref/manual.html#x1-40003
%%
%% Note that in v6.3 "bookmarks" has been changed to "true" in hyperref
%% to improve the accessibility of the compiled pdf file.
%%
%% If you want to create your own macros, you can do so
%% using \newcommand. Your macros should appear before
%% the \begin{document} command.
%%
\usepackage{xfrac}
\usepackage{amsmath}
\usepackage{upgreek}

\newcommand{\vdag}{(v)^\dagger}
\newcommand\aastex{AAS\TeX}
\newcommand\latex{La\TeX}
\newcommand\Rm{{\rm Rm} }
\newcommand\Rey{{\rm Re} }
\newcommand\Pm{{\rm Pm} }
\newcommand\kf{k_{\rm f} }
\newcommand\SNr{\dot\sigma_{\rm sn}}
\newcommand\OSN{\Omega_{\rm sn}}
\newcommand\ESK{E_{\rm kin}}
\newcommand\EST{E_{\rm th}}
\newcommand\ESN{E_{\sigma}}
\newcommand\Ms{M_{\rm s}}
\newcommand{\vect}[1]{{{\mbox{\boldmath $#1$}}}}%also makes bold Greek letters
\newcommand{\mathbfss}[1]{\textbf{\textsf{#1}}}
\newcommand\kpc{~ {\rm kpc}}
\newcommand\pc{~ {\rm pc}}
\newcommand\dx{ {\delta x}}
\newcommand\Myr{~ {\rm Myr}}
\newcommand\erg{~ {\rm erg}}
\newcommand\kms{~ {\rm km~ s}^{-1}}

\definecolor{midblue}{rgb}{0.0,0.4,0.7}
\definecolor{midgreen}{rgb}{0.0,0.8,0.3}
\definecolor{mypurple}{rgb}{0.8,0.2,0.8}
\newcommand{\fg}[1]{\textcolor{midblue}{#1}}
\newcommand{\fag}[1]{\textcolor{midgreen}{FAG: #1}}
%\newcommand{\ns}[1]{\textcolor{orange}{NS: #1}}
\newcommand{\ns}[1]{\textcolor{orange}{#1}}

%% Reintroduced the \received and \accepted commands from AASTeX v5.2
\received{June 1, 2019}
\revised{January 10, 2019}
\accepted{\today}
%% Command to document which AAS Journal the manuscript was submitted to.
%% Adds "Submitted to " the argument.
\submitjournal{ApJL}

%% For manuscript that include authors in collaborations, AASTeX v6.3
%% builds on the \collaboration command to allow greater freedom to 
%% keep the traditional author+affiliation information but only show
%% subsets. The \collaboration command now must appear AFTER the group
%% of authors in the collaboration and it takes TWO arguments. The last
%% is still the collaboration identifier. The text given in this
%% argument is what will be shown in the manuscript. The first argument
%% is the number of author above the \collaboration command to show with
%% the collaboration text. If there are authors that are not part of any
%% collaboration the \nocollaboration command is used. This command takes
%% one argument which is also the number of authors above to show. A
%% dashed line is shown to indicate no collaboration. This example manuscript
%% shows how these commands work to display specific set of authors 
%% on the front page.
%%
%% For manuscript without any need to use \collaboration the 
%% \AuthorCollaborationLimit command from v6.2 can still be used to 
%% show a subset of authors.
%
%\AuthorCollaborationLimit=2
%
%% will only show Schwarz & Muench on the front page of the manuscript
%% (assuming the \collaboration and \nocollaboration commands are
%% commented out).
%%
%% Note that all of the author will be shown in the published article.
%% This feature is meant to be used prior to acceptance to make the
%% front end of a long author article more manageable. Please do not use
%% this functionality for manuscripts with less than 20 authors. Conversely,
%% please do use this when the number of authors exceeds 40.
%%
%% Use \allauthors at the manuscript end to show the full author list.
%% This command should only be used with \AuthorCollaborationLimit is used.

%% The following command can be used to set the latex table counters.  It
%% is needed in this document because it uses a mix of latex tabular and
%% AASTeX deluxetables.  In general it should not be needed.
%\setcounter{table}{1}

%%%%%%%%%%%%%%%%%%%%%%%%%%%%%%%%%%%%%%%%%%%%%%%%%%%%%%%%%%%%%%%%%%%%%%%%%%%%%%%%
%%
%% The following section outlines numerous optional output that
%% can be displayed in the front matter or as running meta-data.
%%
%% If you wish, you may supply running head information, although
%% this information may be modified by the editorial offices.
\shorttitle{Small-scale dynamo in the ISM}
\shortauthors{Gent et al.}
%%
%% You can add a light gray and diagonal water-mark to the first page 
%% with this command:
%% \watermark{text}
%% where "text", e.g. DRAFT, is the text to appear.  If the text is 
%% long you can control the water-mark size with:
%% \setwatermarkfontsize{dimension}
%% where dimension is any recognized LaTeX dimension, e.g. pt, in, etc.
%%
%%%%%%%%%%%%%%%%%%%%%%%%%%%%%%%%%%%%%%%%%%%%%%%%%%%%%%%%%%%%%%%%%%%%%%%%%%%%%%%%

%% This is the end of the preamble.  Indicate the beginning of the
%% manuscript itself with \begin{document}.

\begin{document}

\title{Small-Scale Dynamo in Supernova-Driven Interstellar Turbulence}

%%
%% The \author command is the same as before except it now takes an optional
%% argument which is the 16 digit ORCID. The syntax is:
%% \author[xxxx-xxxx-xxxx-xxxx]{Author Name}
%%
%%
%% Use \affiliation for affiliation information. The old \affil is now aliased
%% to \affiliation. AASTeX v6.3 will automatically index these in the header.
%% When a duplicate is found its index will be the same as its previous entry.
%%
%% Note that \altaffilmark and \altaffiltext have been removed and thus 
%% can not be used to document secondary affiliations. If they are used latex
%% will issue a specific error message and quit. Please use multiple 
%% \affiliation calls for to document more than one affiliation.
%%
%% The new \altaffiliation can be used to indicate some secondary information
%% such as fellowships. This command produces a non-numeric footnote that is
%% set away from the numeric \affiliation footnotes.  NOTE that if an
%% \altaffiliation command is used it must come BEFORE the \affiliation call,
%% right after the \author command, in order to place the footnotes in
%% the proper location.
%%
%% Use \email to set provide email addresses. Each \email will appear on its
%% own line so you can put multiple email address in one \email call. A new
%% \correspondingauthor command is available in V6.3 to identify the
%% corresponding author of the manuscript. It is the author's responsibility
%% to make sure this name is also in the author list.
%%
%% While authors can be grouped inside the same \author and \affiliation
%% commands it is better to have a single author for each. This allows for
%% one to exploit all the new benefits and should make book-keeping easier.
%%
%% If done correctly the peer review system will be able to
%% automatically put the author and affiliation information from the manuscript
%% and save the corresponding author the trouble of entering it by hand.

\correspondingauthor{Frederick Gent}
\email{Email: frederick.gent@aalto.fi, mordecai@amnh.org,\\ maarit.kapyla@aalto.fi, nishant@iucaa.in}

\author[0000-0002-1331-2260]{Frederick A. Gent}
\affiliation{
Astroinformatics, Department of Computer Science, Aalto University, PO Box 15400, FI-00076 Aalto, Finland
 }
\affiliation{
    School of Mathematics, Statistics and Physics,
      Newcastle University, NE1 7RU, UK 
 }

\author[0000-0003-0064-4060]{Mordecai-Mark {Mac Low}}
\affiliation{
  Department of Astrophysics, American Museum of Natural History,
  New York, NY 10024, USA
}
\affiliation{
{Center for Computational Astrophysics, Flatiron Institute, New York,
NY 10010, USA} 
}

\author[0000-0002-9614-2200]{Maarit J. K\"apyl\"a}
\affiliation{
Astroinformatics, Department of Computer Science, Aalto University, PO Box 15400, FI-00076 Aalto, Finland
}
\affiliation{
Max Planck Institute for Solar System Research, Justus-von-Liebig-Weg 3, 37707 G\"ottingen, Germany
}
\affiliation{
    Nordic Institute for Theoretical Physics,
      Roslagstullsbacken 23, 106 91 Stockholm, Sweden 
}

\author[0000-0001-6097-688X]{Nishant K. Singh}
\affiliation{
Inter-University Centre for Astronomy \& Astrophysics, Post Bag 4, Ganeshkhind, Pune 411 007, India
}
\affiliation{
Max Planck Institute for Solar System Research, Justus-von-Liebig-Weg 3, 37707 G\"ottingen, Germany
}

%% AASTeX 6.3 has the new \collaboration and \nocollaboration commands to
%% provide the collaboration status of a group of authors. These commands 
%% can be used either before or after the list of corresponding authors. The
%% argument for \collaboration is the collaboration identifier. Authors are
%% encouraged to surround collaboration identifiers with ()s. The 
%% \nocollaboration command takes no argument and exists to indicate that
%% the nearby authors are not part of surrounding collaborations.

%% Mark off the abstract in the ``abstract'' environment. 
\begin{abstract}
%FAG: streamlining
%Magnetic fields appear to grow quickly even at early cosmological
%times, suggesting the action of a small-scale dynamo (SSD) in the
%interstellar medium (ISM) of galaxies. Previous studies have focused
%on idealized turbulent driving of the SSD. We here simulate more
%realistic supernova-driven turbulence to determine whether it can
%drive an SSD.  We vary the physical resistivity (and thus the magnetic
%Reynolds number), as well as the numerical resolution and supernova
%rate to delineate the regime in which an SSD occurs.
Magnetic fields grow quickly even at early cosmological
times, suggesting the action of a small-scale dynamo (SSD) in the
interstellar medium (ISM) of galaxies. Many studies have focused
on idealized turbulent driving of the SSD. We here simulate more
realistic supernova-driven turbulence to determine whether it can
drive an SSD.  We vary the physical resistivity (and implicitly magnetic
Reynolds number), as well as the numerical resolution and supernova
rate to delineate the regime in which an SSD occurs.
%We find convergence in the growth rate of the SSD with resolution approaching 
%sub-parsec scale for a given supernova explosion frequency
%$\dot\sigma$. Across the modelled range of 0.5 to 4 parsec resolution we find that with
We find convergence for SSD growth rate with resolution approaching 
sub-parsec scale for a given supernova distribution
$\dot\sigma$.
%FAG: reorder keep talking growth
Despite higher Mach numbers and negative impact of compressibility expected on
SSD, growth rates increase for $\dot\sigma=\SNr$ versus $0.2\SNr$, with $\SNr$
the solar neighbourhood rate.
%
%FAG: streamlining
% Across the modelled range of 0.5 to 4 parsec resolution we find that with
%sufficiently low resistivity the SSD saturates consistently at about 5\% of
%the energy equipartion level, independent of the growth rate and low Prandtl
%number of our experiments.
 Across the modelled range of 0.5 to 4 parsec resolution we find that for
sufficiently low resistivity the SSD saturates consistently at about 5\% of
energy equipartion, independent of growth rate and low Prandtl numbers in our
experiments.
%Dynamo
%growth rates for $\dot\sigma=\SNr$, the solar neighbourhood rate, relative to models with
%$\dot\sigma=0.2\SNr$, are greater despite higher Mach numbers.
%This is despite the negative impact on SSD expected from the highly compressible
%nature of the flow and low magnetic Prandtl number of our experiments.
%FAG: streamlining
%As the grid becomes coarser, the minimum physical resistivity that can
%be resolved increases as well. The trend suggests that an SSD cannot
%be excited for grid spacing much exceeding 4~pc, as the numerical
%resistivity suppresses it.
As the grid becomes coarser, the minimum resolved physical resistivity
increases. The trend suggests that numerical resistivity suppresses SSD for
grid spacing much exceeding 4~pc.
\end{abstract}
\keywords{dynamo --- magnetohydrodynamics (MHD) --- ISM: supernova remnants --- ISM: magnetic fields --- turbulence}


\section{Introduction}\label{sec:intro}
%==============================================================================
%\fag{Perhaps here, first a paragraph addressing the importance of MF in the ISM
%at all}

%MJK The landscape to me seems to be the following:
%MJK GE20 do not explicitely state that they do not have SSD, but do go forward with QKTFM, which assumes that no magnetic background turbulence is present, hence they make an implicit claim of having no SSD.
%MJK Balsara04 clearly had an SSD, as LSD was not allowed for, but had an imposed
%MJK field, and the role of that is unclear.
%MJK Fred+gang claim LSD and SSD, but it is unclear whether this is true, as
%MJK separating SSD and LSD is difficult.
%MJK Steinwandel et al. have recently claimed that an SSD is present in a
%MJK galactic-scale simulation, but no LSD, and that SSD "dies" off at later
%MJK stages, when Kazantsev scaling is not observed.
%MJK So, our study now tries to address at which resolution a fully healthy
%MJK SSD can be expected in a full ISM simulation, and helps to interpret
%MJK the mess descripbed above.
%MJK Now we DO not have imposed field, and can confirm the conclusions of Balsara04
%MJK without it.
%MJK It would be interesting if a critical Rm could be nailed down, but I doubt we
%MJK can, and must resort to what is listed above.
%NS: also commented below; could 10^-3 case be used for an estimate of Rm_crit (eq 5 based)?
%FAG 28.9 reword

 This letter addresses the necessary conditions and properties of a small-scale
 dynamo (SSD) in the interstellar medium (ISM).
 SSD acts at small eddy scales of the turbulence, thus driving magnetic field
 growth at correspondingly short turnover times.
 The fastest growing SSD modes are far smaller than the large-scale dynamo
 (LSD) modes generating the magnetic field structures organised at the systemic
 scale of the galactic disk.
 Hence, simulations capable of capturing LSD alongside the faster growing modes
 of SSD are computationally challenging.
 However, the interaction of SSD modes with the LSD likely fundamentally alters
 the evolution and structure of the magnetic field.

 Many simulations of supernova- (SN)-driven turbulence with realistic vertical
 stratification \citep[e.g.,][]{deAvillez:2005,PO07,Hill:2012a,HI14} have no
 mechanisms for inducing LSD, such as rotation and shear.
 The effect of strong ordered magnetic fields is modelled by initial
 imposition of a background, typically uniform, magnetic field.
 If an imposed field is sufficiently close to equipartition its characteristics
 would dominate any magnetized results.
 Some large-scale models do seek to include LSD \citep[e.g.,][]{Korpi:1999b,
 Gressel:2008,HWK09,WA09,Pakmor17,SBADMN19,GE20}, but most show no SSD.
 \citet{SBADMN19} do find an SSD, but no LSD.
 %\citet{Gent:2013b,EGSFB16} do appear to include an SSD, but without clear
 %%FAG/MJK merge to be reviewed
 %understanding of the SSD properties, how can this be confirmed and its effect
 %on the LSD determined?
 %FAG clean up
 \citet{Gent:2013b,EGSFB16} do appear to include an SSD.
 To confirm this and determine its effect of LSD, we must understand the
 properties of the SSD.
 %
     
 Any magnetic noise produced by tangling will also grow exponentially due to
 an LSD if present
 This noise plays an important role in quenching the LSD due to the magnetic
 $\alpha$-effect, causing saturation of large-scale fields.
 We need to discriminate this effect from an SSD.   

% Given uncertainty over the presence of SSD in the large-scale simulations,
% we seek to demonstrate that an SSD indeed occurs in SN-driven turbulence
% in isolation from any drivers of LSD.
% %MJK Perhaps we could cut from HERE .... END CUT
% %MJK see my comments below.
% The extensive resolution and parameter study is intended to identify critical
% %MJK Do we identify some critical conditions now?
% %MJK Except for high enough resolution and low enough resistivity, which, in the end of the day, are the same things. Usually this means high enough Rm, but in our case this cannot be confirmed. 
% %MJK One point that I so far forgot from a possible item to add to conclusions is about the saturation. We always see that the Eb/Ek is the same, independent of the growth rate? That is worth highlighting!
% %MJK btw, there is repetition of this text below, with more comments from Mordecai. Maybe we could just remove this from here?
% conditions for excitation, growth rates and saturation of SSD.
% %MJK END CUT
%% FAG Agree. This all stated in sufficient form elsewhere

 Previous experiments \citep[e.g.,][]{BKMM04,BalKim05,MacLow:2005}

 examined the SN-driven SSD using ideal magnetohydrodynamics (MHD) with
 a limited set of resolution and parameters that did not allow
 demonstration of convergence of the solutions nor dependence on
 magnetic Reynolds (Rm) or Prandtl (Pm) numbers.
 They included a weak imposed uniform field; we shall show
 that the amplification of their field is a result
 of SSD action and not just tangling of the field.
 We aim to find objective criteria with which to determine the presence of SSD
 in simulations \citep[such as][]{Gent:2013b,GE20}.

In this letter we compare the SSD to tangling in an idealized simulation
(Sect.\ 2), and then determine its presence in simulations of SN-driven 
turbulence in isolation from any drivers of an LSD (Sect.\ 3). 
Our numerical implementations use the {\sc Pencil Code}\footnote{
\href{https://github.com/pencil-code}{https://github.com/pencil-code}}.
A broad resolution and parameter study allows us to identify the critical
resistivity and resolution for excitation of an SSD, and follow the growth to
saturation (Sect.\ 4).
This provides objective criteria with which to determine the presence of SSD
in simulations \citep[such as][]{Gent:2013b,GE20,SBADMN19}.
Finally, we conclude in Sect.\ 5.
%FAG removed comments in intro, agreed MM's revisions
%==============================================================================
\begin{figure*}
\gridline{ \fig{figs/ssd-tang-brms.png}{0.45\textwidth}{(a)}
          }
\gridline{     \fig{figs/ssdBpower.png}{0.45\textwidth}{(b)}
          \fig{figs/tanglingBpower.png}{0.45\textwidth}{(c)}
          }
\caption{
%FAG: word limit
%Panel (a) displays mean magnetic energy density, $e_B$,
%evolving due to non-helical random forcing, scaled to time-averaged kinetic
%energy density, $\overline{e_K}$.
%The inset shows a zoom-in of the early linear growth of the tangled field.
%Time is normalised by the eddy turnover time at the forcing scale,
%$1/\kf \overline{u_{\rm rms}}$.
 (a) mean magnetic energy density, $e_B$, with non-helical random forcing,
 scaled to time-averaged kinetic energy density, $\overline{e_K}$.
 Inset: early zoom-in of linear growth of tangled field.
 Time is normalised by eddy turnover time, $1/\kf \overline{u_{\rm rms}}$.
%NS: do we need overline over u_rms, as Rm definition (unnumbered) doesn't have it either
%FAG. I think so - as it is a fluctuating quantity over time. I have changed it subsequently for consistency. In it's last use we refer to varying in space and time and use it in that context without the overbar.
% When I redo the plots in eps I'll change it and without overbar for final submission. 
 (b) SSD and (c) tangling compensated power spectra, times in the legends.
%Sample compensated power spectra at times indicated in the legends are
%displayed for the model with SSD (b) and with tangling (c). 
%NS: kf => kf/k1; is it better to mention other parameters Rm etc in the caption?
%FAG: had done, but streamlined to simplify captions so now explain Rm in text
%The forcing scale, $\kf/k_1=8$, is indicated by the vertical dotted line.
  Forcing scale, $\kf/k_1=8$: vertical dotted line.
\label{fig:tangling}}
\end{figure*}

%==============================================================================
\section{Disentangling the dynamo} \label{sec:ssd-tang}
%==============================================================================

%To illustrate some differences between tangling and SSD we adopt a simplified
%model with non-helical random forcing with wavenumber $\kf/k_1=8$ applied to
%isothermal uniform density in $256^3$, $2\pi$-periodic boxes with $k_1=1$ being
%the lowest wavenumber in the domain.
%A uniform field with energy density $e_B\simeq6\cdot10^{-22}\overline{e_K}$ is
%imposed, where $\overline{e_K}$ is the time-averaged kinetic energy density.
%The two simulations are distinguished only by use of dimensionless
%$\eta=10^{-4}$, exciting an SSD, and $\eta=2\cdot10^{-3}$
%subcritical for dynamo.
%These yield Rm=148.0 with Pm=50.0 for the SSD and Rm=7.4 with Pm=2.5 where
%the dynamo is suppressed and amplification is limited to tangling of the
%imposed magnetic field.
%FAG: streamlining
 To illustrate differences between tangling and SSD we adopt a simplified
 model.
 Non-helical random forcing is applied at wavenumber $\kf/k_1=8$ to
 $256^3$ zone, $2\pi$-periodic, isothermal boxes.
 The lowest wavenumber in the domain is $k_1=1$.
 The imposed uniform field has $e_B\simeq6\cdot10^{-22}~\overline{e_K}$, where
 $\overline{e_K}$ is time-averaged kinetic energy density.
 Two simulations are distinguished by use of dimensionless
 resistivity $\eta=10^{-4}$
%FAG: added nu
 and $\eta=2\cdot10^{-3}$, and viscosity $\nu=5\cdot10^{-3}$.
% {\bf [WHAT IS THE VISCOSITY?  WAS IT EXPLICIT, as suggested by the Pm given?]}
 These yield $\Rm=150$, with $\Pm = 50$, exciting SSD, and $\Rm=7.4$ with
 $\Pm=2.5$, inhibiting the dynamo so that amplification is limited to tangling
 of the imposed field.

%We plot some of the diagnostics from the results of these two simulations in 
%Figure\,\ref{fig:tangling}.
%Panel (a) illustrates that the SSD is characterised by
%exponential growth just over 400 eddy turnover times; see \cite{ZRS83} for the
%properties and excitation conditions of SSD.
%Tangling results only in linear amplification early on (see inset), saturating
%within 100 eddy turnover times at below 5 times the imposed field energy
%density.
%FAG: streamlining
 In Figure\,\ref{fig:tangling}(a) the SSD grows exponentially in just over 400
 eddy turnover times; see \cite{ZRS83} for SSD properties and excitation
 conditions.
 Tangling induces only linear growth (see inset), saturating just above
 the imposed field energy within 50 turnover times.

%In panel (b) we plot for the SSD model some power spectra for the magnetic
%energy, evolving over time alongside a kinetic energy spectrum at late stage.
%$k$ is normalised by $k_1=L/(2\pi)$, where $L$ is the largest length scale in
%the simulation domain, with the largest wavenumber $k/k_1=128$.
%The magnetic energy spectra are compensated by $k^{-3/2}$ and kinetic by
%$k^{5/3}$.
%A profile becoming horizontal, therefore, corresponds to the Kazantsev's
%$3/2$ scaling \citep{Sch02,BS14} or the Kolomogorov turbulent energy spectrum.
%In panel (c) see we show similar for the tangling model.
%The forcing scale $\kf=8$ is evident in the kinetic spectra and highlighted by 
%a vertical dotted line.
%The SSD magnetic spectrum (b) evolves near self-similarly, with an uncompensated
%peak wavenumber above 20 at early times reducing to below 20 upon saturation of
%the dynamo.
%The forcing scale has negligible effect on the magnetic spectra.
%However, for the tangling case (c) the peak wavenumber for the magnetic energy
%spectrum is strongly identified with the forcing scale.
%The Kazantsev range of the spectrum extends to wavenumbers larger than the 
%forcing scale for SSD, while confined to larger scales for the field with
%tangling only.
%FAG: streamlining
 Power spectra for the magnetic energy for the SSD as it evolves are plotted
 in Figure\,\ref{fig:tangling}(b) alongside a kinetic energy spectrum late on
 and similar for tangling in (c).
 $k$ is normalised by $k_1=L/(2\pi)$, where $L$ is the simulations's largest
 length scale and the largest wavenumber $k/k_1=128$.
 Magnetic energy spectra multiply $k^{-3/2}$ and kinetic $k^{5/3}$.
 The horizontal, therefore, corresponds to Kazantsev's $3/2$ scaling
 \citep{Sch02,BS14} or the Kolomogorov turbulent energy spectrum.
 The vertical dotted line indicates $\kf/k_1$.
 The SSD magnetic spectrum (b) evolves near self-similarly, the wavenumber for
 its peak above 20 early on, reducing to below 20 at saturation.
 $\kf$ is strongly identified with the peak in kientic energy and for 
 magnetic energy with tangling, but has negligible effect on magnetic
 spectra for SSD.
 The Kazantsev range of extends to wavenumbers larger than $\kf$ for SSD, while
 confined to scales larger than $\kf$ with tangling.
%FAG: moved from next para
 Thus, in dynamo kinetic energy along the Kolmogorov range transfers energy to
 the magnetic field at these scales inducing an inverse Kazantsev range
 at scales below $\kf$, while tangling transfers energy only at scales between
 $\kf$ and scale of the imposed field.
 At scales smaller than the Kazantsev range there is only dissipation. 
 

%Given the high magnetic Prandtl numbers, the magnetic spectra retain more energy
%at smaller scales than the kinetic spectra.
%Although the kinetic parameters are identical and are subject to the same 
%viscous cutoff, the SSD extends the kinetic energy into the 
%smaller scales
%due to the feedback from the Lorentz force.
%%With SSD there is a short inertial range for $10\leq k/k_1\lesssim 15$.
%%MJK Is there not also for the tangling case?
%%FAG: I say not add clarification
%With SSD there is a short inertial range in the kinetic spectrun 
%for $10\leq k/k_1\lesssim 15$, whereas in the tangling model the spectrum is
%decaying already at $k/k_1\gtrsim10$.
%%MJK Why in Fig 1 the y-scales are so much different when eb/ek is close to one?
%%FAG the spectra are integral quantities and all the kinetic energy is concentrated around k=8, which biases the peak value for eK and also there are the k-power prefactors
%Thus, in the dynamo kinetic energy deposited along the Kolmogorov range to
%smaller scales transfers energy to the magnetic field at these scales
%inducing an inverse Kazantsev range beginning at scales below the forcing
%scale, while with tangling the energy transfers to the magnetic field
%only at scales between the forcing scale and scale of the imposed field.
%There is only dissipation of the field at scales smaller than the Kazantsev 
%range.

\section{SN turbulence model design} \label{sec:model}

%To exclude any large-scale magnetic field dynamics in these simulations, the
%computational domain occupies a cube of length 256 pc, with periodic boundaries
%on all sides.
%In particular rotation, shear and stratification of the simulation domain are
%not included.
%There is no imposed field, in contrast to \citet{BKMM04}, so we can exclude
%tangling as a source of amplification of the initial nG random seed field.
%Grid size $\delta x=0.5$, 1, 2 and 4$\pc$  along each side are considered.
%We solve the system of non-ideal compressible MHD equations, including 
%mass Eq\,\eqref{eq:mass}, momentum Eq\,\eqref{eq:mom}, energy Eq\,\eqref{eq:ent} and
%induction Eq.\,\eqref{eq:ind}:
 We exclude large-scale magnetic field dynamics in simulations occupying 
 a cube of length 256 pc, with periodic boundaries without 
 rotation, shear nor stratification.
 A random nG seed field excludes tangling of an imposed field as the 
 source of any magnetic amplification.
 Grid size $\delta x=0.5$, 1, 2 and 4$\pc$  along each side are considered.
 We solve the system of non-ideal compressible MHD equations, including mass
 Eq\,\eqref{eq:mass}, momentum Eq\,\eqref{eq:mom}, energy Eq\,\eqref{eq:ent}
 and induction Eq.\,\eqref{eq:ind}:
%-------------------------------------------------------------------------------
  \begin{eqnarray}
  \label{eq:mass}
    \frac{D\rho}{Dt} &=& 
    -\rho \vect\nabla \cdot \vect{u}
    +\vect\nabla \cdot\zeta_D\vect\nabla\rho,
  \end{eqnarray}
%-------------------------------------------------------------------------------
  \begin{eqnarray}
  \label{eq:mom}
    \rho\frac{D\vect{u}}{Dt} &=& 
    \vect\nabla{\ESK\sigma}
    -\rho c_{\rm s}^2\vect\nabla\left({s}/{c_{\rm p}}+\ln\rho\right)
    +\vect{j}\times\vect{B}
    \nonumber\\
    &+&\vect\nabla\cdot \left(2\rho\nu{\mathbfss W}\right)
    +\rho\,\vect\nabla\left(\zeta_{\nu}\vect\nabla \cdot \vect{u} \right)
    \nonumber\\
    &+&\vect\nabla\cdot \left(2\rho\nu_3{\mathbfss W}^{(3)}\right),
  \end{eqnarray}
%-------------------------------------------------------------------------------
  \begin{eqnarray}
  \label{eq:ent}
    \rho T\frac{D s}{Dt} &=&
     \EST\dot\sigma +\rho\Gamma
    -\rho^2\Lambda +\eta\mu_0\vect{j}^2 
    \nonumber\\
    &+&2 \rho \nu\left|{\mathbfss W}\right|^{2}
    +\rho\,\zeta_{\nu}\left(\vect\nabla \cdot \vect{u} \right)^2
    \nonumber\\
    &+&\vect\nabla\cdot\left(\zeta_\chi\rho T\vect\nabla s\right)
    +\rho T\chi_3\vect\nabla^6 s,
  \end{eqnarray}
%-------------------------------------------------------------------------------
  \begin{eqnarray}
  \label{eq:ind}
    \frac{\partial \vect{A}}{\partial t} &=&
    \vect{u}\times\vect{B}
    +\eta\vect\nabla^2\vect{A}
    +\eta_3\vect\nabla^6\vect{A},
  \end{eqnarray}
%-------------------------------------------------------------------------------
%-------------------------------------------------------------------------------
 with the ideal gas equation of state closing the system.
%FAG: omitting appendix
%Many symbols have their usual meaning, and a table (\ref{sec:table}) is
%included in the Appendix for clarity.
%In particular, terms containing $\zeta_D,\,\zeta_\nu$ and $\zeta_\eta$ are the
%application of artificial diffusion acting proportional to shocks in 
%Eqs.\,\eqref{eq:mass},\,\eqref{eq:mom} and \eqref{eq:ent}, respectively.
%This machinery is described in detail in \citet{GMKSH20}, which is included to
%resolve shock discontinuities.
%In contrast to \citet{Gent:2013b} shock diffusion is not applied to
%Eq.\,\eqref{eq:ind}, where it would result in unphysical dissipation of the
%magnetic field in the shock fronts, when in fact in the shocks it is highly
%enhanced by compression.
%In Eqs.\,\eqref{eq:mom} -- \eqref{eq:ind}, expressions containing 
%$\nu_3,\,\chi_3$ and $\eta_3$ apply sixth-order hyperdiffusion applying
%primarily at the grid scale to resolve small-scale instabilities
%\citep[see, e.g.,][]{ABGS02,HB04}.
%FAG: word count
 Terms containing $\zeta_D,\,\zeta_\nu$ and $\zeta_\eta$ apply artificial
 diffusion proportional to shocks in Eqs.\,\eqref{eq:mass},\,\eqref{eq:mom} and
 \eqref{eq:ent}, respectively,
 machinery included to resolve shock discontinuities 
 \citep[][for details]{GMKSH20}.
 Unlike \citet{Gent:2013b} shock diffusion is not applied to
 Eq.\,\eqref{eq:ind}, to avoid excessive magnetic dissipation in shocks.
 It is actually enhanced by compression.
 In Eqs.\,\eqref{eq:mom} -- \eqref{eq:ind}, expressions containing 
 $\nu_3,\,\chi_3$ and $\eta_3$ apply sixth-order hyperdiffusion
 to resolve small-scale instabilities at the grid scale
\citep[see, e.g.,][]{ABGS02,HB04}.

%-------------------------------------------------------------------------------
\begin{figure*}
\gridline{\fig{figs/eB-res-4eta.png}{0.45\textwidth}{(a)}
          \fig{figs/eB-res-3eta.png}{0.45\textwidth}{(b)}
          }
\caption{
%The volume averaged magnetic energy density for models with $\dx$ between
%$0.5\pc$ and $4\pc$ are plotted over time.
%These are scaled by reference to their time-averaged statistical-steady kinetic
%energy density.
%Resistivity, $\eta=10^{-4}\kpc\kms$ in panel {\rm(a)} and $10^{-3}$ {\rm(b)}, is
%applied.
%FAG: word count
 Magnetic energy density with $\dx$ between $0.5\pc$ and $4\pc$,
 scaled to time-averaged kinetic energy density.
 (a) $\eta=10^{-4}\kpc\kms$ and (b) $10^{-3}$.
\label{fig:eb-res}}
\end{figure*}
%-------------------------------------------------------------------------------

%The SNe are injected randomly uniform in 3D space and as a Poisson
%process proportional to the solar neigbourhood rate in the Milky Way,
% $\SNr\simeq 50\kpc^{-3}\Myr^{-1}$.
%Each explosion injects $10^{51}\erg$ as thermal energy, $\EST$, but in dense
%media a small proportion may be in kinetic energy, $\ESK$.
%This is described in detail by \citet{GMKSH20}.
%To reduce statistical noise, at all $\dx$ and $\eta$ the SN are
%scheduled at the same time and coordinates for $\dot\sigma=0.2\SNr$ and for 
%correspondingly for $\dot\sigma=\SNr$.
%Non-adiabatic heating, $\Gamma$, and cooling, $\Lambda$, processes are included
%following \citet{Wolfire:1995} and \citet{Sarazin:1987}, as described in 
%\citet{Gent:2013a}.
%FAG word count
 SNe explosions are random uniform in space at a Poisson rate, $\dot\sigma$,
 proportional 
 to $\SNr\simeq 50\kpc^{-3}\Myr^{-1}$, that of the solar neigbourhood.
 Explosions inject $10^{51}\erg$ thermal energy, $\EST$, except for dense ISM
 a proportion is kinetic, $\ESK$ 
 \citep[see][]{GMKSH20}.
 For comparability, at all $\dx$ and $\eta$ SN for a given $\dot\sigma$
 follow a common schedule of timing and location.
 Non-adiabatic heating, $\Gamma$, and cooling, $\Lambda$, processes are
 included \citep{Gent:2013a} following \citet{Wolfire:1995} and
 \citet{Sarazin:1987}.

%In the subset of experiments presented in this letter we set viscosity, $\nu=0$,
%with $\nu_3$ applying optimally for each $\dx$ to ensure the flow is well
%resolved.
%We establish a benchmark for the magnetic field evolution setting $\eta=0$, with
%numerical resistivity acting via the optimal setting for $\eta_3$.
%We then vary $\eta\geq10^{-6}\kpc\kms$ to identify minimal range for which the
%physical resistivity is dynamically dominating the numerical resistivity, and
%examine the dependence of the dynamo on $\eta$ and grid resolution.
%Henceforth, where $\eta$ is specified for the SN simulations it has units of
%$\kpc\kms$. 
%In contrast to our earlier experiments \citep{Gent:2013a,Gent:2013b,GMKSH20},
%we do not include thermal diffusivity, $\chi$, in these experiments as the
%artificial diffusivities are adequate to ensure numerical stability and the
%physical effects of thermal conductivity can be expected to be relevant only
%at time and spatial scales much shorter than considered here.
%\fag{Mordecai, could you suggest some appropriate reference?}
%FAG word count
 The subset of experiments presented in this letter have viscosity, $\nu=0$,
 and $\nu_3$ set optimally for each $\dx$ to ensure the flow is well resolved.
 We benchmark the magnetic field evolution setting $\eta=0$, using only
 numerical resistivity via $\eta_3$.
 We vary $\eta\geq10^{-6}\kpc\kms$ to identify for which range the physical
 resistivity is dynamically effective.
 Henceforth, where $\eta$ is specified for the SN simulations it has units of
 $\kpc\kms$.
 Unlike our earlier experiments \citep{Gent:2013a,Gent:2013b,GMKSH20},
 we do not include thermal diffusivity, $\chi$, as the artificial diffusivities
 are adequate to ensure numerical stability and physical effects of thermal
 conductivity can be expected to be relevant only at much shorter time and
 spatial scales.
\fag{Mordecai, could you suggest some appropriate reference?}

%An important 
%parameter for the excitation of the SSD
%is the magnetic Reynolds number, Rm,
%and the magnetic Prandtl number, $\Pm = \Rm/\Rey = \nu/\eta$, where $\Rey$
%is the fluid Reynolds number.
%%FAG: added overline for consistency and added eq number as NS suggested
%A commonly used definition is 
%\begin{equation}\label{eq:sRm}
%\Rm=\frac{\ell \overline{u_{\rm rms}}}{\eta},
%\end{equation}
%where $\ell$ is the forcing scale.
%However, in these simulations of the ISM, given the 
%%MJK ?
%%FAG opposit of what I meant - reworded
%%Galilean invariance of the 
%inhomogeneity of the 
%turbulence, variation within hot and warm medium, and inclusion of
%hyperdiffusion, defining a single forcing scale from the SN explosions or 
%common rms velocity is unreliable.
%%MJK "a field" sounds alien to me
%%FAG: absolutely no alienation allowed
%%We, therefore, directly compute a field of Reynolds numbers from the equations,
%We, therefore, compute Reynolds numbers at each grid point
%directly from the equations,
%for example,
%%-------------------------------------------------------------------------------
%\begin{eqnarray}\label{eq:Rm}
%  \Rm = \frac{\left|\vect{u}\times\vect{B}\right|}{
%    \left|\eta\vect\nabla^2\vect{A}+\eta_3\vect\nabla^6\vect{A}\right|}.
%\end{eqnarray}
%%-------------------------------------------------------------------------------
%Hence, for the high resolution runs at 20\,Myr we obtain for $\eta=10^{-4}$ 
%$\langle\Rm\rangle=4.1$ with standard deviation 5.7, whereas using
%equation\,\eqref{eq:sRm} 
%%FAG: added overline for consistency
%$\Rm = 15000$, for $\eta=10^{-4}$, $\ell=50\pc$ and
%$\overline{u_{\rm rms}}=30\kms$.
%%FAG: added
%The spread of Rm, using equation\,\eqref{eq:Rm}, extends well above 15000 and to
%$\ll1$.
%Given the multiphase structure of the ISM, the mean Rm appears unhelpful and 
%the concept of a critical Rm for SSD may differ between phases and regions.
%In the real ISM such variation in Rm will be present.
%Microscopic resistivity may be temperature dependent \citep{CSR50}, but even if
%we consider only the turbulent resistivity as relevant to the definition of 
%Rm, this will also vary between phases due to the differences in typical 
%length scales and velocities.
%We shall address this challenge in future work, but here we shall use the 
%explicit input $\eta$ rather than Rm to discriminate between models.
%%
%FAG: word count
%FAG: added overline for consistency and added eq number as NS suggested
 A commonly used expression for Rm is 
 \begin{equation}\label{eq:sRm}
 \Rm=\frac{\ell \overline{u_{\rm rms}}}{\eta},
 \end{equation}
 where $\ell$ is the forcing scale.
 However, the inhomogeneity of the turbulence and changing multiphase
 composition of the ISM render unreliable defining a single forcing scale from
 SN explosions or common rms velocity.
 We, therefore, consider compute Rm at each grid point directly from the
 equation,
 %-------------------------------------------------------------------------------
 \begin{eqnarray}\label{eq:Rm}
   \Rm = \frac{\left|\vect{u}\times\vect{B}\right|}{
     \left|\eta\vect\nabla^2\vect{A}+\eta_3\vect\nabla^6\vect{A}\right|}.
 \end{eqnarray}
 %-------------------------------------------------------------------------------
 Hence, for $\dx=0.5\pc$ and $\eta=10^{-4}$ at 20\,Myr we obtain 
 $\langle\Rm\rangle=4.1\pm5.7$.
 Using equation\,\eqref{eq:sRm} 
 $\Rm = 15000$, for $\eta=10^{-4}$, $\ell=50\pc$ and
 $\overline{u_{\rm rms}}=30\kms$.
 With equation\,\eqref{eq:Rm} $\Rm\in(\ll1,>15000)$0.
 The mean Rm appears unhelpful and the concept of a critical Rm for SSD may
 differ between phases and regions.
 In the real ISM such Rm variation will be present.
 Microscopic resistivity appears temperature dependent \citep{CSR50}.
 Even considering only turbulent resistivity to have relevance, 
 typical length scales and rms velocities differ between phases.
 This challenge is for future work, but here we shall use explicit input $\eta$
 to discriminate between models.


 We have set $\nu=0$ and $\nu_3=\eta_3$ sufficient to numerically resolve the
 grid scale.
 $\Pm = \Rm/\Rey$, and like Rm we have $Pm\in(\ll1,\gg1)$, due
 to inclusion of shock capturing viscosity and differences in
 definition of hyper viscosity and hyper resistivity.
 Hence, some part of each simulation will be characterised by high Pm,
 however $\langle\Pm\rangle<1$.
 This is an even more difficult regime in which to excite SSD than high $\Pm$
 typical of the ISM \citep{HBD04}.
 In separate experiments with $\nu>\eta$ (high Pm), which we shall include in
 more in depth future analysis, we do in fact see increased growth rates for
 lower $\eta$, as expected \citep{Sch07}.
%
%-------------------------------------------------------------------------------
\section{Resolution and resistivity} \label{sec:results}
%-------------------------------------------------------------------------------

%-------------------------------------------------------------------------------
\begin{figure*}
\gridline{\fig{figs/0_5pc-eB-nu4.png}{0.45\textwidth}{(a)}
            \fig{figs/1pc-eB-nu4.png}{0.45\textwidth}{(b)}
          }
\gridline{  \fig{figs/2pc-eB-nu4.png}{0.45\textwidth}{(c)}
            \fig{figs/4pc-eB-nu4.png}{0.45\textwidth}{(d)}
          }
\gridline{
            \fig{figs/2pc-eB-nu5.png}{0.45\textwidth}{(e)}
            \fig{figs/4pc-eB-nu6.png}{0.45\textwidth}{(f)}
          }
  \begin{picture}(0,0)(0,0)
    \put( 56,428){\begin{scriptsize}{\sf{$\delta x=0.5$pc, $\sigma=0.2\SNr$}}\end{scriptsize}}
    \put(445,438){\begin{scriptsize}{\sf{$\delta x=1.0$pc}}\end{scriptsize}}
    \put(445,428){\begin{scriptsize}{\sf{$\sigma=0.2\SNr$}}\end{scriptsize}}
    \put( 56,256){\begin{scriptsize}{\sf{$\delta x=2.0$pc, $\sigma=0.2\SNr$}}\end{scriptsize}}
    \put(445,266){\begin{scriptsize}{\sf{$\delta x=4.0$pc}}\end{scriptsize}}
    \put(445,256){\begin{scriptsize}{\sf{$\sigma=0.2\SNr$}}\end{scriptsize}}
    \put( 56, 78){\begin{scriptsize}{\sf{$\delta x=2.0$pc, $\sigma=\SNr$}}\end{scriptsize}}
    \put(405, 78){\begin{scriptsize}{\sf{$\delta x=4.0$pc, $\sigma=\SNr$}}\end{scriptsize}}
  \end{picture}
\caption{
Resistivity $\eta$ for given $\dx$ and $\dot\sigma$.
$\SNr\simeq 50$\,kpc$^{-3}$\,Myr$^{-1}$.
Time axes vary to accommodate SSD saturation at all $\dx$ with $\eta=10^{-4}$.
\label{fig:eb-nu}}
\end{figure*}
%-------------------------------------------------------------------------------

%FAG: word count
%In Figure\,\ref{fig:eb-res} the response of the magnetic energy density 
%$e_B/\overline{e_K}$ to resolution is shown for resistivity $\eta=10^{-4}$,
%panel\,(a) and $\eta=10^{-3}$, panel\,(b).
%The time in Myr is plotted on a log scale to better present the differences in 
%the higher resolution runs, which saturate much faster than at $\dx=4\pc$.
%Whilst for $\eta=10^{-4}$ the dynamo growth rates are very slow at $\dx=4\pc$,
%the growth rate shows some convergence near $\dx=0.5\pc$ and the saturation
%level of around 5\% of $\overline{e_K}$ is independent of resolution.
%%MJK How do we know that we currently capture the correct one?
%%FAG Not sure to what 'one' you refer?
%%MJK To false/true convergence?
%At $\eta=10^{-3}$, there is a false convergence \citep{FMA91} between
%$\dx=2$ and $4\pc$, with the magnetic energy decaying at similar
%rates. However, at higher resolution we see that the results converge around a
%solution with dynamo amplification of the field occurring between
%40 and 60~Myr.
 Magnetic energy density $e_B/\overline{e_K}$ for each $\dx$ is plotted in
 Figure\,\ref{fig:eb-res}(a) for resistivity $\eta=10^{-4}$ and (b) 
 $\eta=10^{-3}$.
%FAG: redundant 
%The time in Myr is plotted on a log scale to better present the differences in 
%the higher resolution runs, which saturate much faster than at $\dx=4\pc$.
 For $\eta=10^{-4}$ dynamo growth, slower at $\dx=4\pc$, shows some
 convergence near $\dx=0.5\pc$.
 Saturation at around 5\% of $\overline{e_K}$ is independent of resolution.
%MJK How do we know that we currently capture the correct one?
%FAG Not sure to what 'one' you refer?
%MJK To false/true convergence?
%FAG the under resolved kinetic energy from Fig 5 explains key dynamics are missing from the low res. For space I think it worth saying so later in text.
 At $\eta=10^{-3}$, there is false convergence \citep{FMA91} of solutions with
 similar magetic energy decay at $\dx=2$ and $4\pc$.
 However, high resolution solutions converge to rapid magnetic
 amplification occurring between 40 and 60~Myr.
 
%%FAG added
%For $\dx\geq2\pc$ the growth rate changes sporadically.
%Accelerated growth is evident in Figure\,\ref{fig:eb-res}(a) and (b) just
%before 20\,Myr, at around 40\,Myr and again at 100\,Myr.
%In between there is a lower growth rate or decay, depending on $\dx$ or $\eta$.
%This is a consequence of the multiphase structure of the ISM hosting more than
%one turbulent regime, in which dynamo modes with different growth rates 
%switch over time between subcritical and supercritical for SSD.
%%FAG: moved forward from below handle the variation in growth rates and drop hot claim
%Amongst the general suite of simulations in addition to those presented in this
%letter, some models exhibit the phenomenon of thermal runaway
%\citep[see e.g.,][]{LOCBN15}.
%The hot gas fills a sufficient fractional volume, that the supernova remnants 
%form bubbles, which cannot cool rapidly and the domain becomes saturated 
%with hot gas.
%Simulations with high resolution are more easily perturbed into this regime, but
%also low resolution runs with higher $\dot\sigma$.
%The multiphase influence on SSD and cause of the sporadic acceleration of the
%dynamo shall be investigated in subsequent study.
%
%FAG word count
 For $\dx\geq2\pc$ growth rates vary sporadically.
 Growth is accelerated in Figure\,\ref{fig:eb-res}(a) and (b) near 20\,Myr,
 40\,Myr and again at 100\,Myr.
 Elsewhere is decay or lower growth rate, depending on $\dx$ or $\eta$.
%FAG added
 The irregular SN forcing and varying fractional volume of the multiphase ISM
 account for these alternating regimes, but we defer analysis of the details
 to future work.
 
%FAG don't need this now? Defer to paper 2
%This is a consequence of the multiphase structure of the ISM hosting more than
%one turbulent regime, in which dynamo modes with different growth rates 
%switch over time between subcritical and supercritical for SSD.
%%FAG: moved forward from below handle the variation in growth rates and drop hot claim
%Amongst the general suite of simulations in addition to those presented in this
%letter, some models exhibit the phenomenon of thermal runaway
%\citep[see e.g.,][]{LOCBN15}.
%The hot gas fills a sufficient fractional volume, that the supernova remnants 
%form bubbles, which cannot cool rapidly and the domain becomes saturated 
%with hot gas.
%Simulations with high resolution are more easily perturbed into this regime, but
%also low resolution runs with higher $\dot\sigma$.
%The multiphase influence on SSD and cause of the sporadic acceleration of the
%dynamo shall be investigated in subsequent study.
%

%In Figure\,\ref{fig:eb-nu} we show how at each resolution $e_B/\overline{e_K}$
%evolves for various values of $\eta$.
%We note that at $\dx=0.5$, panel (a), and $1\pc$, panel (b), the profile at
%$\eta=10^{-5}$ is indistinguishable from $\eta=0$, and so numerical resistivity
%continues to control the dynamics.
%At $\eta=10^{-4}$ the dynamo diverges from $\eta=0$, such that physical
%%FAG: take out Rm
%%resistivity is dynamically dominant, at least for some of the domain and Rm is
%%unevenly defined by a combination of physical and numerical resistivity.
%resistivity is dynamically dominant, at least for some of the domain and
%unevenly determined by a combination of physical and numerical resistivity.
%%FAG: removing Rm
%%At all $\dx$ $\eta=10^{-3}$ is clearly well resolved and dynamics and
%%Rm well defined by the physical application of resistivity, varying locally with
%%$3.8\lesssim\langle\Rm\rangle\lesssim6$, as discussed above from the induction
%%equation of $\Rm=15000$ using the common definition. 
%At all $\dx$ $\eta=10^{-3}$ is clearly well resolved and dynamics
%well defined by the physical application of resistivity.
%Although models with the same SN rate have the same schedule and location of
%SN, at low resolution the timestep is longer, such that the actual timing and 
%environment of the explosions can differ between models, so the statistical
%noise is more evident, particularly in panels (e) and (f).
%FAG word count
 In Figure\,\ref{fig:eb-nu} at each $\dx$ we plot $e_B/\overline{e_K}$
 for given $\eta$.
 At $\dx=0.5$ and $1\pc$, (a and b), profiles at $\eta=10^{-5}$ are
 indistinguishable from $\eta=0$, and so numerical resistivity still
 controls dynamics.
 At $\eta=10^{-4}$ the dynamo diverges from $\eta=0$, such that physical
 resistivity is dynamically dominant, at least in part of the domain.
 At all $\dx$ physical resistivity $\eta=10^{-3}$ is clearly well resolved
 and determines the dynamics.
 Models with common $\dot\sigma$ have the same schedule and location of SN,
 but the timestep at low resolution is larger, such that actual timing and
 explosive environment can differ between models.
 Increased statistical noise is therefore evident, particularly (e and f).


%FAG moved up 
%%FAG: added to paragraph 2 previous 
%%The evolution is complicated by the impact \emph{thermal runaway}
%%\citep[see e.g.,][]{LOCBN15}, combination of multiple SN into superbubbles
%%resulting in persistent hot regions.
%Amongst the general suite of simulations in addition to those presented in this
%letter, some models exhibit the phenomenon of {thermal runaway}
%\citep[see e.g.,][]{LOCBN15}.
%The hot gas fills a sufficient fractional volume, that the supernova remnants 
%form bubbles, which cannot cool rapidly and the domain becomes saturated 
%with hot gas.
%Simulations with high resolution are more easily perturbed into this regime, but
%also low resolution runs with higher $\dot\sigma$.

%FAG defer this claim to Paper 2
%We observe that during some phases of accelerated dynamo growth, the magnetic
%field is growing most rapidly in the hot gas, whilst during majority of the 
%duration of the runs magnetic fields grow most rapidly in the warm gas and the
%dynamo growth overall follows a lower exponential.
%This is one reason for the complication in characterising the turbulence in 
%these SSDs, with Rm, Pm and $u_{\rm rms}$ varying is space and over time.

%%FAG removing Rm and repetition
%At first glance the growth profiles might appear to contradict conventional
%wisdom with growth rates at high $\eta$ exceeding growth at low $\eta$, for
%example, in Figure\,\ref{fig:eb-nu}(a) the growth rate for $\eta=10^{-3}$ (cyan,
%dotted) at 19\,Myr is clearly higher than for $\eta=10^{-4}$ (red, dash-dotted)
%at 12\,Myr.
%However if we understand this in the context of the changing quality of the 
%turbulence over time and recognise that the background turbulence evolves in
%common for $\dx=0.5$ and 1\,pc, this can be reconciled.
%%growth rates for high resistivity (low Rm) can exceed comparable rates for high
%%Rm, in contradiction to dynamo theory and previous experimental results.
%%growth rates for high resistivity can exceed comparable rates for low $\eta$,
%%in contradiction to dynamo theory and previous experimental results.
%%Closer inspection of each plot, zooming in on specific intervals, however,
%%FAG :changed to eta, as table in paper suggest mean Rm cannot explain this
%%confirms that growth rates are higher as expected for higher Rm.
%%confirms that growth rates are higher as expected for lower $\eta$.
%%MJK Hmmh...
%%MJK Which figure?
%%FAG: added Fig
%%For example between 8 and 15 Myr there is weak growth or decay in panels (a)
%%and (b) with critical $\eta$ for dynamo between $10^{-4}$ and $10^{-3}$.
%%Rm and corresponding growth at low $\eta$ is higher in the $0.5\pc$ simulation
%%with lower numerical resistivity, but 
%%converge for $\eta=10^{-3}$, where the physical resistivity is most resolved.
%For example between 8 and 15\,Myr the strength of growth (or decay) in
%Figure\,\ref{fig:eb-nu}(a) and (b) reduces as $\eta$ increases.
%The accelerated growth around 19\,Myr for $\eta=10^{-3}$ coincides with an
%even higher acceleration for $\eta\le10^{-4}$, consistent with theory.
%%FAG: fast replaced with acceleration, etc.
%%Between 15 and 20\,Myr there is a fast dynamo driven in the hot gas, in which
%%Between 15 and 20\,Myr there is an acceleration in the dynamo driven in the hot
%%gas, in which growth rates are related to the level of resistivity.
%%MJK Rm is not given for all models... difficult to know what are you referring to...
%%FAG removing Rm
%%The high Rm models at $0.5\pc$ already saturate by 20\,Myr, but for $1\pc$ there
%The low $\eta$ models at $0.5\pc$ already saturate by 20\,Myr, but for $1\pc$ there
%%FAG: lower
%%is another slow dynamo driven in the warm gas up to 40\,Myr and then a
%%subsequent 'hot' dynamo resulting in saturation for all models within 60\,Myr,
%is another epoch of lower growth rate for the SSD up to
%40\,Myr and then a subsequent accelerated dynamo resulting in saturation for
%all models within 60\,Myr, including for $\eta=10^{-3}$.
%
%FAG word count
%No need to draw attention to this now just explain
%At first glance the growth profiles might appear to contradict conventional
%wisdom with growth rates at high $\eta$ exceeding growth at low $\eta$, for
%example, in Figure\,\ref{fig:eb-nu}(a) the growth rate for $\eta=10^{-3}$ (cyan,
%dotted) at 19\,Myr is clearly higher than for $\eta=10^{-4}$ (red, dash-dotted)
%at 12\,Myr.
%However if we understand this in the context of the changing quality of the 
%turbulence over time and recognise that the background turbulence evolves in
%common for $\dx=0.5$ and 1\,pc, this can be reconciled.
%FAG added
 Growth rates are sporadic, within models, but these are consistent between
 models.
 For example, between 8 and 15\,Myr the strength of growth (or decay) in 
 Figure\,\ref{fig:eb-nu}(a, b) reduces as $\eta$ increases.
 Accelerated growth near 19\,Myr for $\eta=10^{-3}$ coincides with
 even higher acceleration for $\eta\le10^{-4}$, consistent with theory.
 The low $\eta$ models at $0.5\pc$ saturate already by 20\,Myr, but for $1\pc$
 there is another epoch of lower growth rate up to 40\,Myr and then a
 subsequent acceleration resulting in saturation for all models within
 60\,Myr, including for $\eta=10^{-3}$.


%%MJK Now we have too many different names for this dynamo: hot, fast, and what not. Should choose one.
%%FAG: using terms of accelerated growth and lower growth rate
%At low resolution, with $\dot\sigma=0.2\SNr$ panels (c) and (d), $\eta=10^{-4}$
%%FAG not hot
%%is not distinct from numerical resistivity, and there is no hot dynamo at all
%is not distinct from numerical resistivity, and there is no separate
%acceleration of the dynamo at all for $\dx=4\pc$ with the well resolved
%$\eta=10^{-3}$.
%%FAG not hot
%%At 100\,Myr there is a hot dynamo at $\dx=2\pc$, but this is not
%At 100\,Myr there is a period of accelerated dynamo at $\dx=2\pc$, but this is
%not sustained and for $\eta=10^{-3}$ subsequently diffuses away.
%FAG: word count
 In Figure\,\ref{fig:eb-nu}(c, d) at low resolution with
 $\dot\sigma=0.2\SNr$ profiles for $\eta=10^{-4}$ are not distinct $\eta=0$,
 and there is no dynamo at all for $\dx=4\pc$ with well resolved $\eta=10^{-3}$.
 At 100\,Myr there is a period of accelerated dynamo at $\dx=2\pc$, but 
 for $\eta=10^{-3}$ this is not sustained and subsequently diffuses away.

%%FAG removing Rm and para break
%%At $\dot\sigma=\SNr$, panels (e) and (f), the increased Rm due to the 
%At $\dot\sigma=\SNr$, Figure\,\ref{fig:eb-nu}(e) and (f), the 
%higher forcing rate is sufficient to produce a dynamo at $\eta\geq10^{-3}$, but
%not for $\eta=5\cdot10^{-3}$.
%%FAG: added
%Mean sound speed Mach number, $\overline\Ms=0.8$ for $\dot\sigma=\SNr$,
%compared to $\overline\Ms=0.5$ for $0.2\SNr$.
%High Mach number has been demonstrated to impede the SSD in isothermal 
%turbulence \citep{Haugen:2004M}, which is contradicted by this result.
%%
%Due to the increased statistical noise we cannot confirm that $\eta\leq10^{-4}$
%applies beyond the numerical resistivity.
%However, at $\dx=2\pc$ $\eta=5\cdot10^{-4}$ (magenta, solid) is resolved and
%at $\dx=4\pc$ the resolved limit is $\eta\lesssim10^{-3}$ (cyan, dotted).
%%FAG not hot
%%In panel (e) we see two hot dynamo phases at 90  and 110\,Myr, which are not 
%In panel (e) for $\dx=2\pc$ we see two phases of accelerated dynamo growth at
%90 and 110\,Myr, which are not present for $\dx=4\pc$ in panel (f).
%%FAG removing Rm and para break
%FAG word count 
 At $\dot\sigma=\SNr$, Figure\,\ref{fig:eb-nu}(e, f), the higher forcing rate is sufficient to produce SSD at $\eta\geq10^{-3}$, but not $\eta=5\cdot10^{-3}$.
 Mean sound speed Mach number, $\overline\Ms=0.8$ for $\dot\sigma=\SNr$,
 compared to $0.5$ for $0.2\SNr$.
 High $\Ms$ has been demonstrated to impede SSD in isothermal turbulence
 \citep{Haugen:2004M}, which is contradicted here.
 With the level of statistical noise we cannot affirm $\eta\leq10^{-4}$
 applies beyond numerical resistivity.
 However, $\eta\gtrsim5\cdot10^{-4}$ is
 resolved at $\dx=2\pc$ (magenta, solid) and 
 at $\dx=4\pc$ the resolved limit is $2\cdot10^{-4}<\eta\lesssim10^{-3}$
 (purple, solid) and (cyan, dotted).
 For $\dx=2\pc$, (e), we see two phases of accelerated SSD at 90 and
 110\,Myr, which are not present for $\dx=4\pc$, (f).

%-------------------------------------------------------------------------------
\begin{figure*}
\gridline{\fig{figs/0_5pcPm0e-4_0Bpower.png}{0.45\textwidth}{(a)}
          \fig{figs/0_5pcPm0e-4_0kpower.png}{0.45\textwidth}{(b)}
          }
\gridline{\fig{figs/0_5pcPm0e-3_0Bpower.png}{0.45\textwidth}{(c)}
          \fig{figs/0_5pcPm0e-3_0kpower.png}{0.45\textwidth}{(d)}
          }
\caption{
Compensated energy spectra $\dx=0.5\pc$, Myr in legends;
(a, b) $\eta=10^{-4}$ and  (c, d) $\eta=10^{-3}$.
Compensation profiles: Kazantsev
$k^{3/2}$ (a, c); and Kolmogorov $k^{-5/3}$ (b, d).
\label{fig:4power}}
\end{figure*}
%-------------------------------------------------------------------------------

%In contrast to the simplified model in Section\,\ref{sec:ssd-tang}, SN-driven
%turbulence does not have a well-defined forcing scale, due to the heterogeneous
%temperature and density of the ISM and random clustering of explosions.
%%MJK How did you come up with this value?
%%FAG: k = 2pi/l, k in kpc^{-1}
%%In our simulations the forcing scale will range between
%%10 and $50\pc$,
%%or $k\in(3,16)$.
%In our simulations the forcing scale will be distributed at scales 
%greater than about $60\pc$ \citep[][Table\,3]{HSSFG17}, or $k\lesssim17$.
%%FAG: clarify choice of time interval
%%In Figure\,\ref{fig:4power} we show the evolving compensated spectra for the
%%high resolution kinetic energy in the kinematic phase (b) for $\eta=10^{-4}$
%%and during decay (d) for $\eta=10^{-3}$.
%In Figure\,\ref{fig:4power} we show the evolving compensated spectra for
%$\dx=0.5\pc$ between 9 and 32\,Myr.
%We compare spectra for $\eta=10^{-4}$, (a) and (b), when the magnetic field is 
%amplified to saturation and for $\eta=10^{-3}$, (c) and (d) while the field
%persists near seed strength.
%The kinetic spectra show the same magnitude fluctuations between the models
%over time,
%except at 32\,Myr, when saturation of the dynamo, (b), reduces
%the energy slightly, compared to (d). 
%The compensated magnetic energy spectra in panels (a) and (c) have ranges 
%conforming to the Kazantsev inverse cascade in the SSD case (a) extending to
%$k\gtrsim 20$, consistent with the SSD behaviour of the simplified model in
%Figure\,\ref{fig:tangling}\,(b).
%In panel (c) of Figure\,\ref{fig:4power} the Kazantsev range in similar fashion
%to Figure\,\ref{fig:tangling}\,(c) is in the range 
%$K\lesssim10$ except
%for 22\,Myr, which corresponds to a short growth spurt in
%Figure\,\ref{fig:eb-res}\,(b) and consistent with the presence of SSD.
%So for these SN low Pm models any Kazantsev spectral range dissipates 
%at $k$ below the extent of Kolmogorov range.
%This makes the SSD in the models even less efficient 
%than the high Pm ISM,
%where transfer from kinetic energy can occur at every wavenumber in the
%Kolmogorov spectrum.
%FAG word count
 Unlike the simplified model in Section\,\ref{sec:ssd-tang}, SN-driven
 turbulence does not have a well-defined forcing scale, due to the
 heterogeneous ISM structure and random explosions.
 The forcing scale will be distributed at scales greater than about $60\pc$
 \citep[][Table\,3]{HSSFG17}, or $k\lesssim17$.
 In Figure\,\ref{fig:4power} we show the evolving compensated spectra for 
 $\dx=0.5\pc$ between 9 and 32\,Myr.
 We compare spectra for $\eta=10^{-4}$, (a) and (b), spanning accelerated 
 dynamo and saturation, and for $\eta=10^{-3}$, (c) and (d) as the field
 persists near seed strength.
 The kinetic spectra over time match between models, until saturation
 of the dynamo at 32\,Myr, (b), diminishes its energy, relative to (d).
%FAG added
 The Kolmogorov range extends at all times to $k>20\kpc^{-1}$ and mostly 
 $k>40$. 
%FAG add para

 The compensated magnetic energy spectra in Figure\,\ref{fig:4power}(a, c)
 have ranges conforming to the Kazantsev inverse cascade.
 For $\eta=10^{-4}$ this range extends to $k\gtrsim 20$, above the forcing
 scale and consistent with SSD in Figure\,\ref{fig:tangling}(b), 
 until it contracts upon saturation to $k<10$, consistent with no dynamo in
 Figure\,\ref{fig:tangling}(c).
 In Figure\,\ref{fig:4power}(c) the Kazantsev range occurs at $k\lesssim10$
 except for 19 -- 22\,Myr, which corresponds to a short growth spurt in
 Figure\,\ref{fig:eb-res}(b).
 The low-$k$ Kazantsev versus Kolmogorov range signifies the low Pm in these
 SN models. 
 In the high Pm ISM, where transfer from kinetic energy can occur at every
 wavenumber in the Kolmogorov spectrum, SSD is even easier to excite.

%-------------------------------------------------------------------------------
\begin{figure*}
\gridline{\vspace{-0.7cm} \fig{figs/nu0_Bpower.png}{0.45\textwidth}{\vspace{-1.3cm}\hspace{-2cm}(a)}
          \vspace{-0.7cm} \fig{figs/nu0_kpower.png}{0.45\textwidth}{\vspace{-1.3cm}\hspace{-2cm}(b)}}                                                                             
\gridline{\vspace{-0.7cm}\fig{figs/nu1_Bpower.png}{0.45\textwidth}{\vspace{-1.3cm}\hspace{-2cm}(c)}
          \vspace{-0.7cm}\fig{figs/nu1_kpower.png}{0.45\textwidth}{\vspace{-1.3cm}\hspace{-2cm}(d)}}                                                                             
\gridline{\vspace{-0.7cm} \fig{figs/nu10_Bpower.png}{0.45\textwidth}{\vspace{-1.3cm}\hspace{-2cm}(e)}
          \vspace{-0.7cm} \fig{figs/nu10_kpower.png}{0.45\textwidth}{\vspace{-1.3cm}\hspace{-2cm}(f)}}                                                                             
\gridline{\vspace{-0.3cm}  \fig{figs/SN_Bpower.png}{0.45\textwidth}{\vspace{-1.3cm}\hspace{-2cm}(g)}
          \vspace{-0.3cm}  \fig{figs/SN_kpower.png}{0.45\textwidth}{\vspace{-1.3cm}\hspace{-2cm}(h)} }
  \begin{picture}(0,0)(0,0)
    \put(165,455){\begin{scriptsize}{\sf{$\eta=0$, $\dot\sigma=0.2\SNr$}}\end{scriptsize}}
    \put(420,455){\begin{scriptsize}{\sf{$\eta=0$, $\dot\sigma=0.2\SNr$}}\end{scriptsize}}
    \put(150,325){\begin{scriptsize}{\sf{$\eta=10^{-4}$, $\dot\sigma=0.2\SNr$}}\end{scriptsize}}
    \put(410,325){\begin{scriptsize}{\sf{$\eta=10^{-4}$, $\dot\sigma=0.2\SNr$}}\end{scriptsize}}
    \put(150,192){\begin{scriptsize}{\sf{$\eta=10^{-3}$, $\dot\sigma=0.2\SNr$}}\end{scriptsize}}
    \put(410,192){\begin{scriptsize}{\sf{$\eta=10^{-3}$, $\dot\sigma=0.2\SNr$}}\end{scriptsize}}
    \put(60,62){\begin{scriptsize}{\sf{$\dx=2\pc$}}\end{scriptsize}}
    \put(315,62){\begin{scriptsize}{\sf{$\dx=2\pc$}}\end{scriptsize}}
  \end{picture}
\caption{
Compensated energy spectra as in Figure\,\ref{fig:4power}.
(a) -- (f):
 $t=19.5\Myr$ for $\dx=0.5$ \& $1\pc$ and 100\,Myr for $\dx=2$ \& $4\pc$;
$\dx$ in the legends.
(g) -- (h): $\dot\sigma=0.2\SNr$ $t=100\Myr$ and $\dot\sigma=\SNr$ 
$t=140\Myr$. $\dot\sigma$ and $\eta$ in the legends.
\label{fig:3power}}
\end{figure*}
%-------------------------------------------------------------------------------

In Figure\,\ref{fig:3power} panels (a) -- (f) the effect of resolution on the
compensated energy spectra is shown for $\eta=0$, $10^{-4}$ and $10^{-3}$
with $\dot\sigma=0.2\SNr$.
%FAG: updated 
%The time of the spectral snapshot for $\dx=0.5$ and $1\pc$ correspond to a
%common phase of rapid magnetic growth fastest in the hot gas, while for
%$\dx=2$ and $4\pc$ it corresponds to a period of hot-phase rapid growth at
The time, $19.5\Myr$, of the spectral snapshot
corresponds to a common phase of accelerated magnetic growth for $\dx=0.5$ and $1\pc$.
For $\dx=2$ and $4\pc$ the time 100\,Myr corresponds to a period of accelerated
growth at $2\pc$, which is absent in the $4\pc$ model.
The hydrodynamic parameters are fixed for each resolution and are 
consistent between panels (b), (d) and (f).
We note the convergence in the kinetic energy spectrum for
$\dx=0.5\pc$ and $\dx=1\pc$, apart from the energy cutoff at higher $k$
arising from differences in resolution.
Also, the total kinetic energy is reduced with low resolution due to 
increased viscous dissipation acting at lower $k$.

%FAG para break
%To fully resolve the kinetic energetics for scales larger than $k=24$ corresponding to $\ell\simeq40\pc$ would therefore require $\dx\gtrsim1\pc$.
To fully resolve the kinetic energetics for scales larger than $k=24$, that is
$\ell\gtrsim40\pc$, would therefore require $\dx\lesssim1\pc$.
The energy spectrum at $\dx=2\pc$ shares the same characterstics as the high
%FAG added 
%resolution runs, but with some loss of energy.
resolution runs, but with some loss of energy, as does $\dx=4\pc$, but with 
even greater dissipative losses and peak $k\simeq3$.
The kinetic spectra display a bottleneck effect \citep{Falkovich94,HBD03},
%with a bump at $k\simeq50$, 
an energy cascade less efficient than $k^{-5/3}$ terminating at a bump,
before rapid dissipation at high $k$.
This bottleneck shifts to lower $k$ as $\dx$ increases, but always at higher
$k$ than the Kazantsev range in the corresponding magnetic spectrum, reflective
of the low Pm character of these models.

Perhaps surprisingly, there is very little difference in the kinetic energy
spectrum, Figure\,\ref{fig:3power}(h), between simulations with
$\dot\sigma=0.2\SNr$ and $\SNr$, despite five times as much energy being 
applied from the forcing in the latter case.
From the models with $\eta=10^{-3}$ it is evident that there is more energy
near the bottleneck in the high $\dot\sigma$ case (cyan dotted line), so more
energy can be tranferred to the SSD.
The comparison is obscured for $\eta=10^{-4}$, because the high $\dot\sigma$
magnetic field is already 3 or 4 orders of magnitude stronger, aquiring some of
the kinetic energy (red dash-dotted line).
More hot gas from the higher SN rate may also improve the efficiency of SSD,
which we shall investigate in future work. 



\section{Conclusions}\label{sec:conc}

%MJK Now the summary (but it could be also outdated)
%MJK does not really address the main results presented in the paper. 
%MJK 1) SSD can be excited without imposed field present in the system, i.e. the confirmation of Balsara results being not caused by the presence of it, as Detlef thought.
%MJK 2) Convergence (and hopefully a true one) in between 0.5 and 1 pc res, and their implication for the previously published simulations.
%MJK 3) General agreement (hopefully) with the picture of growth rate being somehow related to Rm, and some (maybe forcefully handwavy) estimate of the critical value.
%NS: As Balsara paper talks also about helicity, may be we could briefly say
%NS: that this being SSD is unlikely to be affected by the kinetic helicity
%NS: at least the growth rates.
%MJK But do we know this for sure? Have we looked at the helicity?
%MJK In Miikka's paper we find evidence for kin. helicity making SSD much easier to excite.
%NS: Ok, I recall some papers (also by Axel I think) where helical/non-helical cases
%NS: produced identical growth rates, but this is perhaps minor issue here.
%FAG: added
In this letter we confirm, without the use of an imposed magnetic field, that
the field amplification demonstrated by \citet{BKMM04} was evidence of SSD in
the ISM and not just caused by tangling of their imposed field.
Through the most extensive resolution and parameter study to date, we conclude
that SN-driven turbulence easily excites SSD even at SN rates well below the
Galactic value.
Our conclusion is supported by noting that the resistivity of the ISM is far
smaller than we can resolve numerically, so the ISM is far more susceptible to
dynamo action than our models.
Our models with $\dx=0.5$ and $1\pc$ with $\eta=10^{-3}\kpc\kms$ (see
Figure\,\ref{fig:eb-res}b), show that a seed field of less than 1~nG can be
amplified to saturation at microgauss levels within about 10\,Myr.

We further show that simulations with insufficient resolution can appear to
converge to a false solution lacking dynamo activity. This can occur because
these simulations are not scale independent.
The SN energy input and the physically motivated ISM cooling processes impose
length and time scales that must be adequately resolved.
To reach true convergence requires resolution of $1\pc$ or better.
In our models, resolutions $\dx\geq2\pc$ not only give an incorrect dynamo
solution, but also exhibit significant kinetic energy losses due to excess
dissipation.
%FAG added
This also affects the energy spectrum at the largest scales, so should be 
considered when interpreting results using adaptive mesh refinement.
%
We do, however find, independent of resolution, that when an SSD is excited it
saturates at about 5\% of the energy equipartition level.
At low resolution this is a lower bound, because the time-averaged kinetic
energy density is understated, due to dissipative losses.
This might account for the discrepency between the energy density of the mean
magnetic field and the random magnetic field in \citet{Gent:2013b}, due to the
LSD being well resolved while the SSD remained under resolved.

We find that the conventional approach from dynamo theory to categorise the 
turbulence according to Rm based on a forcing scale $\ell$, random velocity and
resistivity, is inadequate for such a complicated system.
%FAG: defer thermal runaway discussion to future paper
%The ISM appears to host at least two regimes of SSD, apparently embedded in the
%hot or warm phases, and with radically different 
%dynamo characteristics and growth rates, and which occupy changing fractional
%volumes and epochs.
%This is further exacerbated by the susceptibility to thermal runaway, which 
%alters the SSD properties.
%Explaining the interaction of these SSDs will require more sophisticated
%statistical and perhaps toplogical techniques, in advance of being able to 
%address how such SSD interacts with LSD.
The ISM appears to host multiple regimes occupying changing fractional volumes
and exciting SSD instabilities with different thresholds and growth rates.
Explaining the interaction of these SSDs will require more sophisticated
statistical and perhaps toplogical techniques, in advance of being able to 
address how such SSD interacts with LSD.
%FAG Moved from above
%FAG: defer thermal runaway discussion to future paper
%Due to the more diffusive numerics there is a reduced susceptibility to
%thermal runaway at low resolution. 
%Higher SN rates may permit coarse models to enter thermal
%runaway and more easily excite SSD, but it is likely that models with 
%$\dx$ much above $4\pc$ will suppress SSD in simulations of the ISM.

%FAG Moved from above
%%FAG removing thermal runaway focus
%To exclude SSD in SN-driven turbulence simulations will require 
%$\eta\gtrsim10^{-3}\kpc\kms$ or SN implementation which avoids thermal runaway.
%In \citet{Gressel:2008,GE20}, $\dx$ is 8.3 and $6.7\pc$, respectively
%and $\eta\simeq6.5\cdot10^{-3}\kpc\kms$, which in all probabilty excludes SSD.
%\citet{Gent:2013b} applied $\eta\simeq8\cdot10^{-4}\kpc\kms$ with $\dx=4\pc$,
%which would support SSD with SN rates similar to the solar neighbourhood, even
%without thermal runaway.
With grid resolution $\dx\geq2\pc$ SSD in SN-driven turbulence simulations can be
excluded with $\eta\gtrsim10^{-3}\kpc\kms$.
In \citet{Gressel:2008,GE20}, $\dx$ is 8.3 and $6.7\pc$, respectively
and $\eta\simeq6.5\cdot10^{-3}\kpc\kms$, which in all probabilty excludes SSD.
\citet{Gent:2013b} applied $\eta\simeq8\cdot10^{-4}\kpc\kms$ with $\dx=4\pc$,
which would support SSD with SN rates similar to the solar neighbourhood.
The latter obtain an LSD with galactic angular momentum $\Omega=\OSN$, where
$\OSN=25\kms\kpc^{-1}$ is the rate in the solar neighbourhood.
The former require $\Omega\geq4\OSN$ to excite LSD.
Rm applying at the largest scales would be 7.5 times higher for the latter 
model, so alone is sufficient to explain the LSD at lower $\Omega$.
%FAG: added
Early efforts by \citet{Korpi:1999b} were unable to detect even the LSD.
With a resolution even less than \citet{Gressel:2008} $\Omega\gg\OSN$ would be 
required to excite the LSD, and an SSD would be ruled out.
%
%FAG: added
For grid resolution $\dx\leq1\pc$ SSD cannot yet be excluded with 
$\eta\leq10^{-3}\kpc\kms$, 
unless the regime inducing the accelerated growth rates can be identified
and excluded from the SSD, which will be a focus of our continued investigation.
The higher $\eta$ required to otherwise exclude SSD, may also suppress LSD, for
modest $\Omega$.
%How the SSD affects the galactic dynamo requires further study.
%FAG added ???? Not sure this is actually so relevant having read it? Though I find it puzzling that they have SSD but no LSD
%Using an SPH code \citet{SBADMN19} report an SSD in a global galaxy simulation
%with minimum resolution equivalent to about 20\,pc.
%In this model the SN are not directly modelled and the turbulence is driven
%on a much larger forcing scale by the galactic wind and fountain.
%The disk height is relatively small-scale in the model, so the LSD model
%to which this letter is a preliminary step, might appropriately fit into
%design of a subgrid-scale model for their halo model.
%%

%FAG Moved from above
%As shown in Figure\,\ref{fig:eb-nu}(a) and (b), turbulence 
%can excite accelerated SSD even with high resistivity in galaxies that
%sustain regions of hot gas.
%To understand this we shall examine the nature of the acceleration of the
%dynamo in a later submission.





\acknowledgments
The authors wish to acknowledge CSC – IT Center for Science, Finland, for computational
resources.
 M-MML was partly supported by US NSF grant AST18-15461.
 FAG and MJK acknowledge the support of the Academy of Finland
ReSoLVE Centre of Excellence (grant number 307411).
This project has received funding from the European Research Council (ERC)
under the European Union's Horizon 2020 research and innovation
programme (Project UniSDyn, grant agreement n:o 818665).
\software{Pencil Code \citep{brandenburg2002,Pencil-JOSS}}

\bibliography{refs}{}
\bibliographystyle{aasjournal}

%FAG: omitting appendix - table not really necessar%FAG: omitting appendix - table not really necessary - moved to paper 2
%\appendix
%\fag{
%\begin{itemize}
%\item
%    Abstract – no more than 250 words - currently 220 0ut of 4289
%\item
%    Main Text – no more than 3500 words (not including appendices or other supplementary material)
%\item
%    Figures and Tables – no more than 5 combined figures (each limited to 9 panels) and tables, e.g. 3 figures and 2 tables.
%\item
%    References – no more than 50 references
%\end{itemize}}
%
%\section{Notation}\label{sec:table}
%
%\begin{table*}[h]
%%\begin{deluxetable*}{ccl}
%%\tablenum{A1}
%%\tablecaption{Notation\label{tab:notation}}
%%\tablewidth{0pt}
%%\tablehead{
%%\colhead{Notation Symbol} & \colhead{Denoting} & \colhead{Units/Definition} 
%%\\
%%}
%%\decimalcolnumbers
%%\startdata\\
%\begin{tabular}{ccl}
%\hline\hline\\
%{Notation Symbol} & {Denoting} & {Units/Definition}\\\hline\\
% $\dfrac{D}{Dt}$ & material derivative & $\dfrac{\partial }{\partial t}+\vect{u}\cdot \vect\nabla$ \\
% $\vect\nabla$ & gradient vector & e.g., $\left(\dfrac{\partial }{\partial x},\dfrac{\partial }{\partial y},\dfrac{\partial }{\partial z}\right)$ \\
% $\rho$ & gas density & [g cm$^{-3}$]  \\
% $\vect u$ & gas velocity & [km s$^{-1}$] \\
% $t$ & time & [Myr] \\
% $s$ & specific entropy & [erg g$^{-1}$ K$^{-1}$] \\
% $T$ & gas temperature & [K] \\
% $\vect A$ & magnetic vector potential & [$\upmu$G cm] \\
% $\vect B$ & magnetic field & [$\upmu$G] \\
% $\vect j$ & current density & [Bi cm$^{-2}$] \\
% $\mathbfss W$ & traceless rate of strain tensor &
%   ${\mathsf W}_{ij} = \dfrac{1}{2}\left(\dfrac{\partial u_i}{\partial x_j}
%                  + \dfrac{\partial u_j}{\partial x_i}
%                  -\dfrac{2}{3} \delta_{ij}\vect\nabla\cdot \vect u\right)$ \\
% $|\mathbfss W|^2$ & contracted rate of strain tensor &
%   $|\mathbfss W|^2={\mathsf W}_{ij}{\mathsf W}_{ij}$\\
% $\mathbfss W^{(3)}$ & 6th order rate of strain tensor &
%   ${\mathsf W}_{ij}^{(3)} = \dfrac{1}{2}\left(\dfrac{\partial^5 u_j}{\partial x_i^5}
%                  + \dfrac{\partial^4}{\partial x_i^4}\left(\dfrac{\partial u_i}{\partial x_j}\right)
%                  -\dfrac{1}{3}\dfrac{\partial^4}{\partial x_i^4}\left(\vect\nabla\cdot \vect u\right)\right)$ \\
% $\zeta_{D},\zeta_{\nu},\zeta_{\chi}$ & shock diffusion coefficient& $\propto \left(-\vect\nabla\cdot\vect u\right)_+$\\
% $\nu,\eta$ & viscosity, resistivity coefficient& [kpc km s$^{-1}$]\\
% $\nu_3,\chi_3,\eta_3$ & hyperdiffusion coefficient& [kpc$^{5}$ km s$^{-1}$]\\
% $\Gamma$ & UV-heating& [erg g$^{-1}$ s$^{-1}$]\\
% $\Lambda$ & radiative cooling& [erg cm$^{3}$ g$^{-2}$ s$^{-1}$]\\
% $\ESK+\EST$ & SN explosion energy& [10$^{51}$erg]\\
% $\dot\sigma$ & SN explosion rate & [kpc$^{-3}$ Myr$^{-1}$]\\
% $c_{\rm s}$ & sound speed & [km s$^{-1}$]\\
% $c_{\rm p}$ & specific heat at constant pressure & [erg g$^{-1}$ K$^{-1}$]\\
% $e_B$ & magnetic energy density & [erg cm$^{-3}$]\\
% $\overline{e_K}$ & time-averaged kinetic energy density & [erg cm$^{-3}$]\\
%\hline
%\end{tabular}
%%\enddata
%%\tablecomments{
%%Summary of notation used in the text.}
%%\end{deluxetable*}
%%\tablecomments{\fag{to be continued \ldots}
%%This table ``hides'' the third column in the \latex\ when compiled.
%%The Distance is also centered on the decimals.  Note that when using decimal
%%alignment you need to include the {\tt\string\decimals} command before
%%{\tt\string\startdata} and all of the values in that column have to have a
%%space before the next ampersand.}
%\end{table*}
%%\end{deluxetable*}
%
%%% This command is needed to show the entire author+affiliation list when
%%% the collaboration and author truncation commands are used.  It has to
%%% go at the end of the manuscript.
%%\allauthors
%
%%% Include this line if you are using the \added, \replaced, \deleted
%%% commands to see a summary list of all changes at the end of the article.
%%\listofchanges
\end{document}

